\chapter{A Tutorial Walkthrough}

This section, is designed to get you up and running designing your own
pictures without all the nit-picky details.  Once you've made a few
of your own data files, you'll probably want to advance to the next
section to fill in the gaps.

\section{The First Image}

Let's get right to the meat of the matter and create the data file for a
simple picture.  Since raytracers thrive on spheres, that's what we'll render
first.

First we have to create a viewpoint to tell the computer where our camera is
and where it's looking.  To do this, we use 3D coordinates.  The usual
coordinate system\index{coordinate system}
for DKBTrace has Y pointing up, X pointing to the right, and
Z pointing into the screen as shown in Figure \ref{axes}.

\begin{figure}[htbp]
\begin{centering}
\setlength{\unitlength}{0.0125in}%
\begin{picture}(95,112)(250,660)
\thinlines
\put(250,660){\vector( 1, 2){ 50}}
\put(260,680){\vector( 1, 0){ 80}}
\put(260,680){\vector( 0, 1){ 80}}
\put(300,740){\makebox(0,0)[lb]{\raisebox{0pt}[0pt][0pt]{\elvrm Z}}}
\put(260,765){\makebox(0,0)[lb]{\raisebox{0pt}[0pt][0pt]{\elvrm Y}}}
\put(345,680){\makebox(0,0)[lb]{\raisebox{0pt}[0pt][0pt]{\elvrm X}}}
\end{picture}

\caption{DKBTrace Coordinate Axes}
\label{axes}
\end{centering}
\end{figure}

Using your personal favorite text editor (i.e., user interface),
create a file called {\tt picture1.dat}.  Now, type in the
following:\footnote{The input is case sensitive, so be sure to
get capital and lowercase letters correct.}
\begin{verbatim}
      INCLUDE "colors.dat"
      INCLUDE "shapes.dat"
      INCLUDE "textures.dat"

      VIEW_POINT
         LOCATION  <0 0 0>
         DIRECTION <0 0 1>
         UP        <0 1 0>
         RIGHT     <1.33333 0 0>
      END_VIEW_POINT
\end{verbatim}
The first {\tt INCLUDE}\keyindex{INCLUDE}
statement reads in definitions for various
useful colours.  (Being a proud Canadian, I spell colour the proper
way with a ``u''.  To avoid confusing the rest of the world, however,
I've set up the raytracer to allow either spelling of the word.)

The second and third include statements read in some useful shapes and textures
respectively.  When you get a chance, have a look through them to see
but a few of the many possible shapes and textures available.

Include files may be nested, if you like.  The total pre-defined number of
{\tt INCLUDE}'d files (nested or not) per scene is 10.

Filenames specified in the {\tt INCLUDE} statements will be searched
for in the home (current) directory first, and if not found, will then
be searched for in directories specified by any {\tt -l}\optindex{l}
(library path) options active.  This would facilitate keeping all your
"include" ({\tt .inc}) files, {\tt shapes.dat}, {\tt colors.dat}, and
{\tt textures.dat} in an ``include'' subdirectory, and giving an
{\tt -l} option on the command line to where your library of include files
are.

This viewpoint declaration puts our camera at the center of the
universe ({\tt LOCATION <0 0 0>}) pointing into the Z direction
({\tt DIRECTION <0 0 1>}) and with the camera being held upright
({\tt UP <0 1 0>}).  The final term compensates for the aspect ratio
of the screen
({\tt RIGHT <1.33333 0 0>}).  If your computer has square pixels, you
may want to change this to ``{\tt RIGHT <1.0 0 0>}''.  For details on
exactly how the camera works, see the section ``How it All Works''.

Now, let's place a red sphere into the world:
\begin{verbatim}
      OBJECT
         SPHERE <0 0 3> 1 END_SPHERE
         TEXTURE
            COLOUR Red
         END_TEXTURE
      END_OBJECT
\end{verbatim}
This sphere is 3 units away from the camera and has a radius of 1.  Note that
any parameter that changes the appearance of the surface (as opposed to the
shape of the surface) is called a texture parameter and {\em must} be
placed into a
\verb#TEXTURE#-\verb#END_TEXTURE# block.  In this case, we are just
setting the colour.

One more detail -- we need a light source:
\begin{verbatim}
      OBJECT
         SPHERE <0 0 0> 1 END_SPHERE
         TEXTURE
            COLOUR White
         END_TEXTURE
                             { This is 2 units to our right, }
         TRANSLATE <2 4 -3>  { 4 units above, and 3 units    }
                             { behind our camera.            }
         LIGHT_SOURCE
         COLOUR White
      END_OBJECT
\end{verbatim}
Note: For light sources, {\em always} declare them to be centered at
the origin {\tt <0 0 0>}, then use {\tt TRANSLATE} to put them where
you want.\index{light sources!and {\tt TRANSLATE}}
If you don't do this, the light source won't work right. We
must also specify the colour of the light source {\em outside} the
{\tt TEXTURE} block because the renderer doesn't want to work out the
whole surface colour just to get the colour of the light it
emits.\index{light sources!and {\tt TEXTURE}}

That's it!  Close the file and render a small picture of
it:\footnote{The program name may vary on different systems.}
\begin{verbatim}
      trace -w80 -h100 -f -ipicture1.dat
\end{verbatim}
On the IBM, the command line would be:\footnote{The program name would
be {\tt dkbno87} if you have an 8086/8088 system or no math co-processor.}
\begin{verbatim}
      dkb -w80 -h50 -f -ipicture1.dat
\end{verbatim}

\section{Phong Highlights}

You've now rendered your first picture.  I know you want to run out and show
all your friends how amazing your computer is to be able to generate such an
incredible picture, but just wait a few minutes -- you ain't seen nothin' yet.
(For those people who complained that the picture took too long to draw, just
wait -- you ain't seen nothin' yet, either\ldots)

Let's add a nice little specular highlight\index{specular highlight}
(shiny spot) to the sphere.  It
gives it that neat ``computer graphics'' look.  Change the definition of the
sphere to this:
\begin{verbatim}
      OBJECT
         SPHERE <0 0 3> 1 END_SPHERE
         TEXTURE
            COLOUR Red
            PHONG 1.0
         END_TEXTURE
      END_OBJECT
\end{verbatim}
Now render this.  In all seriousness, the {\tt PHONG} highlight does
add a lot of credibility to the picture.  You'll probably want to use
it in many of your pictures.

\section{Textures}

One of the really nice features of this raytracer is its sophisticated
textures.  Change the definition of our sphere to the following and then
re-render it:
\begin{verbatim}
      OBJECT
         SPHERE <0 0 3> 1 END_SPHERE
         TEXTURE
            Dark_Wood
            SCALE <0.2 0.2 0.2>
            PHONG 1.0
         END_TEXTURE
      END_OBJECT
\end{verbatim}
The textures are set up by default to give you one ``feature'' across
a sphere of radius 1.0.  A ``feature'' is roughly defined as a colour
transition.  For example, a wood texture would have one band on a
sphere of radius 1.0.  By scaling the wood by {\tt <0.2 0.2 0.2>}, we
shrink the texture to give us about five bands. Please note that this
is not a hard and fast rule.  It's only meant to give you a rough idea
for the scale to use for a texture.  Don't start reporting problems if
you get three bands instead of five.  This rule of thumb just puts you
in the ballpark.

One note about the {\tt SCALE}\keyindex{SCALE} operation.  You can
magnify or shrink
along each direction separately.  The first term tells how much to
magnify or shrink in the left-right direction.  The second term
controls the up-down direction and the third term controls the
front-back direction.

I encourage you to look through the {\tt textures.dat} file to see
what textures are defined there and try them out.  Some of them are
quite spectacular.

\section{Other Shapes}

So far, we've just defined spheres.  There are several other kinds of
shapes that can be rendered by DKBTrace.  Let's try one out with a
computer graphics standard -- a checkered floor.  Add the following
object to your {\tt .dat} file:
\begin{verbatim}
      OBJECT
         PLANE <0.0 1.0 0.0> -1.0 END_PLANE
         TEXTURE
            CHECKER
               COLOUR RED 1.0
               COLOUR BLUE 1.0
         END_TEXTURE
      END_OBJECT
\end{verbatim}
The object defined here is an infinite plane.  The vector
{\tt <0.0 1.0 0.0>} is the surface normal of the plane (i.e., if you where
standing on the surface, the normal points straight up.)  The number
afterward is the distance that the plane is displaced along the normal
-- in this case, we move the floor down one unit so that the sphere
(radius 1) is resting on it.  The checker texture specifies the two
colours to use in the checker pattern.

Looking at the floor, you'll notice that the wooden ball casts a shadow on the
floor.  Shadows are calculated accurately (well, almost -- more later) by the
raytracer.

Another kind of shape you can use is called a quadric surface.  To be totally
honest, the shapes you've been using so far have been quadrics.  Spheres and
planes are types of quadric surfaces.  There are many other quadric surfaces
however.  These are all described by a certain kind of mathematical formula
(see the section on Quadrics in the next chapter).  They include cylinders,
cones, paraboloids (like a satellite dish), hyperboloids (saddle-shaped) and
ellipsoids as well as the spheres and planes we've used so far.

All quadrics except for ellipsoids and spheres are infinite in at least one
direction.  For example, a cylinder has no top or bottom - it goes to infinity
at each end.  Quadrics all have one common feature - if you draw any straight
line through a quadric, it will hit the surface at most twice.  A torus
(donut), for example, is not a quadric since a line can hit the surface up to
four times going through.

Enough talk -- let's render one of these ``quadrics''\ldots While
we're at it, we'll add a few features to the surface.  Add the
following definition to your {\tt .dat} file:
\begin{verbatim}
      OBJECT
         QUADRIC Cylinder_Y END_QUADRIC
         TEXTURE
            COLOUR GREEN 0.5
            REFLECTION 0.5
         END_TEXTURE
         SCALE <0.4  0.4  0.4>
         TRANSLATE <2 0 5>
      END_OBJECT
\end{verbatim}
This object is a cylinder along the Y (up-down) axis.  It's green in colour
and has a mirrored surface (hence the reflection of 0.5) this means that half
the light coming from the sphere is reflected from other objects in the room.
A reflection of 1.0 is a perfect mirror.

The object has been shrunk by scaling it by {\tt <0.4 0.4 0.4>}.  Note
that since the cylinder is infinite along the Y axis, the middle term
is kind of pointless.  One four tenths of infinity is still infinity.
(Don't use 0, though.  That will definitely cause a fatal crash!)
Finally, the cylinder has been moved back and to the right so you can
see it more clearly.

\section{Constructive Solid Geometry}

The shapes we've talked about so far are nice, but not terribly useful on
their own for making realistic scenes.  It's hard to make interesting objects
when you're limited to spheres, infinite cylinders, infinite planes, and so
forth.  

Constructive Solid Geometry (CSG) is a technique for taking these simple
building blocks and combining them together.  You can use a cylinder to bore a
hole through a sphere.  You can use planes to cap cylinders and turn them into
flat circular disks (that are no longer infinite).

Before getting into CSG, however, let me talk about inside and
outside.\index{inside, of objects}\index{outside, of objects}%
\index{objects!inside and outside}
Every
primitive (except triangles -- I'll talk about this later) divides the world
into two regions.  One region is inside the surface and one is outside.  So,
given any point in space, you can say it's either inside or outside any
particular primitive object (well, it could be exactly on the surface, but
usually numerical inaccuracies will put it to one side or the other).  Even
planes have an inside and an outside.  By definition, the surface normal of
the plane points towards the outside of the plane.  (For a simple floor, for
example, the space above the floor is ``outside'' and the space below the floor
is ``inside''.  For simple floors this in unimportant, but for planes as parts
of CSG's it becomes much more important).

CSG uses the concepts of inside and outside to combine shapes
together. Consider the situation in Figure \ref{csg0}.\footnote{The
diagrams shown here demonstrate the concepts in 2D and are indended
only as an analogy to the 3D case.  Note that the triangles supported
by DKBTrace cannot be used in CSG (except for unions) since they have
no inside and outside.}\index{triangle!inside and outside}

\begin{figure}[htbp]
\begin{centering}
\setlength{\unitlength}{0.0125in}%
\begin{picture}(200,80)(80,680)
\thicklines
\put( 80,725){\line( 1, 0){ 10}}
\thinlines
\put( 80,745){\line( 1, 0){ 10}}
\thicklines
\put(240,740){\line(-2,-3){ 40}}
\put(200,680){\line( 1, 0){ 80}}
\put(280,680){\line(-2, 3){ 40}}
\thinlines
\put(200,760){\line(-3,-4){ 45}}
\put(155,700){\line( 1, 0){ 85}}
\put(240,700){\line(-2, 3){ 40}}
\put(100,720){\makebox(0,0)[lb]{\raisebox{0pt}[0pt][0pt]{\elvrm Figure B}}}
\put(100,740){\makebox(0,0)[lb]{\raisebox{0pt}[0pt][0pt]{\elvrm Figure A}}}
\end{picture}

\caption{Sample 2D figures}
\label{csg0}
\end{centering}
\end{figure}

There are three CSG operations you can use:

\begin{description}
\item[{\tt UNION A B END_UNION}] \keyindex{UNION}A point is inside
the union if it is
inside A {\em or} it's inside B (or both).  This gives an ``additive''
effect to the component objects, as shown in Figure \ref{csg1}.

\begin{figure}[htbp]
\begin{centering}
\setlength{\unitlength}{0.0125in}%
\begin{picture}(120,80)(160,680)
\thicklines
\put(200,680){\line( 3, 4){ 15}}
\put(240,740){\line(-1,-2){ 10}}
\put(280,680){\line(-2, 3){ 40}}
\put(200,680){\line( 1, 0){ 80}}
\thinlines
\put(200,760){\line( 3,-4){ 30}}
\put(160,700){\line( 1, 0){ 55}}
\put(200,760){\line(-2,-3){ 40}}
\end{picture}

\caption{{\tt UNION} of the Sample Figures}
\label{csg1}
\end{centering}
\end{figure}

\item[{\tt INTERSECTION A B END_INTERSECTION}] \keyindex{INTERSECTION}A
point is inside the
intersection if it's inside both A {\em and} B.  This ``logical
AND's'' the shapes and gets the common part, most useful for
``clipping'' infinite shapes off, etc., as shown in Figure \ref{csg2}.

\begin{figure}[htbp]
\begin{centering}
\setlength{\unitlength}{0.0125in}%
\begin{picture}(25,20)(215,700)
\thinlines
\put(240,700){\line(-1, 0){ 25}}
\put(230,720){\line( 1,-2){ 10}}
\thicklines
\put(230,720){\line(-3,-4){ 15}}
\end{picture}

\caption{{\tt INTERSECTION} of the Sample Figures}
\label{csg2}
\end{centering}
\end{figure}

\item[{\tt DIFFERENCE A B END_DIFFERENCE}] \keyindex{DIFFERENCE}A point
is inside the
difference if it's inside A but not inside B.  The result is a
``subtraction'' of the 2nd shape from the first shape, shown in Figure
\ref{csg3}.

\begin{figure}[htbp]
\begin{centering}
\setlength{\unitlength}{0.0125in}%
\begin{picture}(75,60)(155,700)
\thicklines
\put(230,720){\line(-1,-1){ 20}}
\thinlines
\put(200,760){\line(-3,-4){ 45}}
\put(200,760){\line( 3,-4){ 30}}
\put(155,700){\line( 1, 0){ 55}}
\end{picture}

\caption{{\tt DIFFERENCE} of the Sample Figures}
\label{csg3}
\end{centering}
\end{figure}
\end{description}

Let's give a concrete example by drilling a yellow hole through our sphere.
Go to the definition of the sphere and change it to read the following:
\begin{verbatim}
      OBJECT
         DIFFERENCE
            SPHERE <0 0 3> 1 END_SPHERE
            QUADRIC
               Cylinder_Z
               SCALE <0.2 0.2 0.2>
               COLOUR Yellow
            END_QUADRIC
         END_DIFFERENCE
         TEXTURE
            Dark_Wood
            SCALE <0.2 0.2 0.2>
            PHONG 1.0
         END_TEXTURE
      END_OBJECT
\end{verbatim}

One more point about CSG operations.  You can flip a shape inside-out
by putting the keyword {\tt INVERSE}\keyindex{INVERSE} into the
shape's definition.
This keyword will not change the appearance of the shape unless you're
using CSG.  In the case of CSG, it gives you more flexibility.  For
example, the result of
\begin{verbatim}
   INTERSECTION B A-INVERSE END_INTERSECTION
\end{verbatim}
yields the shape shown in Figure \ref{csg4}.
\begin{figure}[htbp]
\begin{centering}
\setlength{\unitlength}{0.0125in}%
\begin{picture}(80,60)(200,680)
\thinlines
\put(240,700){\line(-1, 2){ 10}}
\put(215,700){\line( 1, 0){ 25}}
\thicklines
\put(200,680){\line( 3, 4){ 15}}
\put(240,740){\line(-1,-2){ 10}}
\put(280,680){\line(-2, 3){ 40}}
\put(200,680){\line( 1, 0){ 80}}
\end{picture}

\caption{{\tt INVERSE-DIFFERENCE} of the Sample Figures}
\label{csg4}
\end{centering}
\end{figure}
Note that a {\tt DIFFERENCE} is really just an {\tt INTERSECTION} of
one shape with the {\tt INVERSE} of another.  This happens to be how
{\tt DIFFERENCE} is actually implemented in the code.

\section{Advanced Textures}

The textures available in DKBTrace are 3D solid textures.  This means that the
texture defines a colour for any 3D point in space.  Just like a real block of
marble or wood, there is colour all through the block -- you just can't see it
until you carve away the wood or marble that's in the way.  Similarly, with a
3D solid texture, you don't see all the colours in the texture -- you only see
the colours that happen to be visible at the surface of the object.

As you've already seen, you can scale, translate, and rotate
textures\index{textures!animation of}%
\index{animation|see{textures, animation of}}.
In fact, you could make an animation in which the objects stay still and the
textures translate and rotate through the object.  The effect would be like
watching a time-lapse film of a cloudy sky -- the clouds would not only move,
but they would also change shape smoothly.

Often, textures are perturbed by noise.  This ``turbulence''
\index{textures!and {\tt TURBULENCE}}\index{perturbation}%
\index{noise}\index{turbulence}
distorts the
texture so it doesn't look quite so perfect.  Try changing the sphere in the
above example to have the following texture:
\begin{verbatim}
         TEXTURE
            Dark_Wood
            TURBULENCE 0.0
            SCALE <0.2 0.2 0.2>
            PHONG 1.0
         END_TEXTURE
\end{verbatim}
When you compare this with the original image, you'll see that the pattern is
much more boring.

Finally, many textures use colour maps.\index{textures!colour maps}
A colour map translates a number
between 0.0 and 1.0 into a colour.  The number typically represents the
distance into a vein of colour -- the further into the vein you get, the more
the colour changes.  Here's a typical colour map.  Try this out on the sphere
defined above by changing the definition to this:
\begin{verbatim}
      OBJECT
         SPHERE <0 0 3> 1 END_SPHERE
         TEXTURE
            WOOD
            SCALE <0.2 0.2 0.2>
            COLOUR_MAP
                [0.0 0.3  COLOUR Red   COLOUR Green]
                [0.3 0.6  COLOUR Green COLOUR Blue]
                [0.6 1.01 COLOUR Blue  COLOUR Red]
            END_COLOUR_MAP
            PHONG 1.0
         END_TEXTURE
      END_OBJECT
\end{verbatim}
This means that as the texture enters into a vein of wood, it changes colour
smoothly from red to green, from green to blue, and from blue to red again.
As it leaves the vein, the transition occurs in reverse.  (Since there is no
turbulence on the wood by default, the veins of colour should show up quite
well.)

You can get more ``bang for your buck'' from textures by using
{\tt ALPHA}\keyindex{ALPHA} properties of colour.  Every
colour you define in DKBTrace is a
combination of red, green, blue and alpha.  The red, green and blue
are simple enough.  The alpha determines how transparent that colour
is.  A colour with an alpha of 1.0 is totally transparent.  A colour
with an alpha of 0.0 is totally opaque.  Here's a neat texture to try:
\begin{verbatim}
   TEXTURE
      TURBULENCE 0.5
      BOZO
      COLOUR_MAP
         { transparent to transparent }
         [0.0 0.6 COLOUR RED 1.0 GREEN 1.0 BLUE 1.0 ALPHA 1.0
                  COLOUR RED 1.0 GREEN 1.0 BLUE 1.0 ALPHA 1.0]
         { transparent to white }
         [0.6 0.8 COLOUR RED 1.0 GREEN 1.0 BLUE 1.0 ALPHA 1.0
                  COLOUR RED 1.0 GREEN 1.0 BLUE 1.0]
         { white to grey }
         [0.8 1.001 COLOUR RED 1.0 GREEN 1.0 BLUE 1.0
                    COLOUR RED 0.8 GREEN 0.8 BLUE 0.8]
      END_COLOUR_MAP
      SCALE <0.4  0.08  0.4>
   END_TEXTURE
\end{verbatim}
This is my (famous) cloud texture.\index{textures!famous cloud}
It creates white clouds with grey linings.
The texture is transparent in the places where the clouds disappear so you can
see through it to the objects that are behind.

Now for one more feature which is new for 2.10 (hold onto your hats!)  You can
now layer textures one on top of another to create more sophisticated textures.
\index{textures!layered}
For example, suppose I want a wood-coloured cloudy texture.  What I do is put
the wood texture down first followed by my cloud texture.  Wherever the cloud
texture is transparent, the wood texture shows through.  Change your sphere to
the following and you'll see.
\begin{verbatim}
  OBJECT
     SPHERE <0 0 3> 1 END_SPHERE
     TEXTURE                { This is the wood texture }
        Dark_Wood           { we used earlier.         }
        TURBULENCE 0.0
        SCALE <0.2 0.2 0.2>
        PHONG 1.0
     END_TEXTURE
     TEXTURE                { This is the cloud texture }
       TURBULENCE 0.5       { we just defined.          }
       BOZO
       COLOUR_MAP
         { transparent to transparent }
         [0.0 0.6 COLOUR RED 1.0 GREEN 1.0 BLUE 1.0 ALPHA 1.0
                  COLOUR RED 1.0 GREEN 1.0 BLUE 1.0 ALPHA 1.0]
         { transparent to white }
         [0.6 0.8 COLOUR RED 1.0 GREEN 1.0 BLUE 1.0 ALPHA 1.0
                  COLOUR RED 1.0 GREEN 1.0 BLUE 1.0]
         { white to grey }
         [0.8 1.001 COLOUR RED 1.0 GREEN 1.0 BLUE 1.0
                    COLOUR RED 0.8 GREEN 0.8 BLUE 0.8]
       END_COLOUR_MAP
       SCALE <0.4  0.08  0.4>
     END_TEXTURE
  END_OBJECT
\end{verbatim}
Each successive texture is layered on top of the previous textures.  In the
places where you can see through the upper texture, you see through to the
lower textures.

\section{Walk-through Wrap-up}

In this walk-through, I've only tried to show the highlights of the raytracer
without getting into all possible options and features.  To get all of those,
you'll have to read through the following section on the Scene Description
Language.  Hopefully it will be fairly straight forward now that you have a
feel for the language and how it works.  Hopefully you'll find that the
textual interface provided by DKBTrace isn't quite as scary as people think at
first.
