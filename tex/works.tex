\chapter{How it All Works (or How to Get What You Want)}

The information in this section is designed for people who are reasonably
familiar with the raytracer and want more information on how things work so
they can push it to its limits.  You probably don't need this level of detail
to make interesting data files, but if you suddenly get confused about
something the program did, this section may help you figure it out.

\section{Viewpoints}

\index{viewpoint}
Viewpoints can be completely defined by four vectors.  The
{\tt LOCATION}\keyindex{LOCATION} is easy.  That's where the camera
is. The {\tt DIRECTION}\keyindex{DIRECTION}
is a vector that starts at the {\tt LOCATION} and points to the center
of a window.  The {\tt UP}\keyindex{UP} vector starts at the center
of the window and points to the center of the top edge.  The
{\tt RIGHT}\keyindex{RIGHT} vector
starts at the center of the window and points to the center of the
right edge. These vectors are illustrated in Figure \ref{vectors}.
 
\begin{figure}[htbp]
\begin{centering}
\setlength{\unitlength}{0.0125in}%
\begin{picture}(181,155)(54,605)
\thinlines
\put(100,680){\circle*{10}}
\put(205,680){\vector( 1,-1){ 20}}
\put(205,680){\vector( 0, 1){ 35}}
\put(100,680){\vector( 1, 0){105}}
\put(180,760){\line( 0,-1){100}}
\put(180,660){\line( 1,-1){ 55}}
\put(235,605){\line( 0, 1){100}}
\put(235,705){\line(-1, 1){ 55}}
\put( 90,675){\makebox(0,0)[rb]{\raisebox{0pt}[0pt][0pt]{\elvrm Location}}}
\put(120,665){\makebox(0,0)[lb]{\raisebox{0pt}[0pt][0pt]{\elvrm Direction}}}
\put(205,650){\makebox(0,0)[lb]{\raisebox{0pt}[0pt][0pt]{\elvrm Right}}}
\put(210,695){\makebox(0,0)[lb]{\raisebox{0pt}[0pt][0pt]{\elvrm Up}}}
\end{picture}

\caption{Viewpoint definition vectors}
\label{vectors}
\end{centering}
\end{figure}

The window is then divided up according to the resolution you specified and
rays are fired through the pixels out into the world.  For an eye ray,
therefore, the equation of the ray is:
\begin{displaymath}
{\tt LOCATION} + t ({\tt DIRECTION} + ((height - y) \cdot {\tt UP}) +
(x \cdot {\tt RIGHT}))
\end{displaymath}
where $t$ is a parameter that determines the distance from the eye to the
object being tested.  The\index{coordinate system!inverted Y} Y
coordinate is inverted by subtracting it from height because most
graphics systems put $(0,0)$ in the top left corner of the screen.

This viewpoint model is very flexible.  It allows you to use
left-handed or right-handed coordinate systems.  It also doesn't
require that the {\tt DIRECTION}, {\tt UP}, and {\tt RIGHT} vectors be
mutually orthogonal.  If you want, you can distort the camera to get
really bizarre results.

Once the basic four vectors are specified, it's possible to use the
{\tt SKY}\keyindex{SKY} and {\tt LOOK_AT}\keyindex{LOOK_AT} vectors
to point the camera.  You must
specify the {\tt SKY} vector first, but let me describe the
{\tt LOOK_AT} vector first.  {\tt LOOK_AT} tells the camera to rotate in
such a way that the {\tt LOOK_AT} point appears in the center of the
screen.  To do this, the camera first turns in the left-to-right
direction (longitude in Earth coordinates) until it's lined up with
the {\tt LOOK_AT} point.  It then turns in the up/down direction (latitude
in Earth coordinates) until it's looking at the desired point.

Ok, now we're looking at the proper point.  What else do we have to
specify?  If you're looking at a point, you can still turn your camera
sideways and still be looking at the same spot.  This it the
orientation that the {\tt SKY} direction determines.  The camera will
try to position itself so that the camera's {\tt UP} direction lines
up as closely as possible to the {\tt SKY} direction.

Put another way -- in airplane terms, the {\tt LOOK_AT} vector
determines your heading (north, south, east, or west), and your pitch
(climbing or descending).  The {\tt SKY} vector determines your
banking angle.

\section{Ray-Object Intersections}

For every pixel on the screen, the raytracer fires at least one ray through
that pixel into the world to see what it hits.  For each hit (well, almost),
it calculates rays to each of the light sources to see if that point is
shadowed from that light source.  For reflecting objects, a reflected ray is
traced.  For refracting objects, a refracting ray is traced.  That all adds up
to a lot of rays.

Every ray is tested against every object in the world to see if the ray hits
that object. This is what slows down the raytracer.  You can make things
easier by using simple bounding shapes on your
objects\index{processing speed}.

Fortunately, all ray-object intersections for all shapes in DKBTrace can be
solved by a simple quadratic equation.  This is why {\tt QUADRIC}s are used in
DKBTrace.  Solving for things like B-Splines, Bezier Splines, NURBS, Tori,
etc. is a lot more complicated.  That's why I haven't implemented primitives
for these shapes.

\section{Transparency and Refraction}

This section gets really complicated because of the way transparency and
refraction are implemented.  If you don't really care, skip to the next
section. If you don't mind slogging through this and getting confused, then
read on -- you've been warned.

The way that transparency and refraction work has changed slightly from
previous versions.  Now, transparency and refraction work together instead of
separately.

First\index{light!reflected vs. filtered}, let me distinguish between
reflected light and filtered light.  Suppose
you painted a table with patches of colour.  You then took some red sand and
sprinkled it on top of the various colours.  The red sand will tint the
colours red, but you will still see some of the original colour.  If, instead,
you took a sheet of red plexiglass and put it on top of the table, all you
could see would be shades of red.  That's because the plexiglass filter
{\em only} allows the red colour to show through.

In DKBTrace, the layered textures\index{textures!layered} work like
the red sand -- the colours on top mix with the underlying colours
depending on the density ({\tt ALPHA}\keyindex{ALPHA}) of the
distribution.

Refraction\index{refraction}, however, works like a filter.  The
surface colour determines that
colours of light are allowed to pass from the inside of the object to the
outside, and vice versa.  Here are some filter colours:
\begin{verbatim}
   { a red filter }
   RED 1.0  GREEN 0.0  BLUE 0.0   ALPHA 1.0
   { a clear glass }
   RED 1.0  GREEN 1.0  BLUE 1.0   ALPHA 1.0
   { a dark filter - this will appear black }
   RED 0.0  GREEN 0.0  BLUE 0.0   ALPHA 1.0
\end{verbatim}
Now, consider the following layered textures:\keyindex{REFRACTION}
\begin{verbatim}
   TEXTURE COLOUR Green ALPHA 0.6 END_TEXTURE
   TEXTURE COLOUR Yellow ALPHA 0.3 REFRACTION 0.5 END_TEXTURE
\end{verbatim}
Keep in mind that the last texture is on top.\footnote{In layered
textures, only the {\tt REFRACTION} component of the {\em last} entry
has any effect.}
The colour you get is calculated as shown in Figure \ref{layers}.
\begin{figure}[htbp]
\begin{centering}
\setlength{\unitlength}{0.0125in}%
\begin{picture}(140,154)(60,640)
\thinlines
\put(140,710){\vector( 4, 3){ 40}}
\put(140,680){\vector( 1,-1){ 40}}
\put(180,700){\line( 1, 0){ 20}}
\put(200,740){\line(-1, 0){ 20}}
\put(180,740){\line( 0,-1){100}}
\put(180,640){\line( 1, 0){ 20}}
\put( 60,710){\line( 1, 0){ 80}}
\put( 80,780){\line(-1, 0){ 20}}
\put( 60,780){\line( 0,-1){100}}
\put( 60,680){\line( 1, 0){ 80}}
\put(200,715){\makebox(0,0)[lb]{\raisebox{0pt}[0pt][0pt]{\elvrm 4/10 Green}}}
\put( 80,690){\makebox(0,0)[lb]{\raisebox{0pt}[0pt][0pt]{\elvrm 3/10 (alpha)}}}
\put( 80,740){\makebox(0,0)[lb]{\raisebox{0pt}[0pt][0pt]{\elvrm 7/10 is Yellow}}}
\put( 60,665){\makebox(0,0)[lb]{\raisebox{0pt}[0pt][0pt]{\elvrm Full darkness}}}
\put( 60,785){\makebox(0,0)[lb]{\raisebox{0pt}[0pt][0pt]{\elvrm Full brightness}}}
\put(200,680){\makebox(0,0)[lb]{\raisebox{0pt}[0pt][0pt]{\elvrm 6/10 (alpha) left for refraction}}}
\put(200,666){\makebox(0,0)[lb]{\raisebox{0pt}[0pt][0pt]{\elvrm of which 5/10 refracted filtered}}}
\put(200,652){\makebox(0,0)[lb]{\raisebox{0pt}[0pt][0pt]{\elvrm by yellow filtered by green}}}
\end{picture}

\caption{Layered color calculation}
\label{layers}
\end{centering}
\end{figure}
The top texture layer supplies some fraction of the light that
reflects off the surface of the object.  If the {\tt ALPHA} was
non-zero, it allows the lower textures to supply the remainder.  If,
after all the textures are processed, there's still some fraction left
over, it is applied to the light that's refracted through the object.

This algorithm probably doesn't coincide with reality, but neither
does the rest of the raytracer, so I'm not terribly concerned about
it.

\section{Textures, Noise, and Turbulence}

\index{textures}
If there's one thing that DKBTrace is known for, it's textures.  Here's how
they really work.  If you want some good reading material, check out ``An Image
Synthesizer'' by Ken Perlin in the SIGGRAPH '84 Conference Proceedings.

Let's start with a marble texture\index{textures!marbled}.  Real marble
is created when different colours of sediments are laid down one on
top of another and compressed to form solid rock.

For example, take the simple block of marble shown in Figure \ref{marble}.
\begin{figure}[htbp]
\begin{centering}
\setlength{\unitlength}{0.0125in}%
\begin{picture}(80,100)(160,620)
\thinlines
\put(160,640){\line( 1, 0){ 60}}
\put(220,640){\line( 1, 1){ 20}}
\put(160,660){\line( 1, 0){ 60}}
\put(220,660){\line( 1, 1){ 20}}
\put(160,680){\line( 1, 0){ 60}}
\put(220,680){\line( 1, 1){ 20}}
\put(240,720){\line( 0,-1){ 80}}
\put(240,640){\line(-1,-1){ 20}}
\put(160,700){\line( 0,-1){ 80}}
\put(160,620){\line( 1, 0){ 60}}
\put(220,620){\line( 0, 1){ 80}}
\put(160,700){\line( 1, 0){ 60}}
\put(220,700){\line( 1, 1){ 20}}
\put(240,720){\line(-1, 0){ 60}}
\put(180,720){\line(-1,-1){ 20}}
\put(190,625){\makebox(0,0)[b]{\raisebox{0pt}[0pt][0pt]{\elvrm White}}}
\put(190,645){\makebox(0,0)[b]{\raisebox{0pt}[0pt][0pt]{\elvrm Red}}}
\put(190,665){\makebox(0,0)[b]{\raisebox{0pt}[0pt][0pt]{\elvrm White}}}
\put(190,685){\makebox(0,0)[b]{\raisebox{0pt}[0pt][0pt]{\elvrm Red}}}
\end{picture}

\caption{Marble texture}
\label{marble}
\end{centering}
\end{figure}
If you carve a shape put of this block of marble, you will get red and white
bands across it.

Now, consider wood\index{textures!wood}.  The rings in wood are
created when the tree grows a new outer shell every year.  Hence, we
have concentric cylinders of colours, as shown in Figure \ref{log}.
\begin{figure}[htbp]
\begin{centering}
\setlength{\unitlength}{0.0125in}%
\begin{picture}(85,85)(200,720)
\thinlines
\put(220,740){\circle{10}}
\put(220,740){\circle{20}}
\put(220,740){\circle{30}}
\put(220,740){\circle{40}}
\put(235,725){\line( 1, 1){ 50}}
\put(205,755){\line( 6, 5){ 60}}
\end{picture}

\caption{Wood texture}
\label{log}
\end{centering}
\end{figure}
Cutting a shape out of a piece of wood will tend to give you rings of colour.

Now, this is fine, but the textures are still a bit boring.  For the next
step, we blend the colours together to create a nice smooth transition.  This
makes the texture look a bit better.  The problem, though, is that it's too
regular -- real marble and wood aren't so perfect.

Before we make our wood and marble look any better, let's look at how we make
noise.  Noise\index{noise} (in raytracing) is sort of like a random
number generator, but
it has the following properties:
\begin{enumerate}
\item It's defined over 3D space i.e., it takes x, y, and z and returns the
noise value there.
\item If two points are far apart, the noise values at those points are
relatively random.
\item If two points are close together, the noise values at those points are
close to each other.
\end{enumerate}
You can visualize this as having a large room and a thermometer that ranges
from 0.0 to 1.0.  Each point in the room has a temperature.  Points that are
far apart have relatively random temperatures.  Points that are close together
have close temperatures. The temperature changes smoothly, but randomly as we
move through the room.

Now, let's place an object into this room along with an artist.  The artist
measures the temperature at each point on the object and paints that point a
different colour depending on the temperature.  What do we get?
{\tt BOZO}\keyindex{BOZO}\index{textures!noisy} texture!

Another function used in texturing is called DNoise.  This is sort of like
noise except that instead of giving a temperature, it gives a direction.  You
can think of it as the direction that the wind is blowing at that spot.

Finally, we have a function called turbulence\index{turbulence} which
uses DNoise to push a particle around a few times -- each time going
half as far as before. This procedure is roughly illustrated in
Figure \ref{turb}.
\begin{figure}[htbp]
\begin{centering}
\setlength{\unitlength}{0.0125in}%
\begin{picture}(154,117)(91,655)
\thinlines
\put(200,700){\vector( 1,-1){ 40}}
\put(220,700){\vector(-1, 0){ 20}}
\put(240,760){\vector(-1,-3){ 20}}
\put(160,760){\vector( 1, 0){ 80}}
\put(240,720){\makebox(0,0)[lb]{\raisebox{0pt}[0pt][0pt]{\elvrm Second move}}}
\put(170,765){\makebox(0,0)[lb]{\raisebox{0pt}[0pt][0pt]{\elvrm First move}}}
\put(245,655){\makebox(0,0)[lb]{\raisebox{0pt}[0pt][0pt]{\elvrm Final position}}}
\put(155,755){\makebox(0,0)[rb]{\raisebox{0pt}[0pt][0pt]{\elvrm Initial Position}}}
\end{picture}

\caption{Effect of turbulence}
\label{turb}
\end{centering}
\end{figure}
This is what we use to create the ``interesting'' marble and wood texture.  We
locate the point we want to colour (P), then push it around a bit using
Turbulence to get to a final point (Q) then look up the colour of point Q in
our ordinary boring wood and marble textures.  That's the colour that's used
for the point P.

\section{Layered Textures}

\index{textures!layered}
As of version 2.10, DKBTrace supports layered textures.  Here's how
that works.  Each object and each shape has a texture that may be
attached to it.  By default, shapes have no texture, but objects have
a default texture.  Internally, textures are marked as being constant
or variable.  A constant texture is one that was
{\tt DECLARE}'d\keyindex{DECLARE} as a
texture and is being shared by many shapes and objects.  Variable
textures are textures that have been declared totally within the
object or have used a {\tt DECLARE}'d texture and modified it in a
destructive way.  The idea here is that we want to save on memory by
sharing textures if possible.

If you have several texture blocks for an object or a shape, they are placed
into a linked list (First-in, Last-out)\index{textures!parsing of}.
For example, take the following
definition:
\begin{verbatim}
   OBJECT
      SPHERE <0 0 0> 1 END_SPHERE
      TEXTURE WOOD END_TEXTURE
      TEXTURE MARBLE END_TEXTURE
   END_OBJECT
\end{verbatim}
Here's what happens while parsing this object: Since this is an object
(as opposed to a shape -- {\tt SPHERE}, {\tt PLANE}, etc.), it starts
out with the default texture attached, as shown in Figure \ref{parse1}.
\begin{figure}[htbp]
\begin{centering}
\input{parse1}
\caption{Object parsing, step 1}
\label{parse1}
\end{centering}
\end{figure}
When the parser sees the first {\tt TEXTURE} block, it looks to see what it has
linked.  Since the thing that's linked is the default texture (not a copy), it
discards it and puts in the new texture, as shown in Figure \ref{parse2}.
\begin{figure}[htbp]
\begin{centering}
\input{parse2}
\caption{Object parsing, step 2}
\label{parse2}
\end{centering}
\end{figure}
On the next texture, it sees that the texture isn't the default one, so it
adds the second texture into the linked list, as shown in Figure \ref{parse3}.
\begin{figure}[htbp]
\begin{centering}
\setlength{\unitlength}{0.0125in}%
\begin{picture}(260,80)(80,700)
\thinlines
\put(240,750){\vector( 1, 0){ 40}}
\put(280,700){\framebox(60,80){}}
\put(140,750){\vector( 1, 0){ 40}}
\put(180,700){\framebox(60,80){}}
\put( 80,720){\framebox(60,60){}}
\put(310,736){\makebox(0,0)[b]{\raisebox{0pt}[0pt][0pt]{\elvrm Texture}}}
\put(310,750){\makebox(0,0)[b]{\raisebox{0pt}[0pt][0pt]{\elvrm WOOD}}}
\put(210,736){\makebox(0,0)[b]{\raisebox{0pt}[0pt][0pt]{\elvrm Texture}}}
\put(210,750){\makebox(0,0)[b]{\raisebox{0pt}[0pt][0pt]{\elvrm MARBLE}}}
\put(110,745){\makebox(0,0)[b]{\raisebox{0pt}[0pt][0pt]{\elvrm Object}}}
\end{picture}

\caption{Object parsing, step 3}
\label{parse3}
\end{centering}
\end{figure}

Now for a problem.  If you want to specify the
{\tt REFRACTION}\index{textures!and {\tt REFRACTION}} of the
texture, the raytracer must first calculate the surface colour.  It
does this by marching through the texture list and mixing all the
colours.  When it's finished, it checks the {\tt ALPHA}\keyindex{ALPHA}
value of the
surface colour and decides whether it should trace a refracting ray.
Where does it get the {\tt REFRACTION} value and the index of
refraction?  It simply takes the one in the topmost (the last one
defined) texture.  I don't see any reason to have refraction values
for any other textures in the layer as it applies to the whole object.

\section{Image Mapping}

\index{mapping, images|see{image mapping}}\index{image mapping}
One major problem people have when designing data files for DKBTrace is how to
position images onto the desired surfaces.  With version 2.10, this problem
becomes slightly easier since the image can be mapped onto the object in the
object's natural coordinate system.  Thereafter, when the object is
translated, rotated, or scaled, the image map will follow it.

The image mapping that DKBTrace currently supports is called a
``parallel projection''\index{projection|see{parallel projection}}%
\index{parallel projection} mapping.  This technique is simple (that's why
I implemented it), but it's not perfect.  It works like a slide
projector casting the desired image onto the scene.  The difference,
however, is that the image never gets larger as you move further away
from the slide projector.  In fact, there is no real slide projector.
Consider the cross section shown in Figure \ref{project}, which
shows an image with colours A, B, C, D, and E being mapped onto three
objects.
\begin{figure}[htbp]
\begin{centering}
\setlength{\unitlength}{0.0125in}%
\begin{picture}(313,120)(87,660)
\thinlines
\put(300,720){\framebox(80,60){}}
\put(385,730){\vector( 1, 0){ 15}}
\put(385,745){\vector( 1, 0){ 15}}
\put(385,760){\vector( 1, 0){ 15}}
\put(285,700){\vector( 1, 0){115}}
\put(275,715){\vector( 1, 0){125}}
\put(100,700){\vector( 1, 0){ 95}}
\put(100,715){\vector( 1, 0){100}}
\put(270,730){\vector( 1, 0){ 25}}
\put(100,730){\vector( 1, 0){110}}
\put(260,745){\vector( 1, 0){ 35}}
\put(255,760){\vector( 1, 0){ 40}}
\put(180,745){\vector( 1, 0){ 35}}
\put(190,760){\vector( 1, 0){ 35}}
\put(100,745){\vector( 1, 0){ 60}}
\put(100,760){\vector( 1, 0){ 50}}
\put(240,780){\line(-1,-2){ 60}}
\put(180,660){\line( 1, 0){120}}
\put(300,660){\line(-1, 2){ 60}}
\put(140,780){\line( 1, 0){ 60}}
\put(200,780){\line(-3,-4){ 30}}
\put(170,740){\line(-3, 4){ 30}}
\put(375,725){\makebox(0,0)[rb]{\raisebox{0pt}[0pt][0pt]{\elvrm C}}}
\put(375,740){\makebox(0,0)[rb]{\raisebox{0pt}[0pt][0pt]{\elvrm B}}}
\put(375,755){\makebox(0,0)[rb]{\raisebox{0pt}[0pt][0pt]{\elvrm A}}}
\put(340,725){\makebox(0,0)[b]{\raisebox{0pt}[0pt][0pt]{\elvrm C}}}
\put(305,725){\makebox(0,0)[lb]{\raisebox{0pt}[0pt][0pt]{\elvrm C}}}
\put(305,740){\makebox(0,0)[lb]{\raisebox{0pt}[0pt][0pt]{\elvrm B}}}
\put(305,755){\makebox(0,0)[lb]{\raisebox{0pt}[0pt][0pt]{\elvrm A}}}
\put(230,755){\makebox(0,0)[lb]{\raisebox{0pt}[0pt][0pt]{\elvrm A}}}
\put(250,755){\makebox(0,0)[rb]{\raisebox{0pt}[0pt][0pt]{\elvrm A}}}
\put(180,755){\makebox(0,0)[rb]{\raisebox{0pt}[0pt][0pt]{\elvrm A}}}
\put(170,745){\makebox(0,0)[b]{\raisebox{0pt}[0pt][0pt]{\elvrm B}}}
\put(255,740){\makebox(0,0)[rb]{\raisebox{0pt}[0pt][0pt]{\elvrm B}}}
\put(260,725){\makebox(0,0)[rb]{\raisebox{0pt}[0pt][0pt]{\elvrm C}}}
\put(270,710){\makebox(0,0)[rb]{\raisebox{0pt}[0pt][0pt]{\elvrm D}}}
\put(275,695){\makebox(0,0)[rb]{\raisebox{0pt}[0pt][0pt]{\elvrm E}}}
\put(205,695){\makebox(0,0)[lb]{\raisebox{0pt}[0pt][0pt]{\elvrm E}}}
\put(210,710){\makebox(0,0)[lb]{\raisebox{0pt}[0pt][0pt]{\elvrm D}}}
\put(220,725){\makebox(0,0)[lb]{\raisebox{0pt}[0pt][0pt]{\elvrm C}}}
\put(225,740){\makebox(0,0)[lb]{\raisebox{0pt}[0pt][0pt]{\elvrm B}}}
\put(160,755){\makebox(0,0)[lb]{\raisebox{0pt}[0pt][0pt]{\elvrm A}}}
\put( 95,695){\makebox(0,0)[rb]{\raisebox{0pt}[0pt][0pt]{\elvrm E}}}
\put( 95,710){\makebox(0,0)[rb]{\raisebox{0pt}[0pt][0pt]{\elvrm D}}}
\put( 95,725){\makebox(0,0)[rb]{\raisebox{0pt}[0pt][0pt]{\elvrm C}}}
\put( 95,740){\makebox(0,0)[rb]{\raisebox{0pt}[0pt][0pt]{\elvrm B}}}
\put( 95,755){\makebox(0,0)[rb]{\raisebox{0pt}[0pt][0pt]{\elvrm A}}}
\end{picture}

\caption{Parallel projection}
\label{project}
\end{centering}
\end{figure}
The raytracer performs a similar operation
to map a 2D picture onto a 3D object.  Note that objects cannot shadow each
other from the image being mapped.  This means that the image will also appear
on the back of the object as a mirror image.

The mapping takes the original image (regardless of the size) and maps
it onto the range 0,0 to 1,1 in two of the 3D coordinates.  Which two
coordinates is specified by the gradient\index{image mapping!gradient}
vector provided after the
image.  This vector must contain one positive number, one negative
number, and one zero.  The positive number identifies the $u$ axis
(the left-right direction in the image) and the negative number
represents the $v$ axis (the picture's up-down direction).  Note that
the magnitude of the number is irrelevant. For example:
\begin{verbatim}
    IMAGEMAP <1 -1 0> GIF "filename"
\end{verbatim}
will map the GIF\index{GIF} picture onto the square from {\tt <0 0 0>} to
{\tt <1 1 0>} as shown in Figure \ref{map1}.
\begin{figure}[htbp]
\begin{centering}
\setlength{\unitlength}{0.0125in}%
\begin{picture}(110,115)(170,685)
\thinlines
\put(180,700){\vector( 0, 1){100}}
\multiput(260,780)(0.00000,-7.61905){11}{\line( 0,-1){  3.810}}
\multiput(180,780)(7.61905,0.00000){11}{\line( 1, 0){  3.810}}
\put(180,700){\vector( 1, 0){100}}
\put(220,705){\makebox(0,0)[b]{\raisebox{0pt}[0pt][0pt]{\elvrm Bottom}}}
\put(220,770){\makebox(0,0)[b]{\raisebox{0pt}[0pt][0pt]{\elvrm Top}}}
\put(260,685){\makebox(0,0)[b]{\raisebox{0pt}[0pt][0pt]{\elvrm 1}}}
\put(175,775){\makebox(0,0)[rb]{\raisebox{0pt}[0pt][0pt]{\elvrm 1}}}
\put(185,750){\makebox(0,0)[lb]{\raisebox{0pt}[0pt][0pt]{\elvrm L}}}
\put(185,741){\makebox(0,0)[lb]{\raisebox{0pt}[0pt][0pt]{\elvrm e}}}
\put(185,732){\makebox(0,0)[lb]{\raisebox{0pt}[0pt][0pt]{\elvrm f}}}
\put(185,723){\makebox(0,0)[lb]{\raisebox{0pt}[0pt][0pt]{\elvrm t}}}
\put(255,760){\makebox(0,0)[rb]{\raisebox{0pt}[0pt][0pt]{\elvrm R}}}
\put(253,749){\makebox(0,0)[rb]{\raisebox{0pt}[0pt][0pt]{\elvrm i}}}
\put(255,742){\makebox(0,0)[rb]{\raisebox{0pt}[0pt][0pt]{\elvrm g}}}
\put(255,730){\makebox(0,0)[rb]{\raisebox{0pt}[0pt][0pt]{\elvrm h}}}
\put(254,720){\makebox(0,0)[rb]{\raisebox{0pt}[0pt][0pt]{\elvrm t}}}
\end{picture}

\caption{Image mapping with vector {\tt <1 -1 0>}}
\label{map1}
\end{centering}
\end{figure}
If we reversed the vector, the picture would be transposed:
\begin{verbatim}
    IMAGEMAP <-1 1 0> GIF "filename"
\end{verbatim}
produces the result shown in Figure \ref{map2}.
\begin{figure}[htbp]
\begin{centering}
\setlength{\unitlength}{0.0125in}%
\begin{picture}(109,115)(170,685)
\thinlines
\put(180,700){\vector( 0, 1){100}}
\multiput(260,780)(0.00000,-7.61905){11}{\line( 0,-1){  3.810}}
\multiput(180,780)(7.61905,0.00000){11}{\line( 1, 0){  3.810}}
\put(179,700){\vector( 1, 0){100}}
\put(184,711){\makebox(0,0)[lb]{\raisebox{0pt}[0pt][0pt]{\elvrm m}}}
\put(186,719){\makebox(0,0)[lb]{\raisebox{0pt}[0pt][0pt]{\elvrm o}}}
\put(186,727){\makebox(0,0)[lb]{\raisebox{0pt}[0pt][0pt]{\elvrm t}}}
\put(186,735){\makebox(0,0)[lb]{\raisebox{0pt}[0pt][0pt]{\elvrm t}}}
\put(186,745){\makebox(0,0)[lb]{\raisebox{0pt}[0pt][0pt]{\elvrm o}}}
\put(185,755){\makebox(0,0)[lb]{\raisebox{0pt}[0pt][0pt]{\elvrm B}}}
\put(220,705){\makebox(0,0)[b]{\raisebox{0pt}[0pt][0pt]{\elvrm Left}}}
\put(220,770){\makebox(0,0)[b]{\raisebox{0pt}[0pt][0pt]{\elvrm Right}}}
\put(260,685){\makebox(0,0)[b]{\raisebox{0pt}[0pt][0pt]{\elvrm 1}}}
\put(175,775){\makebox(0,0)[rb]{\raisebox{0pt}[0pt][0pt]{\elvrm 1}}}
\put(250,745){\makebox(0,0)[lb]{\raisebox{0pt}[0pt][0pt]{\elvrm T}}}
\put(251,736){\makebox(0,0)[lb]{\raisebox{0pt}[0pt][0pt]{\elvrm o}}}
\put(251,728){\makebox(0,0)[lb]{\raisebox{0pt}[0pt][0pt]{\elvrm p}}}
\end{picture}

\caption{Image mapping with vector {\tt <-1 1 0>}}
\label{square-map2}
\end{centering}
\end{figure}
Once the image orientation has been determined, it can be translated, rotated,
and scaled as desired to map properly onto the object.

\section{Output File Formats}

\index{output!formats}
People always ask me to describe the output file formats of DKBTrace.  I
received so many requests for this that I decided to put it into the document.
The normal ``default'' output format is ``Dump''\index{dump format}
format.  This is based on QRT format and goes like this, where each
character is a hex digit:

{\center
\fbox{\begin{tabular}{lll}
Header: \\
& {\tt wwww hhhh} & Width, height (16 bits each, LSB first) \\
\multicolumn{2}{l}{For each data line:} \\
 & {\tt llll}         &
\parbox[t]{2.5in}{line number (16 bits, LSB first, 0 to LINES-1)} \\
 & {\tt rr rr rr rr rr} \ldots & the red components for that line \\
 &                    & (8 bits each - 0=dark, 255=bright) \\
 & {\tt gg gg gg gg gg} \ldots & the green components for that line \\
 &                    & (8 bits each - 0=dark, 255=bright) \\
 & {\tt bb bb bb bb bb} \ldots & the blue components for that line \\
 &                    & (8 bits each - 0=dark, 255=bright) \\
\end{tabular}}}

\vskip10pt
Note that this format is slightly different from QRT's.

The\index{raw format} {\tt +fr}\optindex{f} option of the raytracer
produces ``raw'' files.  These are simply three files with no header
information and no line number information -- just the raw data.

The\index{Targe format} {\tt +ft} option writes out Targa format.
Specifically, the fields are:

{\center
\fbox{\begin{tabular}{lll}
Header: \\
 & {\tt 00 00 02 00 00} & Fixed header information for\ldots \\
 & {\tt 00 00 00} & \hspace{0.2in}\ldots uncompressed type 2 image \\
 & {\tt 0000} & Horizontal offset always is at 0000 \\
 & {\tt llll} &
\parbox[t]{2.5in}{Vertical offset (1st line number, 16 bits, LSB first)} \\
 & {\tt wwww hhhh} &
\parbox[t]{2.5in}{width, height of image (16 bits each, LSB first)} \\
 & {\tt 18 20} & 24 bits per pixel, Top-down raster \\
\multicolumn{2}{l}{For each data line:} \\
 & {\tt bb gg rr bb gg rr} \ldots  &
\parbox[t]{2.5in}{blue, green, and red data, 8 bits for each pixel in
that line.} \\
\end{tabular}}}
