%-*-Mode: LaTeX-*-

% tell LaTeX about a few words to get better breaks
\hyphenation{para-bo-loid para-bo-loids bit-map bit-mapped}

% Extra commands for adding index entries:
% add an entry for a definition of a command line option
\newcommand{\optdefindex}[2]{\optindex{#1}\index{#2|see{{\tt #1} option}}}
% add an entry for a reference to a command line option
\newcommand{\optindex}[1]{\index{#1@{\tt #1} option}\index{command-line options!#1@{\tt #1} option}}
% add an entry for a keyword
\newcommand{\keyindex}[1]{\index{#1@{\tt #1} keyword}}
% add an entry in teletype font
\newcommand{\ttindex}[1]{\index{#1@{\tt #1}}}
% add an entry in emphasized font
\newcommand{\emindex}[1]{\index{#1@{\em #1}}}
% use the \see command for index cross references
\newcommand{\see}[2]{{\em see\/} #1}
\makeindex

% add the twoside option in the square brackets if desired
\documentstyle[11pt]{report}
\setcounter{secnumdepth}{3}
\setcounter{tocdepth}{3}
\begin{document}

% allow underscore
\catcode`\_=12

\title{DKBTrace Ray-Tracer, Version 2.12}
\author{David K. Buck \\
22C Sonnet Cres. \\
Nepean, Ontario \\
Canada, K2H 8W7 \\
\\
``It's free, and it's well worth the price''}
\maketitle

\pagenumbering{roman}
\tableofcontents
\listoffigures
\clearpage
\pagenumbering{arabic}

\chapter{Copyrights and Disclaimers}

DKBTrace has been made freely distributable.  The author retains the
copyright to the program but authorizes free distribution by BBS'es,
networks or by magnetic media.  The distributor may choose to charge
for the cost of the disk but must not sell the software for profit.
Non-profit organizations such as clubs may charge for the software so
long as the price is reasonable (less than \$5.00 more than the cost of
the disk) and so long as the buyers are informed that the program is
freely distributable. 

The images and data files generated by the raytracer are the property
of the user of the software and may be used for any purpose without
restriction. 

The author makes no guarantees or warranties with this program and
claims no responsibility for any damage or loss of time caused by this
program. Bug reports may be sent to the author but the author is under
no obligation to provide bug fixes, features, or any support for this
software. 

I would also like to place the following conditions on the use of this program:
\begin{enumerate}
\item that it should not be used as part of any commercial package without my
explicit written consent.
\item if you make any neat and interesting pictures, please send them to me.
\item If you make any changes to the source code, please let me know. I'd like
to see what you've done.
\item This text file should accompany the program.
\end{enumerate}

The GIF\index{GIF} file reader was written by
Steven A.\ Bennett\index{Bennett, Steve}.  Here's his copyright notice:
{\footnotesize
\begin{verbatim}
 * DECODER.C - An LZW decoder for GIF
 * Copyright (C) 1987, by Steven A. Bennett
 *
 * Permission is given by the author to freely redistribute and include
 * this code in any program as long as this credit is given where due.
 *
 * In accordance with the above, I want to credit Steve Wilhite, who
 * wrote the code which this is heavily inspired by...
 *
 * GIF and 'Graphics Interchange Format' are trademarks (tm) of
 * Compuserve, Incorporated, an H&R Block Company.
\end{verbatim}
}

The {\tt Noise} and {\tt DNoise} functions (used for texturing) were
written by Robert Skinner\index{Skinner, Robert} and used here with
his permission.

\chapter{System Overview}

This program is a ray tracer written completely in C.  It supports
arbitrary quadric surfaces (spheres, ellipsoids, cones, cylinders,
planes, etc.), constructive solid geometry, and various shading models
(reflection, refraction, marble, wood, and many others).  It also has
special-case code to handle spheres, planes, triangles, and smooth
triangles.  By using these special primitives, the rendering can be
done much more quickly than by using the more general quadrics.  In
order to create pictures with this program, you must describe the
objects in the world.  This description is a text file called
{\em filename}.{\tt data}, and {\em filename\/} defaults to {\tt object}
if not specified.  Normally, such files are difficult to write and to
read.  In order to make this task easier, the program contains a
parser to read the data file.  It allows the user to easily create
complex worlds from simple components.  Since the parser allows
include files, the user may put the object descriptions into different
files and combine them all into one final image.

This document is organized as follows. The first chapters describe how
to run DKBTrace, the various options you can give to alter its
behaviour, and provide a ``walk-through'' tutorial to get you familiar
with its concepts and features. The largest chapter then specifies the
scene description language in detail, and serves as a reference manual
for the system. Subsequent chapters discuss issues related to
DKBTrace, such as common questions (with answers), compiling and
porting the system, utilities for use with DKBTrace, and system
history and compatibility notes. Finally, a short bibliography of
ray-tracing references and an index are provided.

\chapter{Getting Started}

This chapter describes what you need to do to jump right in and start
generating images. The next chapter describes the command-line options
in more detail. Here's what you need to do:

\begin{enumerate}
\item Put the file {\tt sunset.dat} into your current directory.
\item Make sure the files {\tt colors.dat}, {\tt shapes.dat}, and
{\tt textures.dat} are present.
\item Set up the default parameters.  This can be done by creating a file
called {\tt trace.def}\index{trace.def@{\tt trace.def} startup file}
or by setting the
{\tt DKBOPT}\index{DKBOPT@{\tt DKBOPT} environment variable}
environment variable.
In either case, use the following line (you may change these defaults
to suit your system):
\begin{quote}
{\tt -w320 -h400 -v +f +d +p +x -a}
\end{quote}
Meaning:
\begin{description}
\item[{\tt -v}] \optindex{v}Don't be verbose, i.e. don't show
line numbers during trace.
\item[{\tt +f}] \optindex{f}Write an output file in ``dump'' or ``Targa'' format (machine
specific; IBM default is ``Targa''\index{Targa}, Amiga and Unix
default is ``dump'').
\item[{\tt +d}] \optindex{d}Display the image while rendering
(on some systems, an
additional character may follow this option to specify the
graphics mode to use for the display).
\item[{\tt -w320}] \optindex{w}Make the image 320 pixels wide.
\item[{\tt -h400}]  \optindex{h}Make the image 400 pixels high.
\item[{\tt +p}]  \optindex{p}Prompt before exitting to let you look at the picture.
\item[{\tt +x}]  \optindex{x}Allow exitting with a key hit before the trace is finished.
\item[{\tt -a}]  \optindex{a}Don't Antialias (Antialiasing smooths out jagged edges).
\end{description}
\item To render a scene, type:
\begin{quote}
{\tt dkb{\em xxx} -isunset.dat -osunset.dis}
\end{quote}
Meaning:
\begin{description}
\item[{\tt dkb{\em xxx}}] On different systems, the name of the
executable may vary.  Check to see what it is on your system.  For the
Amiga, for example, two versions are supplied: {\tt dkb881} and
{\tt dkbieee} for systems with and without a floating-point accelerator.
For the IBM, {\tt DKB.EXE} is the 20286 optimized/uses 80287 math
coprocessor version, and {\tt DKBNO87.EXE} is the plain 8086/no 8087
math coprocessor version.
\item[{\tt -isunset.dat}] \optindex{i}Read the input file {\tt sunset.dat}.
\item[{\tt -osunset.dis}] \optindex{o}Call the output file {\tt sunset.dis}.
This is the usual file name extension for ``dump'' format.
``Targa'' format files (default for IBM's) generally use the extension
{\tt .tga}.
\end{description}
\item Once the image has been rendered, you must use a post-processor to
create the final viewable image file (i.e. an IFF\index{IFF} or
GIF\index{GIF} file), unless
you possess 24-bit display hardware and can view the generated output
files directly.  The post-processor used depends on your system.  See
the section ``Displaying the Images'' for more details on post-processing
the image.
\end{enumerate}

\chapter{Command Line Options}

\newlength{\origitemsep}
\setlength{\origitemsep}{\itemsep}
\newcommand{\noitemsep}{\setlength{\itemsep}{-\parskip}}
\newcommand{\doitemsep}{\setlength{\itemsep}{\origitemsep}}

\index{command-line options}
\index{options,command-line|see{command-line options}}
This program is designed to be run from a command line.  The
command-line options
may be specified in any order.  Repeated options overwrite
the previous values.  Options may also be specified in a file called
{\tt trace.def}\index{trace.def@{\tt trace.def} startup file} or by
the environment variable
{\tt DKBOPT}.\index{DKBOPT@{\tt DKBOPT} environment variable}
\begin{description}
\item[{\tt -w{\em width}}] \optdefindex{w}{width of picture}Width
of the picture in pixels. (On the Amiga, use 320 for full-sized pictures.)

\item[{\tt -h{\em height}}] \optdefindex{h}{height of picture}Height
of the picture in pixels. (On the Amiga, use 400 for full-sized pictures.)

\item[{\tt +v}] \optdefindex{v}{verbosity}Verbose option.
\noitemsep
\item[{\tt -v}] Disable verbose option.
\doitemsep

In verbose mode, the scan line number is printed as each line is traced.

\item[{\tt +f{\em x}}] \optdefindex{f}{output!file generation}Produce
an output file.
\noitemsep
\item[{\tt -f}] Don't produce an output file.
\doitemsep

If the {\tt +f} option is used, the ray tracer will produce an output file
of the picture.  This output file describes each pixel with 24 bits.
Currently, three formats of output files are supported:
\begin{itemize}
\item[{\tt +fd}] Default -- Dump format (QRT-style).
\noitemsep
\item[{\tt +fr}] Raw format -- three files for R, G and B.
\item[{\tt +ft}] Uncompressed Targa-24 format.
\doitemsep
\end{itemize}
Normally, a post-processor is required to create the final finished
image from the data file.  See the section on ``Displaying the Images''
for details.

\item[{\tt +d{\em x}}] \optdefindex{d}{displaying!while tracing}Display
the picture while tracing.
\noitemsep
\item[{\tt -d}] Don't display the picture while tracing.
\doitemsep

If the {\tt +d} option is used, then the picture will be displayed while the
program performs the ray tracing.  On most systems, this picture is
not as good as the one created by the post-processor because it does
not try to make optimum choices for the colour registers.

Depending on the system, a letter may follow the {\tt +d} option to specify
the graphics mode to use.
\begin{description}
\item[All systems:] \mbox{}
\begin{itemize}
\item[{\tt +d}] Default Format (same as {\tt +d0})
\end{itemize}
\item[Amiga:] \mbox{}
\begin{itemize}
\item[{\tt +d0}] Ham format
\item[{\tt +dE}] Ham-E format
\end{itemize}
\item[IBM:] \mbox{}
\begin{itemize}
\item[{\tt +d0}] Autodetect (S)VGA type
\item[{\tt +d1}] Standard VGA 320x200
\item[{\tt +d2}] Simulated SVGA 360x480
\item[{\tt +d3}] Tseng Labs 3000 SVGA 640x480
\item[{\tt +d4}] Tseng Labs 4000 SVGA 640x480
\item[{\tt +d5}] AT\&T VDC600 SVGA 640x400
\item[{\tt +d6}] Oak Technologies SVGA 640x480
\item[{\tt +d7}] Video 7 SVGA 640x480
\item[{\tt +d8}] Video 7 Vega (Cirrus) VGA 360x480
\item[{\tt +d9}] Paradise SVGA 640x480
\item[{\tt +dA}] Ahead Systems Ver. A SVGA 640x480
\item[{\tt +dB}] Ahead Systems Ver. B SVGA 640x480
\item[{\tt +dC}] Chips \& Technologies SVGA 640x480
\item[{\tt +dD}] ATI SGVA 640x480
\item[{\tt +dE}] Everex SVGA 640x480
\item[{\tt +dF}] Trident SVGA 640x480
\item[{\tt +dG}] VESA Standard SVGA Adapter 640x480
\end{itemize}
\end{description}

\item[{\tt +p}] \optdefindex{p}{wait for prompt}Wait for prompt
(IBM: beep and pause) before quitting.
\noitemsep
\item[{\tt -p}] Finish without waiting.
\doitemsep

The {\tt +p} option makes the program wait for a carriage return before
exitting (and closing the graphics screen).  This gives you time to
admire the final picture before destroying it.

\item[{\tt -i{\em filename}}] \optdefindex{i}{input filename}%
\optdefindex{i}{filename!input}Set the input filename.
\noitemsep
\item[{\tt -o{\em filename}}] \optdefindex{o}{output!filename}%
\optdefindex{o}{filename!output}Set the output filename.
\doitemsep

If your input file is not {\tt Object.dat}, then you can use {\tt -i}
to set the filename.  The default output filename will be
{\tt data.dis} for dump mode, {\tt data.red}, {\tt data.grn} or
{\tt data.blu} for raw mode, and {\tt data.tga} for Targa mode.
If you want a different output file name, use the {\tt -o} option.
(On IBM's, the default extensions for raw mode are {\tt .r8},
{\tt .g8}, and {\tt .b8} to conform to PICLAB's ``raw'' format.)

\item[{\tt +a{\em level}}] \optdefindex{a}{anti-aliasing}Anti-alias --
{\em level\/} is an optional tolerance level (default 0.3).
\noitemsep
\item[{\tt -a}] Don't anti-alias.
\doitemsep

The {\tt +a} option enables adaptive anti-aliasing.  The number
after the {\tt +a} option determines the threshold for the
anti-aliasing.  If the colour of a pixel differs from its neighbor
(to the left or above) by more than the threshold, then the
pixel is subdivided and super-sampled.

If the anti-aliasing threshold is 0.0, then every pixel is
supersampled.  If the threshold is 1.0, then no anti-aliasing
is done.  Good values seem to be around 0.2 to 0.4.

The super-samples are jittered to introduce noise and make the
pictures look better.  Note that the jittering "noise" is non-
random and repeatable in nature, based on an object's 3-D
orientation in space.  Thus, it's okay to use anti-aliasing for
animation sequences, as the anti-aliased pixels won't vary and
flicker annoyingly from frame to frame.

\item[{\tt +x}] \optdefindex{x}{exiting early}Allow early exit by
hitting any key.\hfill[IBM only]
\noitemsep
\item[{\tt -x}] Lock in trace until finished.\hfill[IBM only]
\doitemsep

On the IBM, the {\tt -x} option disables the ability to abort the
trace by hitting a key.  If you are unusually clumsy or have
cats that stomp on your keyboard, you
may want to use it.  If you are writing a file, the system
will recognize {\tt Ctrl-C} at the end of line if the system {\tt BREAK}
is {\tt ON}.  If you aren't writing a file, you won't be able to
abort the trace until it's done.

This option was meant for big, long late-nite traces that take
all night (or longer!), and you don't want them interrupted by
anything less important than a natural disaster such as hurricane,
fire, flood, famine, etc.

\item[{\tt -b{\em size}}] \optdefindex{b}{output!buffer size}%
\optdefindex{b}{buffer size}Use an output file buffer of
{\em size} kilobytes.

The {\tt -b} option allows you to assign large buffers to the output
file.  This reduces the amount of time spent writing to the
disk and prevents unnecessary wear (especially for floppies). 
If this parameter is zero (the default), then as each scanline is finished,
the line is written to the file and the file is flushed.  On
most systems, this operation insures that the file is written
to the disk so that in the event of a system crash or other
catastrophic event, at least part of the picture has been
stored properly on disk.

\item[{\tt +c}] \optdefindex{c}{continue rendering}Continue Rendering.

If, for some reason, you abort a raytrace while it's in progress
or if you used the {\tt -e} option (below) to end the raytrace
prematurely, you can use the {\tt +c} option to continue the raytrace
when you get back to it.  This option reads in the previously
generated output file, displays the image to date on the screen,
then proceeds with the raytracing.  In many cases, this feature
can save you a lot of rendering time when things go wrong.\footnote{If
you want to impress your friends with the speed of your
computer, take an image you've already rendered and use {\tt +c}
in the command-line.  It renders {\em real} fast that way!}

\item[{\tt -s{\em line}}] \optdefindex{s}{starting scan line}%
\optdefindex{s}{scan line!starting}Start tracing at line number {\em line}.
\noitemsep
\item[{\tt -e{\em line}}] \optdefindex{e}{ending scan line}%
\optdefindex{e}{scan line!ending}End tracing at line number {\em line}.
\doitemsep

The {\tt -s} option allows you to start rendering an image starting
from a specific scan line.  This is useful for rendering part
of a scene to see what it looks like without having to render
the entire scene from the top.  Alternatively, you can render
groups of scanlines on different systems and concatenate them
later.  WARNING: If you are merging output files from different
systems, make sure that the random number generators are the
same.  If not, the textures from one will not blend in with the
textures from the other.  There is an example of a standard
ANSI-C random number generator in the file {\tt IBM.C}.  Cut it
out and paste it into your machine-specific {\tt .c} file if you
plan to try ``distributed processing'' and are not sure if you
need this standardization.

The {\tt -s} option has no effect when continuing a raytrace using
the {\tt +c} option.  The renderer will figure out where to restart.

\item[{\tt -q{\em level}}] \optdefindex{q}{quality}%
\optdefindex{q}{rendering quality}Rendering quality.

The {\tt -q} option allows you to specify the image rendering quality,
for quickly rendering images for testing.  The {\em level} parameter can
range from 0 to 9.  The values correspond to the following
quality levels:

\begin{itemize}
\item[0,1:] Just show colours.  Ambient lighting only.
\noitemsep
\item[2,3:] Show Diffuse and Ambient light.
\item[4,5:] Render shadows.
\item[6,7:] Create surface textures.
\item[8,9:] Compute reflected, refracted, and transmitted rays.
\end{itemize}
\doitemsep

The default is {\tt -q9} (maximum quality) if not specified.

\item[{\tt -l{\em path}}] \optdefindex{l}{library path}The {\tt -l}
option may be used to specify a ``library'' pathname to
look into for data files to include or for images.  Up to 10
{\tt -l} options may be used to specify a search path.  The home
(current) directory will be searched first followed by the
indicated library directories in order.

\item[{\tt +z}] \optdefindex{z}{debugging}The {\tt +z} option is
an undocumented feature.
You will not see any references to it in this or any other
documentation file for DKBTrace.  In fact, no other section of the
document will even admit that it was mentioned here. If you really
want to know what it does, then you will have to look into the source
code ({\tt trace.c}) and read the comment just above the {\tt +z}
option that says
``Turn on debugging print statements.''  The full purpose of this option
will, therefore, be left as an exercise for the reader, but believe me
-- it's nothing terribly exciting.\footnote{For those people who run
the raytracer on super-fast systems and want to slow it down, you may
try this option.}
\end{description}

You may specify the default parameters by modifying the file
{\tt trace.def}\index{trace.def@{\tt trace.def} startup file}
which contains the parameters in the above format. 
This filename contains a complete command line as though you 
had typed it in, and is processed before any options supplied
on the command line are recognized. You may also set the environment
variable
{\tt DKBOPT}\index{DKBOPT@{\tt DKBOPT} environment variable}
to the desired set of command-line parameters.

\chapter{A Tutorial Walkthrough}

This section, is designed to get you up and running designing your own
pictures without all the nit-picky details.  Once you've made a few
of your own data files, you'll probably want to advance to the next
section to fill in the gaps.

\section{The First Image}

Let's get right to the meat of the matter and create the data file for a
simple picture.  Since raytracers thrive on spheres, that's what we'll render
first.

First we have to create a viewpoint to tell the computer where our camera is
and where it's looking.  To do this, we use 3D coordinates.  The usual
coordinate system\index{coordinate system}
for DKBTrace has Y pointing up, X pointing to the right, and
Z pointing into the screen as shown in Figure \ref{axes}.

\begin{figure}[htbp]
\begin{centering}
\setlength{\unitlength}{0.0125in}%
\begin{picture}(95,112)(250,660)
\thinlines
\put(250,660){\vector( 1, 2){ 50}}
\put(260,680){\vector( 1, 0){ 80}}
\put(260,680){\vector( 0, 1){ 80}}
\put(300,740){\makebox(0,0)[lb]{\raisebox{0pt}[0pt][0pt]{\elvrm Z}}}
\put(260,765){\makebox(0,0)[lb]{\raisebox{0pt}[0pt][0pt]{\elvrm Y}}}
\put(345,680){\makebox(0,0)[lb]{\raisebox{0pt}[0pt][0pt]{\elvrm X}}}
\end{picture}

\caption{DKBTrace Coordinate Axes}
\label{axes}
\end{centering}
\end{figure}

Using your personal favorite text editor (i.e., user interface),
create a file called {\tt picture1.dat}.  Now, type in the
following:\footnote{The input is case sensitive, so be sure to
get capital and lowercase letters correct.}
\begin{verbatim}
      INCLUDE "colors.dat"
      INCLUDE "shapes.dat"
      INCLUDE "textures.dat"

      VIEW_POINT
         LOCATION  <0 0 0>
         DIRECTION <0 0 1>
         UP        <0 1 0>
         RIGHT     <1.33333 0 0>
      END_VIEW_POINT
\end{verbatim}
The first {\tt INCLUDE}\keyindex{INCLUDE}
statement reads in definitions for various
useful colours.  (Being a proud Canadian, I spell colour the proper
way with a ``u''.  To avoid confusing the rest of the world, however,
I've set up the raytracer to allow either spelling of the word.)

The second and third include statements read in some useful shapes and textures
respectively.  When you get a chance, have a look through them to see
but a few of the many possible shapes and textures available.

Include files may be nested, if you like.  The total pre-defined number of
{\tt INCLUDE}'d files (nested or not) per scene is 10.

Filenames specified in the {\tt INCLUDE} statements will be searched
for in the home (current) directory first, and if not found, will then
be searched for in directories specified by any {\tt -l}\optindex{l}
(library path) options active.  This would facilitate keeping all your
"include" ({\tt .inc}) files, {\tt shapes.dat}, {\tt colors.dat}, and
{\tt textures.dat} in an ``include'' subdirectory, and giving an
{\tt -l} option on the command line to where your library of include files
are.

This viewpoint declaration puts our camera at the center of the
universe ({\tt LOCATION <0 0 0>}) pointing into the Z direction
({\tt DIRECTION <0 0 1>}) and with the camera being held upright
({\tt UP <0 1 0>}).  The final term compensates for the aspect ratio
of the screen
({\tt RIGHT <1.33333 0 0>}).  If your computer has square pixels, you
may want to change this to ``{\tt RIGHT <1.0 0 0>}''.  For details on
exactly how the camera works, see the section ``How it All Works''.

Now, let's place a red sphere into the world:
\begin{verbatim}
      OBJECT
         SPHERE <0 0 3> 1 END_SPHERE
         TEXTURE
            COLOUR Red
         END_TEXTURE
      END_OBJECT
\end{verbatim}
This sphere is 3 units away from the camera and has a radius of 1.  Note that
any parameter that changes the appearance of the surface (as opposed to the
shape of the surface) is called a texture parameter and {\em must} be
placed into a
\verb#TEXTURE#-\verb#END_TEXTURE# block.  In this case, we are just
setting the colour.

One more detail -- we need a light source:
\begin{verbatim}
      OBJECT
         SPHERE <0 0 0> 1 END_SPHERE
         TEXTURE
            COLOUR White
         END_TEXTURE
                             { This is 2 units to our right, }
         TRANSLATE <2 4 -3>  { 4 units above, and 3 units    }
                             { behind our camera.            }
         LIGHT_SOURCE
         COLOUR White
      END_OBJECT
\end{verbatim}
Note: For light sources, {\em always} declare them to be centered at
the origin {\tt <0 0 0>}, then use {\tt TRANSLATE} to put them where
you want.\index{light sources!and {\tt TRANSLATE}}
If you don't do this, the light source won't work right. We
must also specify the colour of the light source {\em outside} the
{\tt TEXTURE} block because the renderer doesn't want to work out the
whole surface colour just to get the colour of the light it
emits.\index{light sources!and {\tt TEXTURE}}

That's it!  Close the file and render a small picture of
it:\footnote{The program name may vary on different systems.}
\begin{verbatim}
      trace -w80 -h100 -f -ipicture1.dat
\end{verbatim}
On the IBM, the command line would be:\footnote{The program name would
be {\tt dkbno87} if you have an 8086/8088 system or no math co-processor.}
\begin{verbatim}
      dkb -w80 -h50 -f -ipicture1.dat
\end{verbatim}

\section{Phong Highlights}

You've now rendered your first picture.  I know you want to run out and show
all your friends how amazing your computer is to be able to generate such an
incredible picture, but just wait a few minutes -- you ain't seen nothin' yet.
(For those people who complained that the picture took too long to draw, just
wait -- you ain't seen nothin' yet, either\ldots)

Let's add a nice little specular highlight\index{specular highlight}
(shiny spot) to the sphere.  It
gives it that neat ``computer graphics'' look.  Change the definition of the
sphere to this:
\begin{verbatim}
      OBJECT
         SPHERE <0 0 3> 1 END_SPHERE
         TEXTURE
            COLOUR Red
            PHONG 1.0
         END_TEXTURE
      END_OBJECT
\end{verbatim}
Now render this.  In all seriousness, the {\tt PHONG} highlight does
add a lot of credibility to the picture.  You'll probably want to use
it in many of your pictures.

\section{Textures}

One of the really nice features of this raytracer is its sophisticated
textures.  Change the definition of our sphere to the following and then
re-render it:
\begin{verbatim}
      OBJECT
         SPHERE <0 0 3> 1 END_SPHERE
         TEXTURE
            Dark_Wood
            SCALE <0.2 0.2 0.2>
            PHONG 1.0
         END_TEXTURE
      END_OBJECT
\end{verbatim}
The textures are set up by default to give you one ``feature'' across
a sphere of radius 1.0.  A ``feature'' is roughly defined as a colour
transition.  For example, a wood texture would have one band on a
sphere of radius 1.0.  By scaling the wood by {\tt <0.2 0.2 0.2>}, we
shrink the texture to give us about five bands. Please note that this
is not a hard and fast rule.  It's only meant to give you a rough idea
for the scale to use for a texture.  Don't start reporting problems if
you get three bands instead of five.  This rule of thumb just puts you
in the ballpark.

One note about the {\tt SCALE}\keyindex{SCALE} operation.  You can
magnify or shrink
along each direction separately.  The first term tells how much to
magnify or shrink in the left-right direction.  The second term
controls the up-down direction and the third term controls the
front-back direction.

I encourage you to look through the {\tt textures.dat} file to see
what textures are defined there and try them out.  Some of them are
quite spectacular.

\section{Other Shapes}

So far, we've just defined spheres.  There are several other kinds of
shapes that can be rendered by DKBTrace.  Let's try one out with a
computer graphics standard -- a checkered floor.  Add the following
object to your {\tt .dat} file:
\begin{verbatim}
      OBJECT
         PLANE <0.0 1.0 0.0> -1.0 END_PLANE
         TEXTURE
            CHECKER
               COLOUR RED 1.0
               COLOUR BLUE 1.0
         END_TEXTURE
      END_OBJECT
\end{verbatim}
The object defined here is an infinite plane.  The vector
{\tt <0.0 1.0 0.0>} is the surface normal of the plane (i.e., if you where
standing on the surface, the normal points straight up.)  The number
afterward is the distance that the plane is displaced along the normal
-- in this case, we move the floor down one unit so that the sphere
(radius 1) is resting on it.  The checker texture specifies the two
colours to use in the checker pattern.

Looking at the floor, you'll notice that the wooden ball casts a shadow on the
floor.  Shadows are calculated accurately (well, almost -- more later) by the
raytracer.

Another kind of shape you can use is called a quadric surface.  To be totally
honest, the shapes you've been using so far have been quadrics.  Spheres and
planes are types of quadric surfaces.  There are many other quadric surfaces
however.  These are all described by a certain kind of mathematical formula
(see the section on Quadrics in the next chapter).  They include cylinders,
cones, paraboloids (like a satellite dish), hyperboloids (saddle-shaped) and
ellipsoids as well as the spheres and planes we've used so far.

All quadrics except for ellipsoids and spheres are infinite in at least one
direction.  For example, a cylinder has no top or bottom - it goes to infinity
at each end.  Quadrics all have one common feature - if you draw any straight
line through a quadric, it will hit the surface at most twice.  A torus
(donut), for example, is not a quadric since a line can hit the surface up to
four times going through.

Enough talk -- let's render one of these ``quadrics''\ldots While
we're at it, we'll add a few features to the surface.  Add the
following definition to your {\tt .dat} file:
\begin{verbatim}
      OBJECT
         QUADRIC Cylinder_Y END_QUADRIC
         TEXTURE
            COLOUR GREEN 0.5
            REFLECTION 0.5
         END_TEXTURE
         SCALE <0.4  0.4  0.4>
         TRANSLATE <2 0 5>
      END_OBJECT
\end{verbatim}
This object is a cylinder along the Y (up-down) axis.  It's green in colour
and has a mirrored surface (hence the reflection of 0.5) this means that half
the light coming from the sphere is reflected from other objects in the room.
A reflection of 1.0 is a perfect mirror.

The object has been shrunk by scaling it by {\tt <0.4 0.4 0.4>}.  Note
that since the cylinder is infinite along the Y axis, the middle term
is kind of pointless.  One four tenths of infinity is still infinity.
(Don't use 0, though.  That will definitely cause a fatal crash!)
Finally, the cylinder has been moved back and to the right so you can
see it more clearly.

\section{Constructive Solid Geometry}

The shapes we've talked about so far are nice, but not terribly useful on
their own for making realistic scenes.  It's hard to make interesting objects
when you're limited to spheres, infinite cylinders, infinite planes, and so
forth.  

Constructive Solid Geometry (CSG) is a technique for taking these simple
building blocks and combining them together.  You can use a cylinder to bore a
hole through a sphere.  You can use planes to cap cylinders and turn them into
flat circular disks (that are no longer infinite).

Before getting into CSG, however, let me talk about inside and
outside.\index{inside, of objects}\index{outside, of objects}%
\index{objects!inside and outside}
Every
primitive (except triangles -- I'll talk about this later) divides the world
into two regions.  One region is inside the surface and one is outside.  So,
given any point in space, you can say it's either inside or outside any
particular primitive object (well, it could be exactly on the surface, but
usually numerical inaccuracies will put it to one side or the other).  Even
planes have an inside and an outside.  By definition, the surface normal of
the plane points towards the outside of the plane.  (For a simple floor, for
example, the space above the floor is ``outside'' and the space below the floor
is ``inside''.  For simple floors this in unimportant, but for planes as parts
of CSG's it becomes much more important).

CSG uses the concepts of inside and outside to combine shapes
together. Consider the situation in Figure \ref{csg0}.\footnote{The
diagrams shown here demonstrate the concepts in 2D and are indended
only as an analogy to the 3D case.  Note that the triangles supported
by DKBTrace cannot be used in CSG (except for unions) since they have
no inside and outside.}\index{triangle!inside and outside}

\begin{figure}[htbp]
\begin{centering}
\setlength{\unitlength}{0.0125in}%
\begin{picture}(200,80)(80,680)
\thicklines
\put( 80,725){\line( 1, 0){ 10}}
\thinlines
\put( 80,745){\line( 1, 0){ 10}}
\thicklines
\put(240,740){\line(-2,-3){ 40}}
\put(200,680){\line( 1, 0){ 80}}
\put(280,680){\line(-2, 3){ 40}}
\thinlines
\put(200,760){\line(-3,-4){ 45}}
\put(155,700){\line( 1, 0){ 85}}
\put(240,700){\line(-2, 3){ 40}}
\put(100,720){\makebox(0,0)[lb]{\raisebox{0pt}[0pt][0pt]{\elvrm Figure B}}}
\put(100,740){\makebox(0,0)[lb]{\raisebox{0pt}[0pt][0pt]{\elvrm Figure A}}}
\end{picture}

\caption{Sample 2D figures}
\label{csg0}
\end{centering}
\end{figure}

There are three CSG operations you can use:

\begin{description}
\item[{\tt UNION A B END_UNION}] \keyindex{UNION}A point is inside
the union if it is
inside A {\em or} it's inside B (or both).  This gives an ``additive''
effect to the component objects, as shown in Figure \ref{csg1}.

\begin{figure}[htbp]
\begin{centering}
\setlength{\unitlength}{0.0125in}%
\begin{picture}(120,80)(160,680)
\thicklines
\put(200,680){\line( 3, 4){ 15}}
\put(240,740){\line(-1,-2){ 10}}
\put(280,680){\line(-2, 3){ 40}}
\put(200,680){\line( 1, 0){ 80}}
\thinlines
\put(200,760){\line( 3,-4){ 30}}
\put(160,700){\line( 1, 0){ 55}}
\put(200,760){\line(-2,-3){ 40}}
\end{picture}

\caption{{\tt UNION} of the Sample Figures}
\label{csg1}
\end{centering}
\end{figure}

\item[{\tt INTERSECTION A B END_INTERSECTION}] \keyindex{INTERSECTION}A
point is inside the
intersection if it's inside both A {\em and} B.  This ``logical
AND's'' the shapes and gets the common part, most useful for
``clipping'' infinite shapes off, etc., as shown in Figure \ref{csg2}.

\begin{figure}[htbp]
\begin{centering}
\setlength{\unitlength}{0.0125in}%
\begin{picture}(25,20)(215,700)
\thinlines
\put(240,700){\line(-1, 0){ 25}}
\put(230,720){\line( 1,-2){ 10}}
\thicklines
\put(230,720){\line(-3,-4){ 15}}
\end{picture}

\caption{{\tt INTERSECTION} of the Sample Figures}
\label{csg2}
\end{centering}
\end{figure}

\item[{\tt DIFFERENCE A B END_DIFFERENCE}] \keyindex{DIFFERENCE}A point
is inside the
difference if it's inside A but not inside B.  The result is a
``subtraction'' of the 2nd shape from the first shape, shown in Figure
\ref{csg3}.

\begin{figure}[htbp]
\begin{centering}
\setlength{\unitlength}{0.0125in}%
\begin{picture}(75,60)(155,700)
\thicklines
\put(230,720){\line(-1,-1){ 20}}
\thinlines
\put(200,760){\line(-3,-4){ 45}}
\put(200,760){\line( 3,-4){ 30}}
\put(155,700){\line( 1, 0){ 55}}
\end{picture}

\caption{{\tt DIFFERENCE} of the Sample Figures}
\label{csg3}
\end{centering}
\end{figure}
\end{description}

Let's give a concrete example by drilling a yellow hole through our sphere.
Go to the definition of the sphere and change it to read the following:
\begin{verbatim}
      OBJECT
         DIFFERENCE
            SPHERE <0 0 3> 1 END_SPHERE
            QUADRIC
               Cylinder_Z
               SCALE <0.2 0.2 0.2>
               COLOUR Yellow
            END_QUADRIC
         END_DIFFERENCE
         TEXTURE
            Dark_Wood
            SCALE <0.2 0.2 0.2>
            PHONG 1.0
         END_TEXTURE
      END_OBJECT
\end{verbatim}

One more point about CSG operations.  You can flip a shape inside-out
by putting the keyword {\tt INVERSE}\keyindex{INVERSE} into the
shape's definition.
This keyword will not change the appearance of the shape unless you're
using CSG.  In the case of CSG, it gives you more flexibility.  For
example, the result of
\begin{verbatim}
   INTERSECTION B A-INVERSE END_INTERSECTION
\end{verbatim}
yields the shape shown in Figure \ref{csg4}.
\begin{figure}[htbp]
\begin{centering}
\setlength{\unitlength}{0.0125in}%
\begin{picture}(80,60)(200,680)
\thinlines
\put(240,700){\line(-1, 2){ 10}}
\put(215,700){\line( 1, 0){ 25}}
\thicklines
\put(200,680){\line( 3, 4){ 15}}
\put(240,740){\line(-1,-2){ 10}}
\put(280,680){\line(-2, 3){ 40}}
\put(200,680){\line( 1, 0){ 80}}
\end{picture}

\caption{{\tt INVERSE-DIFFERENCE} of the Sample Figures}
\label{csg4}
\end{centering}
\end{figure}
Note that a {\tt DIFFERENCE} is really just an {\tt INTERSECTION} of
one shape with the {\tt INVERSE} of another.  This happens to be how
{\tt DIFFERENCE} is actually implemented in the code.

\section{Advanced Textures}

The textures available in DKBTrace are 3D solid textures.  This means that the
texture defines a colour for any 3D point in space.  Just like a real block of
marble or wood, there is colour all through the block -- you just can't see it
until you carve away the wood or marble that's in the way.  Similarly, with a
3D solid texture, you don't see all the colours in the texture -- you only see
the colours that happen to be visible at the surface of the object.

As you've already seen, you can scale, translate, and rotate
textures\index{textures!animation of}%
\index{animation|see{textures, animation of}}.
In fact, you could make an animation in which the objects stay still and the
textures translate and rotate through the object.  The effect would be like
watching a time-lapse film of a cloudy sky -- the clouds would not only move,
but they would also change shape smoothly.

Often, textures are perturbed by noise.  This ``turbulence''
\index{textures!and {\tt TURBULENCE}}\index{perturbation}%
\index{noise}\index{turbulence}
distorts the
texture so it doesn't look quite so perfect.  Try changing the sphere in the
above example to have the following texture:
\begin{verbatim}
         TEXTURE
            Dark_Wood
            TURBULENCE 0.0
            SCALE <0.2 0.2 0.2>
            PHONG 1.0
         END_TEXTURE
\end{verbatim}
When you compare this with the original image, you'll see that the pattern is
much more boring.

Finally, many textures use colour maps.\index{textures!colour maps}
A colour map translates a number
between 0.0 and 1.0 into a colour.  The number typically represents the
distance into a vein of colour -- the further into the vein you get, the more
the colour changes.  Here's a typical colour map.  Try this out on the sphere
defined above by changing the definition to this:
\begin{verbatim}
      OBJECT
         SPHERE <0 0 3> 1 END_SPHERE
         TEXTURE
            WOOD
            SCALE <0.2 0.2 0.2>
            COLOUR_MAP
                [0.0 0.3  COLOUR Red   COLOUR Green]
                [0.3 0.6  COLOUR Green COLOUR Blue]
                [0.6 1.01 COLOUR Blue  COLOUR Red]
            END_COLOUR_MAP
            PHONG 1.0
         END_TEXTURE
      END_OBJECT
\end{verbatim}
This means that as the texture enters into a vein of wood, it changes colour
smoothly from red to green, from green to blue, and from blue to red again.
As it leaves the vein, the transition occurs in reverse.  (Since there is no
turbulence on the wood by default, the veins of colour should show up quite
well.)

You can get more ``bang for your buck'' from textures by using
{\tt ALPHA}\keyindex{ALPHA} properties of colour.  Every
colour you define in DKBTrace is a
combination of red, green, blue and alpha.  The red, green and blue
are simple enough.  The alpha determines how transparent that colour
is.  A colour with an alpha of 1.0 is totally transparent.  A colour
with an alpha of 0.0 is totally opaque.  Here's a neat texture to try:
\begin{verbatim}
   TEXTURE
      TURBULENCE 0.5
      BOZO
      COLOUR_MAP
         { transparent to transparent }
         [0.0 0.6 COLOUR RED 1.0 GREEN 1.0 BLUE 1.0 ALPHA 1.0
                  COLOUR RED 1.0 GREEN 1.0 BLUE 1.0 ALPHA 1.0]
         { transparent to white }
         [0.6 0.8 COLOUR RED 1.0 GREEN 1.0 BLUE 1.0 ALPHA 1.0
                  COLOUR RED 1.0 GREEN 1.0 BLUE 1.0]
         { white to grey }
         [0.8 1.001 COLOUR RED 1.0 GREEN 1.0 BLUE 1.0
                    COLOUR RED 0.8 GREEN 0.8 BLUE 0.8]
      END_COLOUR_MAP
      SCALE <0.4  0.08  0.4>
   END_TEXTURE
\end{verbatim}
This is my (famous) cloud texture.\index{textures!famous cloud}
It creates white clouds with grey linings.
The texture is transparent in the places where the clouds disappear so you can
see through it to the objects that are behind.

Now for one more feature which is new for 2.10 (hold onto your hats!)  You can
now layer textures one on top of another to create more sophisticated textures.
\index{textures!layered}
For example, suppose I want a wood-coloured cloudy texture.  What I do is put
the wood texture down first followed by my cloud texture.  Wherever the cloud
texture is transparent, the wood texture shows through.  Change your sphere to
the following and you'll see.
\begin{verbatim}
  OBJECT
     SPHERE <0 0 3> 1 END_SPHERE
     TEXTURE                { This is the wood texture }
        Dark_Wood           { we used earlier.         }
        TURBULENCE 0.0
        SCALE <0.2 0.2 0.2>
        PHONG 1.0
     END_TEXTURE
     TEXTURE                { This is the cloud texture }
       TURBULENCE 0.5       { we just defined.          }
       BOZO
       COLOUR_MAP
         { transparent to transparent }
         [0.0 0.6 COLOUR RED 1.0 GREEN 1.0 BLUE 1.0 ALPHA 1.0
                  COLOUR RED 1.0 GREEN 1.0 BLUE 1.0 ALPHA 1.0]
         { transparent to white }
         [0.6 0.8 COLOUR RED 1.0 GREEN 1.0 BLUE 1.0 ALPHA 1.0
                  COLOUR RED 1.0 GREEN 1.0 BLUE 1.0]
         { white to grey }
         [0.8 1.001 COLOUR RED 1.0 GREEN 1.0 BLUE 1.0
                    COLOUR RED 0.8 GREEN 0.8 BLUE 0.8]
       END_COLOUR_MAP
       SCALE <0.4  0.08  0.4>
     END_TEXTURE
  END_OBJECT
\end{verbatim}
Each successive texture is layered on top of the previous textures.  In the
places where you can see through the upper texture, you see through to the
lower textures.

\section{Walk-through Wrap-up}

In this walk-through, I've only tried to show the highlights of the raytracer
without getting into all possible options and features.  To get all of those,
you'll have to read through the following section on the Scene Description
Language.  Hopefully it will be fairly straight forward now that you have a
feel for the language and how it works.  Hopefully you'll find that the
textual interface provided by DKBTrace isn't quite as scary as people think at
first.

\chapter{The Scene Description Language}

The Scene Description Language allows the user to describe the world in a
readable and convenient way. This chapter is effectively a reference
manual for DKBTrace.

The language delimits comments\index{comments} by the left and right
braces (``{\tt \{}'' and ``{\tt \}}''). Nested comments are allowed,
but no sane person uses them anyway, right?
The language allows include files to be specified by placing the line:
\keyindex{INCLUDE}
\begin{verbatim}
     INCLUDE "filename"
\end{verbatim}
at any point in the input file (include files may be nested).  The filename
must be enclosed in double quotes and may be up to 40 characters long,
including the two double-quote ({\tt "}) characters.  You may have at most 10
{\tt INCLUDE}'d files per scene trace, whether nested or not.

\section{The Basic Data Types}

There are several basic types of data.

\subsection{Floats}

\index{float data type}\index{data types!float}
Floats are represented by an optional sign ({\tt -}), some digits, an optional
decimal point, and more digits.  Version 2.10 now supports the
``{\tt e}'' notation for exponents.  The following are valid floats:
\begin{verbatim}
     1.0  -2.0  -4  34  3.4e6 2e-5
\end{verbatim}

\subsection{Vectors}

\index{vector data type}\index{data types!vector}
Vectors are arrays of three floats.  They are bracketed by angle brackets
(``{\tt <}'' and ``{\tt >}''), and the three terms usually represent x, y,
and z. For example:
\begin{verbatim}
     < 1.0  3.2  -5.4578 >
\end{verbatim}
Vectors typically refer to relative points.  For example, the vector:
\begin{verbatim}
     <1 2 3>
\end{verbatim}
means the point that's 1 unit to the right, 2 units up, and 3 units in front.
``Of what?'' you ask?  Well, usually it's relative to the center of the
``universe'' at {\tt <0 0 0>}.

\subsection{Transformations -- {\tt TRANSLATE}, {\tt ROTATE}, and {\tt SCALE}}

\index{transformations}\keyindex{TRANSLATE}\keyindex{ROTATE}\keyindex{SCALE}
In a few places here and there, vectors are used as a convenient notation but
don't represent a point in space. This is the case for the transformations
{\tt TRANSLATE}, {\tt ROTATE}, and {\tt SCALE}:

\begin{description}
\item[{\tt TRANSLATE <{\em x y z\/}>}] Move the object {\em x} units
to the right, {\em y} units up, and {\em z} units away from us.
\item[{\tt SCALE <{\em xs ys zs\/}>}] Scale the object by {\em xs}
units in the left/right direction, {\em ys} units in the up/down
direction and {\em zs} units in the front/back direction.
\item[{\tt ROTATE <{\em xr yr zr\/}>}] Rotate the object {\em xr}
degrees about the X axis, then {\em yr} degrees about the Y axis,
then {\em zr} degres about the Z axis (note that the order matters).
\end{description}

To work out the rotation directions, you must perform the famous ``Computer
Graphics Aerobics'' exercise.  Hold up your left hand.
\index{left-handed coordinate system}\index{coordinate system!left-handed}
point your thumb in the
positive direction of the axis you want to rotate about.  Your fingers will
curl in the positive direction of rotation.  This is the famous ``left-hand
coordinate system'', illustrated in Figure \ref{lefthand}.
\begin{figure}[htbp]
\begin{centering}
\setlength{\unitlength}{0.0125in}%
\begin{picture}(60,110)(180,610)
\thinlines
\put(230,700){\line( 0, 1){ 10}}
\put(230,710){\line( 1, 0){ 10}}
\put(240,710){\vector( 0,-1){ 10}}
\put(210,620){\line( 0,-1){ 10}}
\put(185,620){\line( 0,-1){ 10}}
\multiput(220,630)(-0.40000,-0.40000){26}{\makebox(0.1111,0.7778){\fivrm .}}
\multiput(180,625)(0.41667,-0.41667){13}{\makebox(0.1111,0.7778){\fivrm .}}
\put(180,625){\line( 0, 1){ 75}}
\put(180,700){\line( 1, 0){ 10}}
\put(190,700){\line( 0, 1){ 15}}
\put(190,715){\line( 1, 0){ 10}}
\put(200,715){\line( 0, 1){  5}}
\put(200,720){\line( 1, 0){ 10}}
\put(210,720){\line( 0,-1){ 15}}
\put(210,705){\line( 1, 0){ 10}}
\put(220,705){\line( 0,-1){ 60}}
\multiput(220,645)(0.41667,-0.41667){13}{\makebox(0.1111,0.7778){\fivrm .}}
\put(225,640){\line( 1, 0){ 15}}
\put(240,640){\line( 0,-1){ 10}}
\put(240,630){\line(-1, 0){ 20}}
\put(210,705){\line( 0,-1){ 55}}
\put(200,715){\line( 0,-1){ 65}}
\put(190,700){\line( 0,-1){ 50}}
\end{picture}

\caption{Left-hand Coordinate System}
\label{lefthand}
\end{centering}
\end{figure}
If you want to use a right-hand system, as some CAD
systems do, the {\tt RIGHT} vector in the {\tt VIEW_POINT} needs to
be changed.  See the
detailed description of the {\tt VIEW_POINT}, and use your right hand for the
``Aerobics''. 

\subsection{Colours}

\index{colour!data type}\index{data types!colour}
A colour consists of a red component, a green component, a blue
component, and possibly an alpha component.  All four components are
floats in the range 0.0 to 1.0.  The syntax for Colours is the word
{\tt COLOUR} followed by any or all of the {\tt RED}, {\tt GREEN},
{\tt BLUE} or {\tt ALPHA} components in any order. For example:
\begin{verbatim}
     COLOUR  RED 1.0  GREEN 1.0  BLUE 1.0
     COLOUR  BLUE 0.56
     COLOUR  GREEN 0.45 RED 0.3 BLUE 0.1 ALPHA 0.3
\end{verbatim}
{\tt ALPHA} is a transparency indicator.  If an object's colour
contains some transparency, then you can see through it.  If
{\tt ALPHA} is 0.0, the object is totally opaque.  If it is 1.0, it is
totally transparent.

For those people who spell ``Colour'' the American way as ``Color'',
the program will also accept {\tt COLOR}
\index{colour!and color}
as equivalent to {\tt COLOUR} in all instances.

\subsection{Colour Maps}

\index{colour map data type}\index{data types!colour map}
For wood, marble, spotted, agate, granite, and gradient texturing, the user
may specify arbitrary colours to use for the texture.  This is done by a
colour map or ``colour spline''.  When the object is being textured, a number
between 0.0 and 1.0 is generated which is then used to form the colour of the
point.  A Colour map specifies the mapping used to change these numbers into
colours. The syntax is as follows:
\begin{verse}
{\tt COLOUR_MAP} \\
{\tt \ \ \ \ \ [{\em start-value end-value colour1 colour2\/}]} \\
{\tt \ \ \ \ \ [{\em start-value end-value colour1 colour2\/}]} \\
\ \ \ \ \ ... \\
{\tt END_COLOUR_MAP}
\end{verse}
The numeric value between 0.0 and 1.0 is located in the colour map and the
final colour is calculated by a linear interpolation (a smooth blending)
between the two colours in the located range.

\section{More Complex Data Types -- {\tt DECLARE}}

\keyindex{DECLARE}\index{declarations}\index{data types!complex}
The data types used to describe the objects in the world are a bit more
difficult to describe.  To make this task easier, the program allows users to
describe these types in two ways.  The first way is to define it from first
principles specifying all of the required parameters.  The second way allows
the user to define an object as a modification of another object (the other
object is usually defined from first principles but is much simpler).  Here's
how it works.

You can use the term {\tt DECLARE} to declare a type of object with a
certain description.\index{declarations!first principles}
The object is not included in the world but is
made known to the program that it can be used as a ``prototype'' for
defining other objects by the name given, as this basic example shows:
\begin{verbatim}
     DECLARE Sphere = QUADRIC  { First principles definition }
          <1.0 1.0 1.0>        {   of a sphere               }
          <0.0 0.0 0.0>
          <0.0 0.0 0.0>
          -1.0
     END_QUADRIC
\end{verbatim}
To then reference the declaration elsewhere in your source file or in another
included one, and to actually include the object in the world, you would
define the object using the {\tt DECLARE}'d object's name, like this:
\begin{verbatim}
     OBJECT
          QUADRIC Sphere
               SCALE <20.0 20.0 20.0>
          END_QUADRIC
          TEXTURE
             COLOUR White
             AMBIENT 0.2
             DIFFUSE 0.8
          END_TEXTURE
     END_OBJECT
\end{verbatim}

The real power of declaration becomes apparent when you declare
several primitive types of objects and then define an object with
several component shapes, using either {\tt COMPOSITE} methods or the
CSG methods {\tt INTERSECTION}, {\tt UNION}, or {\tt DIFFERENCE}.
\index{declarations!composite}
Also, using the {\tt DECLARE} keyword to pre-define textures can make
several objects share a texture without the need for each object to
store a duplicate copy of the same texture, for more efficient memory
usage.\index{declarations!textures}
For example:
\begin{verbatim}
     DECLARE Dull TEXTURE  { A Basic, Boring Texture }
          AMBIENT 0.3
          DIFFUSE 0.7
     END_TEXTURE
     OBJECT         { A Hot dog in a Hamburger Bun (?) }
          UNION
               QUADRIC Sphere
                    SCALE <20.0, 10.0, 20.0>
            TEXTURE Dull END_TEXTURE
               END_QUADRIC
               QUADRIC Cylinder_X
                    SCALE <40.0, 20.0, 20.0>
            TEXTURE Dull END_TEXTURE
               END_QUADRIC
          END_UNION
     END_OBJECT
\end{verbatim}

Layered textures, new to version 2.10, may also be {\tt DECLARE}'d.
The {\tt DECLARE} mechanism keeps looking for successive {\tt TEXTURE}
definitions and will layer them onto the same texture until another
{\tt DECLARE} (or any other statement except another {\tt TEXTURE}) is
encountered in the input data file.\index{declarations!layered textures}
For example:
\begin{verbatim}
     DECLARE Cloud_Wood TEXTURE  { This is the cloudy wood }
       Dark_Wood                 {   texture used earlier. }
       TURBULENCE 0.0
       SCALE <0.2 0.2 0.2>
       PHONG 1.0
     END_TEXTURE
     TEXTURE            { This is layered onto the wood }
       TURBULENCE 0.5   {   texture just defined.       }
       BOZO
       COLOUR_MAP
         { transparent to transparent }
         [0.0 0.6 COLOUR RED 1.0 GREEN 1.0 BLUE 1.0 ALPHA 1.0
                  COLOUR RED 1.0 GREEN 1.0 BLUE 1.0 ALPHA 1.0]
         { transparent to white }
         [0.6 0.8 COLOUR RED 1.0 GREEN 1.0 BLUE 1.0 ALPHA 1.0
                  COLOUR RED 1.0 GREEN 1.0 BLUE 1.0]
         { white to grey }
         [0.8 1.001 COLOUR RED 1.0 GREEN 1.0 BLUE 1.0
                    COLOUR RED 0.8 GREEN 0.8 BLUE 0.8]
       END_COLOUR_MAP
       SCALE <0.4  0.08  0.4>
     END_TEXTURE

    DECLARE (etc.) { Ends the layered definition. }
\end{verbatim}

\section{Viewpoint}

\index{viewpoint}
The viewpoint tells the ray tracer the location and orientation of the
camera.  The viewpoint is described by four vectors --
{\tt LOCATION},\index{viewpoint!location}\keyindex{LOCATION}
{\tt DIRECTION},\index{viewpoint!direction}\keyindex{DIRECTION}
{\tt UP},\index{viewpoint!up}\keyindex{UP} and
{\tt RIGHT}.\index{viewpoint!right}\keyindex{RIGHT}
Location determines where the camera is
located.  Direction determines the direction that the camera is
pointed.  Up determines the ``up'' direction of the camera.  Right
determines the direction to the right of the camera.

A first principle's declaration of a viewpoint would look like this:
\index{viewpoint!first principles declaration of}
\begin{verbatim}
     VIEWPOINT
          LOCATION < 0.0  0.0  0.0>
          DIRECTION < 0.0  0.0  1.0>
          UP < 0.0  1.0  0.0 >
          RIGHT < 1.0  0.0  0.0>
     END_VIEWPOINT
\end{verbatim}
This format becomes cumbersome, however, because the vectors must be hand
calculated.  This is especially difficult when the vectors are not lined up
with the X, Y, and Z axes as they are in the above example. To make it easier
to define the viewpoint, you can define one viewpoint, then modify the
description.\index{viewpoint!transformation of}  For example,
\begin{verbatim}
     VIEWPOINT
          LOCATION < 0.0  0.0  0.0>
          DIRECTION < 0.0  0.0  1.0>
          UP < 0.0  1.0  0.0 >
          RIGHT < 1.0  0.0  0.0 >
          TRANSLATE < 5.0  3.0  4.0 >
          ROTATE < 30.0  60.0  30.0 >
     END_VIEWPOINT
\end{verbatim}
In this example, the viewpoint is created, then translated to another point in
space and rotated by 30 degrees about the X axis, 60 degrees about the Y axis,
and 30 degrees about the Z axis.

Unfortunately, even this is somewhat cumbersome.  So, in version 2.0 and
above, you can now specify two more parameters:

\begin{verse}
{\tt SKY} {\em vector} \\
{\tt LOOK_AT} {\em vector}
\end{verse}

The {\tt SKY}\keyindex{SKY} keyword tells the viewpoint where the sky
is. It tries
to keep the camera's {\tt UP} direction aligned as closely as possible
to the sky. The {\tt LOOK_AT}\keyindex{LOOK_AT} keyword tells the
camera to look at a
specific point.  The camera is rotated as required to point directly
at that point.  By changing the {\tt SKY} vector, you can twist the
camera while still looking at the same point.

One subtle point: the {\tt SKY} direction is not necessarily the same
as the {\tt UP} direction. For example, consider the situation shown
in Figure \ref{skyup}.
\begin{figure}[htbp]
\begin{centering}
\setlength{\unitlength}{0.0125in}%
\begin{picture}(35,112)(210,660)
\thinlines
\put(220,700){\vector( 1, 2){ 15}}
\put(220,700){\vector( 1,-2){ 20}}
\put(220,700){\vector( 0, 1){ 60}}
\put(210,695){\makebox(0,0)[lb]{\raisebox{0pt}[0pt][0pt]{\elvrm C}}}
\put(245,660){\makebox(0,0)[lb]{\raisebox{0pt}[0pt][0pt]{\elvrm O}}}
\put(240,735){\makebox(0,0)[b]{\raisebox{0pt}[0pt][0pt]{\elvrm U}}}
\put(220,765){\makebox(0,0)[b]{\raisebox{0pt}[0pt][0pt]{\elvrm S}}}
\end{picture}

\caption{The Difference between {\tt SKY} and {\tt UP}}
\label{skyup}
\end{centering}
\end{figure}
If you said that the camera C has a {\tt SKY} direction S and is
looking at O, the up direction will not point to the sky.  {\tt UP}'s
like putting an antenna on your camera.  If you point the camera
downwards, the antenna will no longer point straight up.

The {\tt RIGHT} vector, as was mentioned previously, controls the
aspect ratio of the screen display.  It also determines the
``handedness'' of the coordinate system in use.  A normal
(left-handed)\index{left-handed coordinate system}%
\index{coordinate system!left-handed}
 {\tt RIGHT} statement would be:
\begin{verbatim}
     RIGHT < 1.3333 0.0 0.0 >
\end{verbatim}
To use a right-handed coordinate system,\index{right-handed coordinate system}%
\index{coordinate system!right-handed}
as is popular in some CAD
programs and some other ray-tracers, such as Sculpt for the Amiga, use
a {\tt RIGHT} like:
\begin{verbatim}
     RIGHT < -1.3333 0.0 0.0 >
\end{verbatim}
Some CAD systems also have the peculiar conception that the Z axis is
the ``elevation'' and is the {\tt UP} direction instead of the Y axis.
\index{elevation, and {\tt UP} vector}
If this is the case you will want to change your {\tt UP} statement
to:
\begin{verbatim}
     UP < 0.0 0.0 1.0 >
\end{verbatim}
Note that a pinhole camera model\index{pinhole camera model}%
\index{camera model}
is used, so no focus or depth-of-field
effects are supported at this time.  For more detailed information on the
camera model, see the section ``How it All Works''.

\section{Fog}

\index{fog}
Version 2.0 of the raytracer includes the ability to render fog.  To add fog
to a scene, place the following declaration outside of any object definitions:
\begin{verse}
{\tt FOG} \\
{\tt\ \ \ \ \ COLOUR  {\em colour} \{ the fog colour \}}\\
{\tt\ \ \ \ \ 200.0   \{ the fog distance \}} \\
{\tt END_FOG}
\end{verse}

The fog to colour ratio is calculated as
$1 - e^{-depth / distance}$,
so at depth 0, the colour is pure (1.0)
with no fog (0.0).  At the fog distance, you'll get 63% of the colour
from the object's colour and 37% from the fog colour.

\section{Shapes}

Shapes describe the shape of an object without mentioning any surface
characteristics like colour, surface lighting and reflectivity.

\subsection{Quadric Shapes}

The most general shape used by this system is called a Quadric Surface.
Quadric Surfaces can produce shapes like spheres, cones, cylinders,
paraboloids (dish shapes), and hyperboloids (saddle or hourglass
shapes).

The easiest way to define these shapes is to include the standard file
{\tt shapes.dat}\index{shapes.dat@{\tt shapes.dat} definitions file} into
your program and to transform these shapes
(using {\tt TRANSLATE}, {\tt ROTATE}, and {\tt SCALE}) into the ones
you want.  To be complete, however, I will describe the mathematical
principles behind quadric surfaces.  Those who are not interested in
the mathematical details can skip to the next section.

The quadric\index{quadric surfaces!definition of}
\begin{verbatim}
     QUADRIC
         <A B C>
         <D E F>
         <G H I>
         J
     END_QUADRIC
\end{verbatim}
defines a surface in three dimensions which satisfies the following
equation
\begin{displaymath}
{\tt A}x^{2} + {\tt B}y^{2} + {\tt C}z^{2} +
{\tt D}xy + {\tt E}xz + {\tt F}yz + {\tt G}x + {\tt H}y +{\tt I}z +
{\tt J} = 0
\end{displaymath}
Did you really want to know that?  I didn't think so.

Different values of {\tt A}, {\tt B}, {\tt C}, \ldots {\tt J} will
give different shapes.  So, if you take any three dimensional point
and use its x, y, and z coordinates in the above equation, the answer
will be 0 if the point is on the surface of the object.  The answer
will be negative if the point is inside the object and positive if the
point is outside the object.  Here are some examples:
\begin{eqnarray*}
x^{2} + y^{2} + z^{2} - 1 & = & 0 \mbox{\ -- Sphere} \\
x^{2} + y^{2} - 1 & = & 0 \mbox{\ -- Cylinder along Z axis} \\
x^{2} + y^{2} - z^{2} & = & 0 \mbox{\ -- Cone along Z axis}
\end{eqnarray*}

\subsection{Quadric surfaces the easy way}

Now that doesn't sound so hard, does it?  Well, actually, it does.
Only the hard-core graphics fanatic would define his objects using the
{\tt QUADRIC} definition directly.  Even I don't do it that way and I
know how it works.\footnote{Well, at least I worked it out once
or twice.}

Fortunately, there is an easier way.
\index{quadric surfaces!transformation of}
The file {\tt shapes.dat} already
includes the definitions of many quadric surfaces.  They are centered
about the origin {\tt < 0 0 0 >} and have a radius of 1.  To use them,
you can define shapes simply as follows:
\begin{verbatim}
     INCLUDE "colors.dat"
     INCLUDE "shapes.dat"   { important for this example }
     INCLUDE "textures.dat"

     QUADRIC
          Cylinder_X
          SCALE < 50.0  50.0  50.0 >
          ROTATE < 30.0  10.0  45.0 >
          TRANSLATE < 2.0  5.0  6.0 >
     END_QUADRIC
\end{verbatim}
You may have as many transformation lines (scale, rotate, and translate) as
you like in any order.  Usually, however, it's easiest to do a scale first,
one or more rotations, then finally a translation.  Otherwise, the results may
not be what you expect. (The transformations always transform the object about
the origin.  If you have a sphere at the origin and you translate it then
rotate it, the rotation will spin the sphere around the origin like planets
about the sun).\index{transformations!order of}

\subsection{Spheres}

\index{sphere!primitive}\keyindex{SPHERE}
Since spheres are so common in ray traced graphics, a {\tt SPHERE}
primitive has been added to the system.  This primitive will render
much more quickly than the corresponding quadric shape.  The syntax
is:
\begin{verbatim}
     SPHERE  <center> radius END_SPHERE
\end{verbatim}

You can also add translations, rotations, and scaling to the sphere. For
example, the following two objects are identical:
\begin{verbatim}
     OBJECT
          SPHERE  < 0.0 25.0 0.0 > 10.0 END_SPHERE
          TEXTURE
             COLOR Blue
             AMBIENT 0.3
             DIFFUSE 0.7
          END_TEXTURE
     END_OBJECT
\end{verbatim}

\begin{verbatim}
     OBJECT
          SPHERE  < 0.0 0.0 0.0 > 1.0
               TRANSLATE <0.0  25.0  0.0> 
               SCALE <10.0  10.0  10.0>
          END_SPHERE
          TEXTURE
             COLOR Blue
             AMBIENT 0.3
             DIFFUSE 0.7
          END_TEXTURE
     END_OBJECT
\end{verbatim}
Note that Spheres may only be scaled uniformly.\index{sphere!scaling of}
You cannot use:
\begin{verbatim}
     SCALE <10.0 5.0 2.0>
\end{verbatim}
on a sphere.  If you need oblate (flattened) spheroids, use a scaled quadric
``Sphere'' shape instead, as it can be arbitrarily scaled in any dimension.

\subsection{Planes}

\index{plane!primitive}\keyindex{PLANE}
Another primitive provided to speed up the raytracing is the {\tt PLANE}.
This is a flat infinite plane.  To declare a {\tt PLANE}, you
specify the direction of the surface normal to the plane (the {\tt UP}
direction) and the distance from the origin of the plane to the
world's origin. As with spheres, you can translate, rotate, and scale
planes.  Examples:
\begin{verbatim}
     { plane in the X-Z axes 10 units below the origin. }
     PLANE <0.0  1.0  0.0> -10.0 END_PLANE
     { plane in the X-Z axes 10 units above the origin. }
     PLANE <0.0  1.0  0.0>  10.0 END_PLANE
     { plane in the X-Y axes 10 units behind the origin.}
     PLANE <0.0  0.0  1.0>  -10.0 END_PLANE
\end{verbatim}

\subsection{Triangles}

\index{triangle!primitive}\keyindex{TRIANGLE}
In order to make more complex objects than the class of quadrics will
permit, a new primitive shape for triangles has been added.  There are
two different types of triangles:  flat shaded triangles and smooth
shaded (Phong) triangles.

Flat shaded triangles are defined by listing the three vertices of the 
triangle.  For example:
\begin{verbatim}
     TRIANGLE  < 0.0   20.0  0.0>
               < 20.0  0.0   0.0>
               <-20.0  0.0   0.0>
     END_TRIANGLE
\end{verbatim}

The smooth shaded triangles
\index{triangle!smooth shaded primitive}\keyindex{SMOOTH_TRIANGLE}
use Phong Normal Interpolation to calculate the
surface normal for the triangle.  This makes the triangle appear to be a
smooth curved surface.  In order to define a smooth triangle, however, you
must supply not only the vertices, but also the surface normals at those
vertices.  For example:
\begin{verbatim}
     SMOOTH_TRIANGLE
          {      points             surface normals     }
          <  0.0  30.0  0.0 >    <0.0   0.7071   -0.7071>
          < 40.0 -20.0  0.0 >    <0.0   -0.8664  -0.5   >
          <-50.0 -30.0  0.0 >    <0.0   -0.5     -0.8664>
     END_SMOOTH_TRIANGLE
\end{verbatim}
As with the other shapes, triangles can be translated, rotated, and scaled.

\begin{description}
\item[Note 1:] Triangles cannot be used in CSG {\tt INTERSECTION}
or {\tt DIFFERENCE} types (described next) since triangles have no
clear ``inside''.\index{triangle!and CSG}
The CSG {\tt UNION} type works acceptably, but with no real
difference from a {\tt COMPOSITE} made up of {\tt TRIANGLE} primitives.
\item[Note 2:] To be honest, I don't expect mere mortals to work out
the surface normals of triangles.\index{triangle!surface normals}
Ideally, you'd have another program
generate them for you.  See the section on ``Common Questions and
Answers'' for details on how this might be done.
\end{description}

\subsection{Quartic Surfaces}

\index{quartic surfaces}\index{surfaces!quartic}
% use the underscores as usual
\catcode`\_=8
Quartic surfaces are 4th order surfaces, and can be used to describe a large
class of shapes including the torus, the lemniscate, etc.  The general
equation for a quartic equation in three variables is (hold onto your hat):
\begin{displaymath}
\begin{array}{cccccccccccccl}
a_{00}x^{4}      & + & a_{01}x^{3}y  & + &
a_{02}x^{3}z     & + & a_{03}x^{3}   & + &
a_{04}x^{2}y^{2} & + \\
a_{05}x^{2}yz    & + & a_{06}x^{2}y  & + &
a_{07}x^{2}z^{2} & + & a_{08}x^{2}z  & + &
a_{09}x^{2}      & + \\
a_{10}xy^{3}     & + & a_{11}xy^{2}z & + &
a_{12}xy^{2}     & + & a_{13}xyz^{2} & + &
a_{14}xyz        & + \\
a_{15}xy         & + & a_{16}xz^{3}  & + &
a_{17}xz^{2}     & + & a_{18}xz      & + &
a_{19}x          & + \\
a_{20}y^{4}      & + & a_{21}y^{3z}  & + &
a_{22}y^{3}      & + & a_{23}y^{2}z^{2} & + &
a_{24}y^{2}z     & + \\
a_{25}y^{2}      & + & a_{26}yz^{3}  & + &
a_{27}yz^{2}     & + & a_{28}yz      & + &
a_{29}y          & + \\
a_{30}z^{4}      & + & a_{31}z^{3}   & + &
a_{32}z^{2}      & + & a_{33}z       & + &
a_{34}           & = 0
\end{array}
\end{displaymath}
To declare a quartic surface requires that each of the coefficients
($a_{00} \ldots a_{34}$) be placed in order into a single long vector
of 35 terms.

As an example, the following object declaration is for a torus having major
radius 6.3 minor radius 3.5 (Making the maximum width just under 10).
\index{quartic surfaces!first principles declaration of}
\keyindex{QUARTIC}
\begin{verbatim}
     { Torus with major radius sqrt(40),
                  minor radius sqrt(12) }
     OBJECT
      QUARTIC
      < 1.0   0.0   0.0    0.0     2.0
        0.0   0.0   2.0    0.0  -104.0
        0.0   0.0   0.0    0.0     0.0
        0.0   0.0   0.0    0.0     0.0
        1.0   0.0   0.0    2.0     0.0
       56.0   0.0   0.0    0.0     0.0
        1.0   0.0 -104.0   0.0   784.0 >
      END_QUARTIC
      BOUNDED_BY
       SPHERE <0 0 0> 10 END_SPHERE
      END_BOUND
     END_OBJECT
\end{verbatim}
The code to calculate Ray-Surface intersections for quartics is somewhat
slower than that for intersections with quadric surfaces.  Benchmarks on
a stock 8Mhz AT (with FPU) give results of around 1400 solutions per second
for 2nd order equations (quadrics) vs 444 per second for 3rd order equations
and 123 per second for 4th order equations (quartics).  So clever use of
bounding shapes can make a big difference in processing
time.\index{processing speed}\index{speed|see{processing speed}}

While a great deal of time has gone into debugging the code for handling
quartic surfaces, I know for a fact that there are bad combinations of
surface equations and lighting orientations that cause math errors (crash).
If this happens to you, as the joke goes, ``then don't do that.''
\index{errors}\index{quartic surfaces!math errors in}

Here are some surfaces to get you started. A Torus can be represented
by the equation:\index{torus}
\begin{eqnarray*}
& & x^{4} + y^{4} + z^{4} + 2x^{2}y^{2} + 2x^{2}z^{2} + 2y^{2}z^{2} - \\
& & 2(r_{0}^{2} + r_{1}^{2})x^{2}
+ 2(r_{0}^{2} - r_{1}^{2})y^{2}
- 2(r_{0}^{2} - r_{1}^{2})z^{2} + \\
& & \hspace*{2.2in}(r_{0}^{2} - r_{1}^{2})^{2} = 0 
\end{eqnarray*}

Where $r_{0}$ is the ``major'' radius of the torus -- the distance
from the hole of the donut to the middle of the ring of the donut, and
$r_{1}$ is the ``minor'' radius of the torus -- the distance from the
middle of the ring of the donut to the outer surface (see Figure
\ref{torus}).
\begin{figure}[htbp]
\begin{centering}
\setlength{\unitlength}{0.0125in}%
\begin{picture}(45,69)(180,705)
\thinlines
\put(200,740){\circle{20}}
\put(200,740){\circle{40}}
\put(220,710){\vector(-4, 3){ 20}}
\put(215,765){\vector(-3,-4){ 15}}
\put(185,740){\vector(-1, 0){  0}}
\put(185,740){\vector( 1, 0){ 30}}
\put(225,705){\makebox(0,0)[lb]{\raisebox{0pt}[0pt][0pt]{\elvrm Minor radius}}}
\put(220,765){\makebox(0,0)[lb]{\raisebox{0pt}[0pt][0pt]{\elvrm Major radius}}}
\end{picture}

\caption{Radii of a torus}
\label{torus}
\end{centering}
\end{figure}
Described
another way, a torus is a circle that is revolved around an axis.  The
major radius is the distance from the axis to the center of the
circle, the minor radius is the radius of the circle.

Note that scaling surfaces like a torus scales everything. In order to change
the relationship between the size of the hole and the size of the ring, it is
necessary to enter new coefficients for the torus.\index{torus!scaling}

The only coefficients of the 35 that need to be filled in
are:
\begin{displaymath}
\begin{array}{ccccccl}
a_{00} & = & a_{20} & = & a_{30} & = & 1 \\
a_{04} & = & a_{07} & = & a_{23} & = & 2 \\
a_{09} & = & a_{32} &   &        & = & -2(r_{0}^{2} + r_{1}^{2}) \\
a_{25} &   &        &   &        & = & 2(r_{0}^{2} - r_{1}^{2}) \\
a_{34} &   &        &   &        & = & 2(r_{0}^{2} - r_{1}^{2})^{2}
\end{array}
\end{displaymath}
The torus can then be rotated and translated into position once its shape
has been defined. See the file {\tt TORUS.DAT} for examples.

A generalization of the lemniscate of Gerono can be represented by the
equation:\index{lemniscate curve}
\begin{displaymath}
c^{2}x^{4} - a^{2}c^{2}x^{2} + y^{2} + z^{2} = 0
\end{displaymath}
where good start values are $a = 1.0$ and $c = 1.0$, giving the
coefficients:
\begin{displaymath}
\begin{array}{ccccccl}
a_{00} & = & a_{25} & = & a_{32} & = & 1 \\
a_{09} &   &        &   &        & = & -1 \\
\end{array}
\end{displaymath}
See the example file {\tt LEMNISCA.DAT} for a more complete example.

A generalization of a piriform can be represented by the
equation:\index{piriform curve}
\begin{displaymath}
-bc^{2}x^{4} - ac^{2}x^{3} + y^{2} + z^{2} = 0
\end{displaymath}
where good start values are $a = 4$, $b = -4$, $c = 1$, giving the
coefficients
\begin{displaymath}
\begin{array}{ccccl}
a_{00} &   &        & = & 4 \\
a_{03} &   &        & = & -4 \\
a_{25} & = & a_{32} & = & 1 \\
\end{array}
\end{displaymath}
See the file {\tt PIRIFORM.DAT} for more on this\ldots
% back to leaving the underscores as they are
\catcode`\_=12

There are really so many different quartic shapes, we can't even begin to
list or describe them all.  If you are interested and mathematically inclied,
an excellent reference book for curves and surfaces where you'll find more
quartic shape formulas is:\index{Seggern@von Seggern, David}
\begin{verse}
{\em The CRC Handbook of Mathematical Curves and Surfaces} \\
David von Seggern, CRC Press, 1990.
\end{verse}

\subsection{Constructive Solid Geometry}

\index{constructive solid geometry}\index{CSG|see{constructive solid geometry}}
This ray tracer supports Constructive Solid Geometry (CSG) in order to
make the object definition abilities more powerful.  Constructive
Solid Geometry allows you to define shapes which are the union,
intersection, or difference of other shapes.  Unions superimpose the
two shapes.  This has the same effect as defining two separate
objects, but is simpler to create and/or manipulate.  Intersections
define the space where the two surfaces meet.  Differences allow you
to cut one object out of another.

CSG Intersections, Unions, and Differences can consist of two or more
shapes.
\index{constructive solid geometry!union}\keyindex{UNION}
\index{constructive solid geometry!difference}\keyindex{DIFFERENCE}
\index{constructive solid geometry!intersection}\keyindex{INTERSECTION}
They are defined as follows:
\begin{verbatim}
     OBJECT
          INTERSECTION
               QUADRIC
                    ...
               END_QUADRIC

               QUADRIC
                    ...
               END_QUADRIC

               QUADRIC
                    ...
               END_QUADRIC
          END_INTERSECTION
          ...
     END_OBJECT
\end{verbatim}
{\tt UNION} or {\tt DIFFERENCE} may be used instead of
{\tt INTERSECTION}.  The order of the shapes doesn't matter except for the
{\tt DIFFERENCE} shapes.  For CSG differences, the first shape is
visible and the remaining shapes are cut out of the first (in reverse
order from version 1.2 {\tt DIFFERENCE}'s).

Constructive solid geometry shapes may be translated, rotated, or scaled in
the same way as a Quadric surface.  The quadric surfaces making up the CSG
object may be individually translated, rotated, and scaled as
well.\index{constructive solid geometry!transforming}

When using CSG, it is often useful to invert an shape so that it's
inside-out.  The {\tt INVERSE}\keyindex{INVERSE} keyword can be
used to do this for any
{\tt SPHERE}, {\tt PLANE}, or {\tt QUADRIC}.  When {\tt INVERSE} is
used, the ``inside'' of the object is flipped to become the
``outside''.  For Planes\index{plane!inside and outside}, ``inside'' is
defined to be in the opposite direction from the ``normal'' or ``up''
direction.

Note that performing an {\tt INTERSECTION} between an shape and some
other {\tt INVERSE} shapes is the same as performing a
{\tt DIFFERENCE}.  In fact, the {\tt DIFFERENCE} is actually implemented in
this way in the code.

\section{Objects}

\index{objects}
There is more to defining an object than just its shape.  You must tell the
ray tracer about the properties of the object like surface colour, alpha,
reflectivity, refractivity, the index of refraction, and so on.  This may be
done by specifying it in the shape or the object.  Generally, an
{\tt OBJECT}\keyindex{OBJECT}
definition will contain two pieces of information about an object: First, what
shape it is, and second, what it's looks like (on the surface and throughout).
A typical object definition looks something like this:
\begin{verbatim}
     OBJECT
          QUADRIC Sphere
               TRANSLATE < 40.0 40.0 60.0 >
          END_QUADRIC

          TEXTURE
             AMBIENT  0.3
             DIFFUSE   0.7
             REFLECTION  0.3
             REFRACTION  0.3
             IOR 1.05
             COLOUR RED 1.0 GREEN 1.0 BLUE 1.0 ALPHA 0.5
          END_TEXTURE
     END_OBJECT
\end{verbatim}

\noindent
The following keywords may be used when defining objects:

\subsection{{\tt COLOUR}}

\index{objects!colour of}\keyindex{COLOUR}
When used on an {\tt OBJECT} (i.e., not inside a {\tt TEXTURE} block),
the {\tt COLOUR} keyword determines what colour to use for a low
quality rendering when {\tt TEXTURE}'s are ignored.  See the
{\tt +q}\optindex{q}
option for details on setting the low-quality option. Examples:
\begin{verbatim}
     COLOUR RED 1.0  BLUE 0.4
\end{verbatim}
or
\begin{verbatim}
     DECLARE Yellow = COLOUR RED 1.0 GREEN 1.0
          ...
     COLOUR Yellow
\end{verbatim}

\subsection{{\tt TRANSLATE}, {\tt ROTATE}, and {\tt SCALE}}

\index{objects!transforming}
Objects can be translated, rotated, and scaled just like surfaces.
If\index{objects!transforming!and textures}
an object is transformed, all textures attached to the object at that
time are transformed as well.  This means that if you have a
{\tt TRANSLATE}\keyindex{TRANSLATE},
{\tt ROTATE}\keyindex{ROTATE},
or {\tt SCALE}\keyindex{SCALE} before a {\tt TEXTURE}, the
texture will {\em not} be transformed.  If the {\tt SCALE},
{\tt TRANSLATE}, or {\tt ROTATE} is after the {\tt TEXTURE}, the texture
will be transformed.

\subsection{{\tt LIGHT_SOURCE}}

\index{objects!as light sources|see{light sources}}
If the {\tt LIGHT_SOURCE}\keyindex{LIGHT_SOURCE} keyword is used
in the definition of an
object, then the object is included in the list of light sources.  It
can light objects and produce shadows.  (You should also specify the
{\tt COLOUR} of the light source, it will usually be ``White'').
Light sources have a peculiar
restriction:\index{light sources!position restriction}
The light source {\em must}
be {\tt TRANSLATE}'d to its final position in the scene, so the normal
way to define a light source is a sphere or quadric centered about the
origin, then {\tt TRANSLATE}'d to where desired in the final scene.
For example:
\begin{verbatim}
     OBJECT
          { we use a sphere, could use a quartic, too }
          SPHERE <0.0  0.0  0.0> 2.0 END_SPHERE
          TRANSLATE <100.0  120.0  40.0>

          LIGHT_SOURCE
          COLOUR White     { colour of the light emitted }
          TEXTURE
             COLOUR White  { surface colour of the sphere }
             AMBIENT 1.0
             DIFFUSE 0.0
          END_TEXTURE
     END_OBJECT
\end{verbatim}

\begin{description}
\item[Note:] \index{light sources!colour}You {\em must}
specify the colour of the light outside the
{\tt TEXTURE} block. This allows the renderer to quickly determine the
colour of the light source without having to plow through the
textures.  Any colour information inside the {\tt TEXTURE} block is
used to render the light source object itself if it is visible in the
scene.  The subtle difference between the actual sphere object and the
light ray emanation point (the center of the sphere) is why
{\tt LIGHT_SOURCE}'s must be defined at {\tt <0,0,0>} then
{\tt TRANSLATE}'d to where you want them.  It ties together and
{\tt TRANSLATE}'s both the object itself and the light ray source point to
the specified point in the scene.  Usually, light sources have an
{\tt AMBIENT} value of 1.0 and a {\tt DIFFUSE} of 0.0, but this is not a
hard and fast rule.
\end{description}

\section{Texture}

\index{textures}
The texture feature is an experiment into functionally based modelling.  This
feature allows you to assign colouring schemes to objects.  Many procedural
surface textures are provided, and by using different colour maps with them,
nearly infinite permutations are possible.  For example, you can make some
object look like wood or marble, etc.  In DKBTrace, any parameter that changes
the appearance of the surface {\em must} be put into a
{\tt TEXTURE}\keyindex{TEXTURE} block.

The basic {\tt TEXTURE} syntax is as follows:
\begin{verbatim}
     TEXTURE
          0.05
          WOOD
          TURBULENCE 0.2
          TRANSLATE < 1.0 2.0 3.0 >
          ROTATE < 0.0 10.0 40.0 >
          SCALE < 10.0 10.0 10.0 >
     END_TEXTURE
\end{verbatim}
Transformations are optional.  They allow you to transform the texture
independent of the object itself.  If you are doing animation, then the
transformations should be the same as the object transformations so that the
texture follows the object through 3-D space.\index{textures!transforming}

The\index{textures!randomness}
floating-point value given immediately following the texture
keyword is an optional ``texture randomness'' value, which causes a
minor random scattering of calculated colour values and produces a
sort of ``dithered'' appearance.  Note this is {\bf bad, bad, bad} for
animations!!  This is the {\em only} ``truly random'' thing in all of
DKB, and will produce a most annoying flicker of flying pixels on any
textures animated with a ``randomness'' value used.

Instead\index{textures!types} of using
{\tt WOOD},\keyindex{WOOD}\index{textures!coloring!wood} you may use
{\tt MARBLE},\keyindex{MARBLE}\index{textures!coloring!marbled}
{\tt BOZO}\keyindex{BOZO},
{\tt CHECKER}\keyindex{CHECKER}, or a handful of other interesting
textures.  The
{\tt WOOD} and {\tt MARBLE} textures are perturbed by a turbulence
function.  This makes them look more random and irregular than they
would normally appear.  The amount of turbulence\index{textures!turbulence}
can be changed by the
{\tt TURBULENCE}\keyindex{TURBULENCE} keyword followed by a number.
Values from 0.1 to 0.3
seem to give the best results.  The default is 0.0, or no turbulence.

Note some of the textures given are coloration textures, such as
{\tt MARBLE}, {\tt WOOD}, {\tt CHECKER}, {\tt GRANITE}, and {\tt AGATE}.
These work with colour maps\index{textures!colour maps}, and have
default ``colour maps'' that
they resort to if none are given.  The rest of the textures available
are ``surface perturbation'' textures, and do not directly affect the
colour of the object, but rather the surface's apparent orientation in
space.  Examples of these are {\tt WAVES}, {\tt RIPPLES}, {\tt DENTS},
{\tt BUMPS}, and {\tt WRINKLES}.  Note\index{textures!combining} that
any given texture may
include up to two actual textures, one coloration and one surface
perturbation choice per texture.  This would allow rippled wood, or
dented granite combinations, etc., but, keep in mind that any texture
transformations applied to one texture (i.e. {\tt TRANSLATE} or
{\tt SCALE}) will also transform the other one in the same fashion.

As of version 2.10, it is possible to create layered\index{textures!layered}
textures. If you use more
that one texture block, the raytracer will compute the colour of the last
texture and if there's any transparency in the colour (i.e., any alpha), it
will mix in some of the colour from the underlying
textures.

\begin{description}
\item[Important Note:] As of version 2.10, the keywords in this
following section {\em cannot} be used outside of a
{\tt TEXTURE}-{\tt END_TEXTURE} structure. This is a change in the
input language from prior versions.
\end{description}

\subsection{Object Surface Lighting Characteristics}

The following object surface lighting characteristics are available.
Each keyword should be followed by a value in the specified range.

\subsubsection{{\tt AMBIENT}}

\keyindex{AMBIENT}\index{objects!surface lighting!ambient}
Ambient light is light that is scattered everywhere in the room.  An object
lit only by ambient light will appear to have the same brightness over the
entire surface.  The default value is very little ambient light (0.3).  The
value can range from 0.0 to 1.0. 

\subsubsection{{\tt DIFFUSE}}

\keyindex{DIFFUSE}\index{objects!surface lighting!diffuse}
Diffuse light is light coming from a light source that is scattered in all
directions.  An object lit only by diffuse light looks like a rubber ball with
a spot light shining on it.  The value can range from 0.0 to 1.0.  By default,
there is mostly diffuse lighting (0.7).

\subsubsection{{\tt BRILLIANCE}}

\keyindex{BRILLIANCE}\index{objects!surface lighting!brilliance}
Objects can be made to appear more metallic by increasing their brilliance.
This controls the tightness of the basic diffuse illumination on objects and
minorly adjusts the appearance of surface shininess.  The default value is
1.0.  Higher values from 3.0 to about 10.0 can give objects a somewhat more
shiny or metallic appearance.  This is best used in concert with either the
{\tt SPECULAR}\keyindex{SPECULAR} or {\tt PHONG}\keyindex{PHONG} highlighting.

\subsubsection{{\tt REFLECTION}}

\keyindex{REFLECTION}\index{objects!surface lighting!reflection}
By setting the reflection value to be non-zero, you can give the object a
mirrored finish.  It will reflect all other objects in the room.  The value
can range from 0.0 to 1.0.  By default there is no reflection. 

\subsubsection{{\tt REFRACTION}}

\keyindex{REFRACTION}\index{objects!surface lighting!refraction}
By setting the refraction value to be non-zero, the object is made transparent.
Light will be refracted through the object like a lens.  The value can be set
between 0.0 and 1.0.  There is no refraction by default. 

\begin{description}
\item[Note 1:] New for
2.10:\index{objects!surface lighting!refraction, and
{\tt ALPHA}}\keyindex{ALPHA}
In order to refraction to work properly, you must have
some {\tt ALPHA} component in the surface colour.  In the places where the
{\tt ALPHA} is high, the refracted light can get through.  In places where
the {\tt ALPHA} is low, the refracted light is suppressed.  This is a
change in the input language from prior versions.
\item[Note 2:] The refracted light is filtered by (takes on) the
surface colour.
\item[Note 3:] In\index{objects!surface lighting!refraction, and {\tt IOR}}
layered textures, the {\tt REFRACTION} and
{\tt IOR} keywords {\em must} be in the
last texture, otherwise they will not take effect.
\item[Note 4:] If a texture has an {\tt ALPHA} component and no value
for {\tt REFRACTION} was
supplied, the renderer will simply transmit the ray through the
surface with no bending.
\end{description}

\subsubsection{{\tt IOR}}

\keyindex{IOR}\index{objects!surface lighting!index of refraction}
If the object is refracting light, then the {\tt IOR} or ``index of
refraction'' should be set.  This determines how dense the object is.
A value of 1.0 will give no refraction.  The index of refraction for
air is 1.0, water is 1.33, glass is 1.5, and diamond is 2.4.

\subsubsection{{\tt PHONG}}

\keyindex{PHONG}\index{objects!surface lighting!phong highlighting}
Controls the amount of Phong Specular Reflection highlighting on the
object.  Causes bright shiny spots on the object, the colour of the
light source that is being reflected.  The size of the spot is defined
by the value given for {\tt PHONGSIZE} below.  {\tt PHONG}'s value is
typically from 0.0 to 1.0, where 1.0 causes complete saturation of the
object's colour to the light source's colour at the brightest area
(center) of the highlight.  There is no {\tt PHONG} highlighting given
by default.

\subsubsection{{\tt PHONGSIZE}}

\keyindex{PHONGSIZE}\index{objects!surface lighting!phong highlighting}
Controls the size of the {\tt PHONG} Highlight on the object, sort of
an arbitrary ``glossiness'' factor.  Values range from 1.0 (Very Dull)
to 100 (Highly Polished).  Default {\tt PHONGSIZE} is 40 (plastic?) if
not specified.  This is simulating the fact that slightly reflective
objects, especially metallic ones, have microscopic facets, some of
which are facing in the mirror direction.  The more that are facing
that way, the shinier the object appears, and the tighter the specular
highlights become.  Phong measures the average of facets facing in the
mirror direction from the light sources to the viewer.

\subsubsection{{\tt SPECULAR}}

\keyindex{SPECULAR}\index{objects!surface lighting!specular reflection}
Very similar to {\tt PHONG} Specular Highlighting, but a better model
is used for determining light ray/object intersection, so a more
credible spreading of the highlights occur near the object horizons,
supposedly.  {\tt PHONG} is thus included for mostly academic reasons,
but try them both and you decide which you like better.  This effect
is most obvious for light sources behind objects.  The size of the
spot is defined by the value given for {\tt ROUGHNESS} below.  Like
{\tt PHONG}, {\tt SPECULAR} values are typically from 0.0 to 1.0 for
full saturation.  Default is no {\tt SPECULAR} highlighting.

Note that Specular and Phong highlights are {\em not} mutually
exclusive.  It is possible to specify both and they will both take
effect.  Normally, however, you will only specify one or the other.

\subsubsection{{\tt ROUGHNESS}}

\keyindex{ROUGHNESS}\index{objects!surface lighting!specular reflection}
Controls the size of the {\tt SPECULAR} Highlight on the object,
relative to the object's ``roughness''.  Values range from 1.0 (Very
Rough) to 0.001 (Very Smooth).  The default value, if not specified,
is 0.05 (Plastic?).  The roughness or average directional distribution
of the microfacets facing in the same direction as the perpendicular
surface ``normal'' cause the most notable reflection of the highlight
to the observer.


\subsubsection{{\tt METALLIC}}

\keyindex{METALLIC}\index{objects!surface lighting!metallic}
This keyword indicates that the colour of the specular and phong
hightlights will be the surface colour instead of the lightsource
colour.  This creates a metallic appearance.  When using this feature,
you should set {\tt AMBIENT} to about 0.5 and set {\tt DIFFUSE} to
0.0. This keyword does not take a value.

\subsection{Surface Colouring Textures}

The first type of advanced procedural textures provided by DKBTrace
are surface coloring textures. These affect the colour of an object.

\subsubsection{{\tt CHECKER}}

\keyindex{CHECKER}\index{textures!coloring!checkered}
{\tt CHECKER} texturing gives a checker-board appearance.  This option
works best on planes.  When using the {\tt CHECKER} texturing, you
must specify two colours immediately following the word {\tt CHECKER}.
These colours are the colours of alternate squares in the checker
pattern.  The default orientation of the {\tt CHECKER} texture is on
an X-Z plane (good for ground work, etc.) but to use it on an object
which has mostly X-Y orientation (such as a sphere, for instance), you
must {\tt ROTATE} the texture. To rotate the {\tt CHECKER} texture
onto an X-Y plane:
\begin{verbatim}
     TEXTURE
          CHECKER COLOUR White COLOUR Red
          SCALE <10.0 10.0 10.0>
          ROTATE <-90.0 0.0 0.0>   { now in the X-Y plane }
     END_TEXTURE
\end{verbatim}

\subsubsection{{\tt CHECKER_TEXTURE}}

\keyindex{CHECKER_TEXTURE}\index{textures!coloring!checkered}
I've had many requests for a checker pattern to allow you to alternate
between wood and marble or any other two textures.  So, in version
2.10, I've added a texture called {\tt CHECKER_TEXTURE} that takes
two textures instead of two colours.

In order to support layered textures, I've made the syntax a bit more verbose
than it would be otherwise.  The syntax is:
\begin{verbatim}
     TEXTURE
        CHECKER_TEXTURE
           TEXTURE ... texture ... END_TEXTURE
           TEXTURE ... optional layers ... END_TEXTURE
        TILE2
           TEXTURE ... second texture ... END_TEXTURE
           TEXTURE ... optional layers ... END_TEXTURE
        END_CHECKER_TEXTURE
        AMBIENT ...
        DIFFUSE ...
    END_TEXTURE
\end{verbatim}
Note that the textures in {\tt CHECKER_TEXTURE} only use the surface
colouring texture information.  Information about {\tt AMBIENT},
{\tt DIFFUSE}, {\tt REFLECTION}, etc.  and surface normal information
({\tt WAVES}, {\tt RIPPLES}) are ignored inside the {\tt CHECKER_TEXTURE}.
(Hey, what do you want for a 10 minute change?)

\subsubsection{{\tt BOZO}}

\keyindex{BOZO}\index{textures!coloring!noisy}
{\tt BOZO} texture basically takes a noise function and maps it onto
the surface of an object.  This ``noise'' is well-defined for every
point in space.  If two points are close together, they will have
noise values that are close together.  If they are far apart, their
noise values will be fairly random relative to each other.

As mentioned above, for coloration textures such as {\tt WOOD},
{\tt MARBLE}, and {\tt BOZO}, etc., you may change the colouring scheme by
using a colour map.\index{textures!colour maps}
This map allows you to convert numbers from 0.0
to 1.0 (which are generated by the ray tracer) into ranges of colours.
For example, the default {\tt BOZO} colouring can be specified by:
\begin{verbatim}
     TEXTURE
          BOZO
          COLOUR_MAP
               [0.0 0.4 COLOUR White COLOUR White]
               [0.4 0.6 COLOUR Green COLOUR Green]
               [0.6 0.8 COLOUR Blue COLOUR Blue]
               [0.8 1.0 COLOUR Red COLOUR Red]
          END_COLOUR_MAP
     END_TEXTURE
\end{verbatim}

The easiest way to see how it works is to try it.  With a good choice of
colours it produces some of the most realistic looking cloudscapes you have
ever seen indoors!  Try a cloud color map such as:
\begin{verbatim}
     TEXTURE
          BOZO
          TURBULENCE 1.0    { A blustery day.  For a calmer }
          COLOUR_MAP        {   one, try 0.2.                }
               { blue to blue }
               [0.0 0.5  COLOUR RED 0.5 GREEN 0.5 BLUE 1.0
                         COLOUR RED 0.5 GREEN 0.5 BLUE 1.0]
               { blue to white }
               [0.5 0.6  COLOUR RED 0.5 GREEN 0.5 BLUE 1.0
                         COLOUR RED 1.0 GREEN 1.0 BLUE 1.0]
               { white to grey }
               [0.6 1.001 COLOUR RED 1.0 GREEN 1.0 BLUE 1.0
                          COLOUR RED 0.5 GREEN 0.5 BLUE 0.5]
          END_COLOUR_MAP
          SCALE <800.0 800.0 800.0>
          TRANSLATE <200.0 400.0 100.0>
     END_TEXTURE
\end{verbatim}
The color map above indicates that for small values of texture, use a sky blue
color solidly until about halfway turbulent, then fade through to white on a
fairly narrow range.  As the white clouds get more turbulent and solid towards
the center, pull the color map toward grey to give them the appearance of
holding water vapor (like typical clouds). 
Check out {\tt sunset.dat} for a really neat (but slow) sky pattern
using {\tt ALPHA}.\keyindex{ALPHA}

\subsubsection{{\tt SPOTTED}}

\keyindex{SPOTTED}\index{textures!coloring!spotted}
Spotted texture is a sort of swirled random spotting of the colour of the
object.  If you've ever seen a metal organ pipe up close you know about what
it looks like (a galvanized garbage can is close). Play with this one, it
might render a decent cloudscape during a very stormy day.  No extra
keywords are required.  Should work with colour maps.  With small scaling
values, looks like masonry or concrete.

\subsubsection{{\tt AGATE}}

\keyindex{AGATE}\index{textures!coloring!marbled}
This texture is similar to {\tt MARBLE}, but uses a different turbulence
function.  The {\tt TURBULENCE} keyword has no effect, and as such it
is always very turbulent.\keyindex{TURBULENCE}

\subsubsection{{\tt GRADIENT}}

\keyindex{GRADIENT}\index{textures!coloring!gradient}
This is a specialized texture that uses approximate local coordinates
of an object to control colour map gradients.  This texture {\em only}
works with colour maps (one {\em must} be given!) and has a special
{\tt <X,Y,Z>} triple given after the {\tt GRADIENT} keyword, which
specifies any (or all) axes to perform the gradient action on.
Example: a Y gradient {\tt <0.0 1.0 0.0>} will give an ``altitude
colour map'', along the Y axis.  Values given other than 0.0 are taken
as 1.0.

For smooth repeating gradients, you should use a ``circular'' colour map, that
is, one in which the first colour value (0.0) is the same as the last one
(1.001) so that it ``wraps around'' and will cause smooth repeating gradient
patterns.  Scaling the texture is normally required to achieve the number of
repeating shade cycles you want.  Transformation of the texture is useful to
prevent a ``mirroring'' effect from either side of the central 0 axes.  Here is
an example of a gradient texture that uses a sharp ``circular'' color mapped
gradient rather than a smooth one, and uses both X and Y gradients to get a
diagonally-oriented gradient.  It produces a dandy candy cane
texture!\index{textures!coloring!candy cane}
\begin{verbatim}
     TEXTURE
          GRADIENT < 1.0 1.0 0.0 >
          COLOUR_MAP
              [0.00 0.25  COLOUR RED 1.0 GREEN 0.0 BLUE 0.0
                          COLOUR RED 1.0 GREEN 0.0 BLUE 0.0]
              [0.25 0.75  COLOUR RED 1.0 GREEN 1.0 BLUE 1.0
                          COLOUR RED 1.0 GREEN 1.0 BLUE 1.0]
              [0.75 1.001 COLOUR RED 1.0 GREEN 0.0 BLUE 0.0
                          COLOUR RED 1.0 GREEN 0.0 BLUE 0.0]
          END_COLOUR_MAP
          SCALE <30.0 30.0 30.0>
          TRANSLATE <30.0 -30.0 0.0>
     END_TEXTURE
\end{verbatim}

You may also specify a {\tt TURBULENCE}\keyindex{TURBULENCE}
value with the gradient to give a more
irregular colour gradient.  This may help to do neat things like fire or
corona effects.

\subsubsection{{\tt GRANITE}}

\keyindex{GRANITE}\index{textures!coloring!granite}
This texture uses a simple $1/f$ fractal noise function to give a pretty darn
good grey granite texture.  Typically used with small scaling values (2.0 to
5.0).  Also looks good with a little dithering (texture randomness).  Should
work with colour maps, so try your hand at pink granite or alabaster! 

\subsection{Surface Perturbation Textures}

The other type of advanced procedural texture provided by DKBTrace are
surface perturbation textures. These affect the 3D appearance of an
object, rather than its colour directly. A surface perturbation
texture can be used in conjunction with the above coloration textures.

\subsubsection{{\tt RIPPLES}}

\keyindex{RIPPLES}\index{textures!perturbation!rippled}
{\tt RIPPLES} makes a surface look like ripples of water.  The
{\tt RIPPLES} option requires a value to determine how deep the ripples
are:
\begin{verbatim}
     TEXTURE
          WOOD
          RIPPLES 0.3
          TRANSLATE < 1.0 2.0 3.0 >
          ROTATE < 0.0 10.0 40.0 >
          SCALE < 10.0 10.0 10.0 >
     END_TEXTURE
\end{verbatim}
In this case, the {\tt WOOD}, {\tt MARBLE}, or {\tt BOZO}, etc.
keywords are optional.  If a different colouring is specified
({\tt WOOD}, {\tt MARBLE}, or {\tt BOZO}), then the
{\tt COLOUR}\keyindex{COLOUR}%
\index{textures!perturbation!{\tt COLOUR} ignored in}
parameter is ignored (except for light sources where it gives the
light colour or when rendering with a low {\tt -q}\optindex{q} option).

\subsubsection{{\tt WAVES}}

\keyindex{WAVES}\index{textures!perturbation!wavy}
Another option that you may want to experiment with is called {\tt
WAVES}.  This works in a similar way to {\tt RIPPLES} except that it
makes waves with different frequencies.  The effect is to make waves
that look more like deep ocean waves. (I haven't done much testing on
{\tt WAVES}, so I can't guarantee that it works very well).

\subsubsection{{\tt PHASE}}

\keyindex{PHASE}\index{textures!perturbation!phase}
Both {\tt WAVES} and {\tt RIPPLES} respond to a texturing option
called {\tt PHASE}.  The {\tt PHASE} option allows you to create
animations in which the water seems to move.  This is done by making
the {\tt PHASE} increment slowly between frames.  The range from 0.0
to 1.0 gives one complete cycle of a wave.

\subsubsection{{\tt FREQUENCY}}

\keyindex{FREQUENCY}\index{textures!perturbation!frequency}
The {\tt WAVES} and {\tt RIPPLES} textures also respond to a keyword
called {\tt FREQUENCY}.  If you increase the frequency of the waves,
they get closer together.  If you decrease it, they get farther apart.


\subsubsection{{\tt BUMPS}}

\keyindex{BUMPS}\index{textures!perturbation!bumpy}
Approximately the same turbulence function as
{\tt SPOTTED}\keyindex{SPOTTED}, but uses
the derived value to perturb the surface normal.  This gives the
impression of a ``bumpy'' surface, random and irregular, sort of like
an orange.  After the {\tt BUMPS} keyword, supply a single floating
point value for the amount of surface perturbation.  Values typically
range from 0.0 (No Bumps) to 1.0 (Extremely Bumpy).  Values beyond 1.0
may do weird things.

\subsubsection{{\tt DENTS}}

\keyindex{DENTS}\index{textures!perturbation!dented}
Interesting when used with
{\tt METALLIC}\keyindex{METALLIC} textures, it gives impressions into
the metal surface that
look like dents.  A single value is supplied after the {\tt DENTS}
keyword to indicate the amount of denting required.  Values range from
0.0 (Showroom New) to 1.0 (Insurance Wreck).  Use larger values at
your own risk.  Scale the texture to make the pitting more or less
frequent.

\subsubsection{{\tt WRINKLES}}

\keyindex{WRINKLES}\index{textures!perturbation!wrinkled}
This is sort of a 3-D (normal perturbing) {\tt GRANITE}\keyindex{GRANITE}.
It uses a
similar $1/f$ fractal noise function to perturb the surface normal in
3-D space.  With {\tt ALPHA}\keyindex{ALPHA} values of greater than
0.0, could look
like wrinkled cellophane.  Requires a single value after the
{\tt WRINKLES} keyword to indicate the amount of wrinkling desired.  Values
from 0.0 (No Wrinkles) to 1.0 (Very Wrinkled) are typical.

\subsection{{\tt IMAGEMAP}}

\keyindex{IMAGEMAP}\index{image mapping}
\index{textures!image mapped|see{image mapping}}
This is a very special coloration texture that allows you to import a
bitmapped file in ``dump'' format (the format output by the ray-tracer),
IFF\index{IFF} format or Compuserve GIF\index{GIF} format and map that
bitmap onto an object.  In the texture of an object, you must give the
{\tt IMAGEMAP} keyword, the format, and a file name.  The syntax is:
\begin{verse}
{\tt IMAGEMAP {\em optional-gradient} DUMP "{\em filename\/}" {\em optional-}ONCE} \\
{\tt IMAGEMAP {\em optional-gradient} IFF "{\em filename\/}" {\em optional-}ONCE} \\
{\tt IMAGEMAP {\em optional-gradient} GIF "{\em filename\/}" {\em optional-}ONCE}
\end{verse}
The texture will then be mapped onto the object as a repeating
pattern.  The optional
{\tt ONCE}\keyindex{ONCE}\index{image mapping!and {\tt ONCE}} keyword
places only one image onto the object
instead of an infinitely repeating tiled pattern.  When {\tt ONCE} is
used, the colour outside the mapped texture is set to transparent.
You can use the layered textures to place other textures or colours
below the image.

In version 2.10 and up, you can specify the
{\tt ALPHA}\keyindex{ALPHA}\index{image mapping!and {\tt ALPHA}}
values for the
colour registers of IFF\index{IFF} or GIF\index{GIF} pictures (at
least for the modes that
use colourmaps).  You can do this by putting the keyword {\tt ALPHA}
immediately following the filename followed by the register value and
transparency.  If the
{\tt ALL}\keyindex{ALL}\index{image mapping!and {\tt ALL}} keyword
is used instead of a register number, then all colours in that
colourmap get that alpha value. For example:
\begin{verbatim}
     IMAGEMAP <1.0 -1.0 0.0> IFF "mypic.iff"
         ALPHA ALL 0.0
         ONCE
\end{verbatim}
or
\begin{verbatim}
     IMAGEMAP <1.0 -1.0 0.0> IFF "mypic.iff"
         ALPHA 0   0.5
         ALPHA 1   1.0
         ALPHA 2   0.3
         ONCE
\end{verbatim}
By default, the image is mapped onto the X-Y plane in the range (0.0,
0.0) to (1.0, 1.0).  If you would like to change this default, you may
use an optional gradient\index{image mapping!gradient}%
\index{gradient|see{image mapping}} {\tt <x y z>} vector after the word
{\tt IMAGEMAP}.  This vector indicates which axes are to be used as the $u$
and $v$ (local surface X-Y) axes.  The vector should contain one
positive number and one negative number to indicate the $u$ and $v$
axes, respectively.  You\index{image mapping!transformation} may
translate, rotate, and scale the texture
to map it onto the object's surface as desired.  Here is an example:
\begin{verbatim}
     INCLUDE "shapes.data"
     OBJECT
          QUADRIC Plane_XY END_QUADRIC
          TRANSLATE <0.0  -20.0  0.0>
          TEXTURE
               { make this texture use the x and z axes
                 for the mapping.                       }
               IMAGEMAP <1.0  0.0  -1.0> GIF "image.gif"
               SCALE <40.0 40.0 40.0>
          END_TEXTURE
     END_OBJECT
\end{verbatim}

Filenames\index{image mapping!filenames}
specified in the {\tt IMAGEMAP} statements will be searched
for in the home (current) directory first, and if not found, will then
be searched for in directories specified by any {\tt -l}\optindex{l}
(library path) options active.  This would facilitate keeping all your
imagemaps ({\tt .dis}, {\tt .gif} or {\tt .iff}) files in a
``textures'' subdirectory, and giving an {\tt -l} option on the
command line to where your library of imagemaps are.

When I was bored with nothing to do, I decided that it would be neat
to have turbulent texture maps.  So, I said -- ``What the hell!''  You
can specify
{\tt TURBULENCE}\keyindex{TURBULENCE}%
\index{image mapping!and {\tt TURBULENCE}}
with texture maps and it will perturb the
image.  It may give some bizarre results.  Is this useful?  I dunno.
It was easy to do so I did it.  Try it out and see what you get.

\section{Composite Objects}

\index{objects!composite}\keyindex{COMPOSITE}
Often it's useful to combine several objects together to act as a whole.  A
car, for example, consists of wheels, doors, a roof, etc.  A composite object
allows you to combine all of these pieces into one object.  This has two
advantages.  It makes it easier to move the object as a whole and it allows
you to speed up the ray tracing by defining bounding shapes that contain the
objects.  (Rays are first tested to see if they intersect the bounding shape.
If not, the entire composite object is ignored).  Composite objects are
defined as follows:
\begin{verbatim}
     COMPOSITE
          OBJECT
               ...
          END_OBJECT
          OBJECT
               ...
          END_OBJECT
          ...
          SCALE < 2.0 2.0 2.0 >
          ROTATE < 30.0 45.0 160.0 >
          TRANSLATE < 100.0 20.0 40.0 >
     END_COMPOSITE
\end{verbatim}
Composite objects can contain other composite objects as well as regular
objects. Composite\index{objects!composite!as light sources}%
\index{light sources!composite objects}
objects cannot be light sources (although any number of
their components can).  This is because it is nearly impossible to determine
the true ``center'' of the composite object, and our light sources are pinpoint
ray sources from the center of the light source object, wherever that may be.

\section{Bounding Shapes}

\index{bounding shapes}
Let's face it.  This program is no speed demon.  You can save yourself a lot
of time, however, if you use bounding shapes around any complex objects.
Bounding shapes tell the ray tracer that the object is totally enclosed by a
simple shape.  When tracing rays, the ray is first tested against the simple
bounding shape.  If it strikes the bounding shape, then the ray is further
tested against the more complicated object inside.

\begin{description}
\item[Note:] Don't use bounding shapes instead of
CSG\index{constructive solid geometry!and bounding shapes} to clip objects.
You will not get the result you want.  For the raytracer to work
properly, you must have the entire object inside the bounding shape.
\end{description}

To use bounding shapes, you simply include the following lines into the
declaration of your {\tt OBJECT} or {\tt COMPOSITE}:
\begin{verse}
{\tt BOUNDED_BY} \\
{\tt \ \ \ \ \ {\em shape}} \\
{\tt END_BOUND}
\end{verse}
An example of a Bounding Shape is:
\begin{verbatim}
     OBJECT
          INTERSECTION
               SPHERE <0.0 0.0 0.0> 2.0 END_SPHERE
               PLANE <0.0 1.0 0.0> 0.0 END_PLANE
               PLANE <1.0 0.0 0.0> 0.0 END_PLANE
          END_INTERSECTION
          BOUNDED_BY
               SPHERE <0.0 0.0 0.0> 2.0 END_SPHERE
          END_BOUND
     END_OBJECT
\end{verbatim}
The best bounding shape is a {\tt SPHERE}\keyindex{SPHERE} since
this shape is highly optimized.\index{processing speed}
Any shape may be used, however, if more convenient.

\chapter{Displaying the Images}

\index{displaying!after tracing}\index{post-processing}
When the ray tracer draws the picture on the screen, it doesn't make good
choices for the colour registers.  Without knowing all the needed colours
ahead of time, only approximate guesses can be made.  Usually, a post-
processor is really needed to view the final image correctly.

\section{Amiga Systems}

A post-processor has been provided for the Amiga which scans the picture and
chooses the best possible colour registers.  It then redisplays the picture.
For the Amiga, {\tt DumpToIFF}\ttindex{DumpToIFF}\index{IFF format}
can optionally save this picture in IFF\index{IFF} format. The syntax for the
{\tt DumpToIFF} post-processor is:
\begin{verbatim}
     DumpToIFF filename
\end{verbatim}
where the filename is the one given in the {\tt -o}\optindex{o}
parameter of the ray tracer.  If you didn't specify the {\tt -o}
option, then use:
\begin{verbatim}
     DumpToIFF data.dis
\end{verbatim}
If you want to save to an IFF file, then put the name of the IFF file after
the name of the data file:
\begin{verbatim}
     DumpToIFF data.dis picture
\end{verbatim}
This will create a file called {\tt picture} which contains the IFF image.

An alternative approach is to buy the commercial package called
{\em The Art Department}\emindex{The Art Department} from ASDG.
You can then use the {\tt +fr}\optindex{f} option
of the raytracer to produce raw files which can be read in to {\em TAD}
using Sculpt mode.  You can also render using {\tt +fd} to produce a
dump format file, and use {\tt d2iff}\ttindex{d2iff} to convert this
to a 24-bit IFF image to load into {\em TAD}.

\section{IBM Systems}

For the IBM, you will probably want to use the {\tt +ft}\optindex{f}
option (default if {\tt +f} is given) and write the image out in
Targa-24\index{Targa}
format.  If you have a Targa or compatible display adapter, you may
view the picture in the full 16 million colors (that's why they still
cost the big bucks, but Hercules and Everex, notably, are introducing
their lower-priced SVGA-compatible 24-bit color display systems for
the IBM PC and compatibles).  If you don't have one of these, there
are several different post-processing programs available to convert
the TARGA true-color image into a more suitable color-mapped image
(such as {\tt .GIF}).  If you have a VGA/MCGA or SVGA
adapter capable
of 320x200 by 256 colors or better, then you may use the +d option
which will display the image as it generates using only approximate
screen colors.  The {\tt +d} option will Autodetect the type of
display adapter card you have and briefly say what kind it found
before displaying the picture.  If you say {\tt +d{\em x}} where
{\em x} is one of the predefined IBM (S)VGA display adapter types, no
hardware test is performed, so if you don't have that type of (S)VGA
card, {\em don't} use that particular {\tt +d{\em x}} option!

When displaying the image to screen, a HSV conversion method is used (hue,
saturation, value).  This is a convenient way of translating colors from a
``true color'' format (16 million) down a ``colour mapped'' format of something
reasonable (like 256), while still approximating the color as closely as the
available display hardware permits.  As mentioned previously, the tracer has
no way of knowing which colors will be finally used in the image, nor can it
deal properly with all of the colors which will be generated (up to 16M), so
only 4 shades each of 64 possible hues are mapped into the palette DAC, as
well as black, white, and two grey levels. The advantage a post-processor has
in choosing mapped colors is that it can throw away all the unused colors in
the palette map, and thereby free up some space for making better gradient
shades of the colors that are actually used.

There are several available image processing programs that can do
this, a public domain one that is very good is
{\tt PICLAB}\ttindex{PICLAB}, by the
Stone Soup Group (the folks who brought you {\tt FRACTINT}).  The
procedure is to load the TARGA file, then use the {\tt MAKEPAL}
command to generate a 256 color map which is the histogram-weighted
average of the most-used colors in the image (You could also
{\tt PLOAD} a palette file from {\tt FRACTINT} or one you previously had
saved using {\tt PSAVE}).  You then {\tt MAP} the palette onto the
image one of two ways:
\begin{enumerate}
\item If the {\tt DITHER} variable is {\tt OFF}, a nearest-match
color-fit is used, which can sometimes produce unwanted ``banding'' of
colored regions (called false contours).
\item If the {\tt DITHER} variable is {\tt ON}, then a standard dither
is used to determine final color values.  This is much better at
blending the color bands, but can produce noise in reflections and
make mirrors appear dirty or imperfect.
\end{enumerate}

Then you would typically {\tt SHOW} the image or {\tt GSAVE} it into
GIF format.  While the picture is still in the unmapped form (TARGA,
etc.) you can perform a variety of advanced image processing
transformations and conversions, so save the {\tt .TGA} or {\tt .RAW}
files you make (in case you ever get a TARGA card, or give them to a
friend who has one!).

A commercial product that also does a good job of nearest-match
color-fit is the {\tt CONVERT} utility of
{\em AutoDesk Animator}\emindex{Autodesk Animator}.
However, the dither effect is not as good as that of {\tt PICLAB}.  To
convert the file in {\em AA}'s {\tt CONVERT}, you {\tt LOAD} TARGA, then in
the {\tt CONVERT} menu, go to the {\tt SCALE} function and just hit
{\tt RENDER}.  Click on the {\tt DITHER} (lights up with an asterisk
when on) if you want it to use it's dithering.  {\tt CONVERT} will
scale (if asked to) and then do a histogram of total used colors like
{\tt PICLAB}, but then makes 7 passes on the color map and image to
determine shading thresholds of each hue.  This nearly eliminates the
color banding (false contours) in a lot of cases without resorting to
good 'ol dithering.  By now you must get the feeling {\tt DITHER} is a
4-letter word.  If you have a low-resolution display, it is.  If you
have too few colors, however, it can be a saving grace.  At
resolutions of 640x400 or higher the ``spray paint'' effect of
dithering and anti-aliasing is much less noticeable, and effects a
much smoother blending appearance.

A new package to show up in the public domain/shareware circles for the IBM
is something called {\em Image Alchemy}\emindex{Image Alchemy}, by
Handmade Software.  It will convert
Targa\index{Targa} format to GIF\index{GIF} files and do a decent
job of palette selection and dithering.  To use it simply say
\begin{verbatim}
     ALCHEMY file.tga file.gif -g -8 -c256
\end{verbatim}
It also features a quick-and-dirty display mode where it uses a
standardized palette in much the same way DKB's {\tt +d} option does,
but offers dithering of the image while using the pre-defined palette,
for a somewhat better quick display.

\section{Unix Systems}

I don't have many details on Unix systems, but I hear that the
FBM\index{FBM utilities} utilities
work well to convert the Dump format files into various formats of images.
For people with access to anonymous FTP over USENET, the FBM utilities are
available from {\tt nl.cs.emu.edu} (128.2.222.56) in directory
{\tt /usr/mlm/ftp}.

\chapter{DKBTrace Utilities}

\index{utilities}
In many cases, creating data files for DKBTrace is difficult and tedious.  To
help remedy this problem, I and various other people have developed some
utilities to create data files.  These utilities are described below.

As well, there are some utilities that perform operations on the image files
created by DKBTrace.  These utilities convert between various formats and
allow you to modify or merge output files together.

I'd like to thank all the people who wrote these utilities and sent them to
me.  If anybody else comes up with other utilities, please let me know and
I'll include them in the distribution.

Some of these utilities are written in BASIC in IBM systems.  As such, they
are not easily portable from system to system.  If anyone wants to convert
them to C, let me know and I'll post the C versions.

\section{Data File Creation Utilities}

The data creation utilities fall into two categories:  Those that convert from
some other format into DKB format, and those that generate DKB files using
algorithmic techniques.  These utilities are described below.

\subsection{{\tt SA2DKB}}

\ttindex{SA2DKB}
This program converts Sculpt-Animate 3D and 4D data files into DKB
format.  It currently only supports the basic triangles and textures.
It doesn't support smooth triangles (it treats them like normal
triangles), light sources, cameras, or floors.  (This utility was
formerly called {\tt Sculpt2DKB} but the IBM systems out there kept
calling it {\tt SCULPT2D}, then couldn't figure out what a 2D program
had to do with raytracing or what the nonexistent Amiga program called
{\tt Sculpt-2D} was.

\subsection{{\tt DXF2DKB}}

\ttindex{DXF2DKB}\index{Collins, Aaron}
This utility converts AutoCAD DXF (Drawing eXchange Format) files into
DKBTrace format scene description files.  It was written by Aaron
Collins.  It does not support all of the DXF primitives, but will
suffice for simple objects and scenes after {\tt EXPLODE}'ing and
{\tt DXFOUT}'ing then in {\em AutoCAD}\emindex{AutoCad}.

\subsection{{\tt ShellGen}}

\ttindex{ShellGen}\index{Farmer, Dan}\index{Pickover, Clifford}
{\tt ShellGen} is a BASIC program written by Dan Farmer.  It's based
on a short code fragment from Clifford Pickover's book
{\em Computers, Pattern, Chaos, and Beauty} (St. Martin's Press).
This code fragment was reprinted in {\em Ray Tracing News}, issue 3.3.

As far as I know, the BASIC program only works on IBM's.  It does,
however, allow you to change the parameters and see a quick outline of
what the result will look like.  For those people without IBM's, I've
changed the original code fragment to at least output a DKB-format
file.  No user interface has been provided, however.

\subsection{{\tt Twister}}

\ttindex{Twister}\index{Wells, Drew}
{\tt Twister} is a C program written by Drew Wells (CIS {\tt 73767,1244}).
It creates data files for twisted shapes.  The program
uses a text interface and prompts the user with a question/answer
format.

\subsection{{\tt Chem2DKB}}

\ttindex{Chem2DKB}\index{Farmer, Dan}\index{Puhl, Larry}
{\tt Chem2DKB} is an IBM BASIC program written by Dan Farmer.  It takes models
generated by the {\tt CHEM.EXE} program written by Larry Puhl.

\subsection{{\tt Lissajou}}

\ttindex{Lissajou}\index{Farmer, Dan}\index{Pickover, Clifford}
This is an IBM BASIC program written by Dan Farmer.  It creates data
files for lissajous figures.  The basic algorithms were from Clifford
Pickover.  See {\em Scientific American}, January 1991 and {\em Omni},
February 1990 for examples.

\section{Output File Manipulation Utilities}

These utilities perform some useful manipulations on the dump format
and Targa\index{Targa}
format output files from DKBTrace.  I'd like to thank the people who wrote
these utilities and provided them for general distribution.

\subsection{{\tt dump2i24}}

\ttindex{dump2i24}\index{Rasmussen, Helge E.}
Also known as {\tt DumpToIFF24}, this program was written by Helge
E.\ Rasmussen ({\tt her@compel.dk}).  It converts the dump format files
produced by DKBTrace into 24-bit IFF\index{IFF} format files.  These files can
then be read by a variety of programs including
{\em The Art Department} by ASDG.

\subsection{{\tt catdump}}

\ttindex{catdump}\index{Saari, Ville}
This utility was written by Ville Saari ({\tt vsaari@niksula.hut.fi},
and copyright by the Ferry Island Pixelboys.)  It takes two or more
partially rendered files in DKBTrace's dump format and merges them
into one file.  This is useful for all sorts of things like rendering
different parts on different computers and combining the results.

\begin{description}
\item[NOTE:] Be careful if you combine pictures produced on
different systems.  If the random number generator works differently
between the two systems, the textures may look completely different
from one another.  So long as you use the same executable, you should
be fine.
\end{description}

\subsection{{\tt combdump}}

\ttindex{compdump}\index{Saari, Ville}
This utility was also written by Ville Saari.  It takes two images generated
with DKBTrace with slightly different viewpoints, and creates one dump-format
image file to be viewed with Red-Blue or Red-Green 3D glasses.  The program
allows you to compensate for the exact filtering characteristics of your
glasses to get the best possible result.

\subsection{{\tt dump2mtv}}

\ttindex{dump2mtv}\index{Saari, Ville}
This is yet another utility written by Ville Saari.  This one converts
DKBTrace dump format files onto MTV format used by the MTV and RayShade
raytracers.

\subsection{{\tt dump2raw}}

\ttindex{dump2raw}\index{Collins, Aaron}
The {\tt dump2raw} utility was written by Aaron Collins to convert the
dump format output of DKBTrace into three separate files for red,
green, and blue.  On the IBM, the extensions for these files are
{\tt r8}, {\tt g8}, and {\tt b8}.  On the other systems, they are
{\tt red}, {\tt grn} and {\tt blu}.

Version 2.10 of the raytrace allows you to use the {\tt +fr}\optindex{f}
option to output raw format files directly without the need for
a conversion program like this.

\subsection{{\tt halftga}}

\ttindex{halftga}\index{Collins, Aaron}
The {\tt halftga} utility (written by Aaron Collins) shrinks a
Targa\index{Targa} file to exactly half its original size.  This file can
then be converted into a GIF\index{GIF} image and used in an
{\tt IMAGE_MAP}\keyindex{IMAGE_MAP}
statement.  For systems with little memory available for imagemaps,
this command can be a life-saver.

\subsection{{\tt gluetga}}

\ttindex{gluetga}\index{Collins, Aaron}
This utility (by Aaron Collins) is similar to {\tt catdump} but works
for Targa\index{Targa} format files.  It takes several
partially-rendered Targa\index{Targa} files and glues them together
into one image. 

\subsection{{\tt tga2dump}}

\ttindex{tga2dump}\index{Collins, Aaron}
This utility was written by Aaron Collins.  It converts Targa\index{Targa}
format 16, 24, and 32 bit images into DKB's dump format for use in
image-mapping.

\section{Animation Utilities}

One of the most frequent questions I'm asked is whether or not DKBTrace
supports animation.  The answer is no, not directly.  However, I have made
some changes to the program to provide frame-to-frame consistency so you can
use it for animation if you want to.  The problem, then, is creating the data
files for each individual frame.  That's what this section is all about.

\subsection{RayScene}

\index{RayScene}\index{Kuhkunen@K\^{u}hk\"{u}nen, Jari}\index{Hassi, Paul}
Although RayScene is not being distributed with this raytracer, I thought I'd
at least mention it and tell you where you can get it.  RayScene is a program
that creates data files for DKBTrace based on a high-level (higher-level?)
description of the motion of the camera and the objects.  It was written by
Jari K\^{u}hk\"{u}nen ({\tt hole@tolsun.oulu.fi}) and
Panu Hassi ({\tt oldfox@tolsun.oulu.fi}) and
is available by anonymous FTP from {\tt tolsun.oulu.fi} (128.214.5.6) in the
directory {\tt /pub/rayscene}
or from {\tt iear.arts.rpi.edu} in the directory
{\tt /pub/graphics/ray/rayscene}.
This explanation of RayScene was sent to me by Panu Hassi:
\begin{quotation}
I've tried animation with DBW before DKBTrace2.0 was released.  
The procedure was this: First I wrote the first scene file, copied it 
for {\tt NUMBER_OF_FRAMES} times and then edited some parts of those files 
to create movement etc.  If something went wrong (I accidentally edited 
wrong value etc), I had to edit all those scene files again to make the
changes.  Not so nice if there are 100 files to edit\ldots

So a friend of mine, Jari K\^{u}hk\"{u}nen, and I decided to write RayScene to
make that process even a little easier.  With RayScene the process
goes like this:  you create a scene file and mark the places that 
should be changed with a variable, like:
\begin{verbatim}
    BOUNDED_BY                              
         SPHERE <0.0 0.0 0.0> #sphere_size# END_SPHERE
    END_BOUND
\end{verbatim}

Then you create another file where the values for these variables are
listed.  Rayscene simply creates N scene files inserting current value of
each variable to proper place.  That's all.
    
We have included couple of simple utilities that help with creating
those variable values, but the original scene files are still created
``manually''.  Still, the results have been really nice.  There are 
several animations for Amiga and PC in {\tt tolsun.oulu.fi}.
\end{quotation}

\chapter{How it All Works (or How to Get What You Want)}

The information in this section is designed for people who are reasonably
familiar with the raytracer and want more information on how things work so
they can push it to its limits.  You probably don't need this level of detail
to make interesting data files, but if you suddenly get confused about
something the program did, this section may help you figure it out.

\section{Viewpoints}

\index{viewpoint}
Viewpoints can be completely defined by four vectors.  The
{\tt LOCATION}\keyindex{LOCATION} is easy.  That's where the camera
is. The {\tt DIRECTION}\keyindex{DIRECTION}
is a vector that starts at the {\tt LOCATION} and points to the center
of a window.  The {\tt UP}\keyindex{UP} vector starts at the center
of the window and points to the center of the top edge.  The
{\tt RIGHT}\keyindex{RIGHT} vector
starts at the center of the window and points to the center of the
right edge. These vectors are illustrated in Figure \ref{vectors}.
 
\begin{figure}[htbp]
\begin{centering}
\setlength{\unitlength}{0.0125in}%
\begin{picture}(181,155)(54,605)
\thinlines
\put(100,680){\circle*{10}}
\put(205,680){\vector( 1,-1){ 20}}
\put(205,680){\vector( 0, 1){ 35}}
\put(100,680){\vector( 1, 0){105}}
\put(180,760){\line( 0,-1){100}}
\put(180,660){\line( 1,-1){ 55}}
\put(235,605){\line( 0, 1){100}}
\put(235,705){\line(-1, 1){ 55}}
\put( 90,675){\makebox(0,0)[rb]{\raisebox{0pt}[0pt][0pt]{\elvrm Location}}}
\put(120,665){\makebox(0,0)[lb]{\raisebox{0pt}[0pt][0pt]{\elvrm Direction}}}
\put(205,650){\makebox(0,0)[lb]{\raisebox{0pt}[0pt][0pt]{\elvrm Right}}}
\put(210,695){\makebox(0,0)[lb]{\raisebox{0pt}[0pt][0pt]{\elvrm Up}}}
\end{picture}

\caption{Viewpoint definition vectors}
\label{vectors}
\end{centering}
\end{figure}

The window is then divided up according to the resolution you specified and
rays are fired through the pixels out into the world.  For an eye ray,
therefore, the equation of the ray is:
\begin{displaymath}
{\tt LOCATION} + t ({\tt DIRECTION} + ((height - y) \cdot {\tt UP}) +
(x \cdot {\tt RIGHT}))
\end{displaymath}
where $t$ is a parameter that determines the distance from the eye to the
object being tested.  The\index{coordinate system!inverted Y} Y
coordinate is inverted by subtracting it from height because most
graphics systems put $(0,0)$ in the top left corner of the screen.

This viewpoint model is very flexible.  It allows you to use
left-handed or right-handed coordinate systems.  It also doesn't
require that the {\tt DIRECTION}, {\tt UP}, and {\tt RIGHT} vectors be
mutually orthogonal.  If you want, you can distort the camera to get
really bizarre results.

Once the basic four vectors are specified, it's possible to use the
{\tt SKY}\keyindex{SKY} and {\tt LOOK_AT}\keyindex{LOOK_AT} vectors
to point the camera.  You must
specify the {\tt SKY} vector first, but let me describe the
{\tt LOOK_AT} vector first.  {\tt LOOK_AT} tells the camera to rotate in
such a way that the {\tt LOOK_AT} point appears in the center of the
screen.  To do this, the camera first turns in the left-to-right
direction (longitude in Earth coordinates) until it's lined up with
the {\tt LOOK_AT} point.  It then turns in the up/down direction (latitude
in Earth coordinates) until it's looking at the desired point.

Ok, now we're looking at the proper point.  What else do we have to
specify?  If you're looking at a point, you can still turn your camera
sideways and still be looking at the same spot.  This it the
orientation that the {\tt SKY} direction determines.  The camera will
try to position itself so that the camera's {\tt UP} direction lines
up as closely as possible to the {\tt SKY} direction.

Put another way -- in airplane terms, the {\tt LOOK_AT} vector
determines your heading (north, south, east, or west), and your pitch
(climbing or descending).  The {\tt SKY} vector determines your
banking angle.

\section{Ray-Object Intersections}

For every pixel on the screen, the raytracer fires at least one ray through
that pixel into the world to see what it hits.  For each hit (well, almost),
it calculates rays to each of the light sources to see if that point is
shadowed from that light source.  For reflecting objects, a reflected ray is
traced.  For refracting objects, a refracting ray is traced.  That all adds up
to a lot of rays.

Every ray is tested against every object in the world to see if the ray hits
that object. This is what slows down the raytracer.  You can make things
easier by using simple bounding shapes on your
objects\index{processing speed}.

Fortunately, all ray-object intersections for all shapes in DKBTrace can be
solved by a simple quadratic equation.  This is why {\tt QUADRIC}s are used in
DKBTrace.  Solving for things like B-Splines, Bezier Splines, NURBS, Tori,
etc. is a lot more complicated.  That's why I haven't implemented primitives
for these shapes.

\section{Transparency and Refraction}

This section gets really complicated because of the way transparency and
refraction are implemented.  If you don't really care, skip to the next
section. If you don't mind slogging through this and getting confused, then
read on -- you've been warned.

The way that transparency and refraction work has changed slightly from
previous versions.  Now, transparency and refraction work together instead of
separately.

First\index{light!reflected vs. filtered}, let me distinguish between
reflected light and filtered light.  Suppose
you painted a table with patches of colour.  You then took some red sand and
sprinkled it on top of the various colours.  The red sand will tint the
colours red, but you will still see some of the original colour.  If, instead,
you took a sheet of red plexiglass and put it on top of the table, all you
could see would be shades of red.  That's because the plexiglass filter
{\em only} allows the red colour to show through.

In DKBTrace, the layered textures\index{textures!layered} work like
the red sand -- the colours on top mix with the underlying colours
depending on the density ({\tt ALPHA}\keyindex{ALPHA}) of the
distribution.

Refraction\index{refraction}, however, works like a filter.  The
surface colour determines that
colours of light are allowed to pass from the inside of the object to the
outside, and vice versa.  Here are some filter colours:
\begin{verbatim}
   { a red filter }
   RED 1.0  GREEN 0.0  BLUE 0.0   ALPHA 1.0
   { a clear glass }
   RED 1.0  GREEN 1.0  BLUE 1.0   ALPHA 1.0
   { a dark filter - this will appear black }
   RED 0.0  GREEN 0.0  BLUE 0.0   ALPHA 1.0
\end{verbatim}
Now, consider the following layered textures:\keyindex{REFRACTION}
\begin{verbatim}
   TEXTURE COLOUR Green ALPHA 0.6 END_TEXTURE
   TEXTURE COLOUR Yellow ALPHA 0.3 REFRACTION 0.5 END_TEXTURE
\end{verbatim}
Keep in mind that the last texture is on top.\footnote{In layered
textures, only the {\tt REFRACTION} component of the {\em last} entry
has any effect.}
The colour you get is calculated as shown in Figure \ref{layers}.
\begin{figure}[htbp]
\begin{centering}
\setlength{\unitlength}{0.0125in}%
\begin{picture}(140,154)(60,640)
\thinlines
\put(140,710){\vector( 4, 3){ 40}}
\put(140,680){\vector( 1,-1){ 40}}
\put(180,700){\line( 1, 0){ 20}}
\put(200,740){\line(-1, 0){ 20}}
\put(180,740){\line( 0,-1){100}}
\put(180,640){\line( 1, 0){ 20}}
\put( 60,710){\line( 1, 0){ 80}}
\put( 80,780){\line(-1, 0){ 20}}
\put( 60,780){\line( 0,-1){100}}
\put( 60,680){\line( 1, 0){ 80}}
\put(200,715){\makebox(0,0)[lb]{\raisebox{0pt}[0pt][0pt]{\elvrm 4/10 Green}}}
\put( 80,690){\makebox(0,0)[lb]{\raisebox{0pt}[0pt][0pt]{\elvrm 3/10 (alpha)}}}
\put( 80,740){\makebox(0,0)[lb]{\raisebox{0pt}[0pt][0pt]{\elvrm 7/10 is Yellow}}}
\put( 60,665){\makebox(0,0)[lb]{\raisebox{0pt}[0pt][0pt]{\elvrm Full darkness}}}
\put( 60,785){\makebox(0,0)[lb]{\raisebox{0pt}[0pt][0pt]{\elvrm Full brightness}}}
\put(200,680){\makebox(0,0)[lb]{\raisebox{0pt}[0pt][0pt]{\elvrm 6/10 (alpha) left for refraction}}}
\put(200,666){\makebox(0,0)[lb]{\raisebox{0pt}[0pt][0pt]{\elvrm of which 5/10 refracted filtered}}}
\put(200,652){\makebox(0,0)[lb]{\raisebox{0pt}[0pt][0pt]{\elvrm by yellow filtered by green}}}
\end{picture}

\caption{Layered color calculation}
\label{layers}
\end{centering}
\end{figure}
The top texture layer supplies some fraction of the light that
reflects off the surface of the object.  If the {\tt ALPHA} was
non-zero, it allows the lower textures to supply the remainder.  If,
after all the textures are processed, there's still some fraction left
over, it is applied to the light that's refracted through the object.

This algorithm probably doesn't coincide with reality, but neither
does the rest of the raytracer, so I'm not terribly concerned about
it.

\section{Textures, Noise, and Turbulence}

\index{textures}
If there's one thing that DKBTrace is known for, it's textures.  Here's how
they really work.  If you want some good reading material, check out ``An Image
Synthesizer'' by Ken Perlin in the SIGGRAPH '84 Conference Proceedings.

Let's start with a marble texture\index{textures!marbled}.  Real marble
is created when different colours of sediments are laid down one on
top of another and compressed to form solid rock.

For example, take the simple block of marble shown in Figure \ref{marble}.
\begin{figure}[htbp]
\begin{centering}
\setlength{\unitlength}{0.0125in}%
\begin{picture}(80,100)(160,620)
\thinlines
\put(160,640){\line( 1, 0){ 60}}
\put(220,640){\line( 1, 1){ 20}}
\put(160,660){\line( 1, 0){ 60}}
\put(220,660){\line( 1, 1){ 20}}
\put(160,680){\line( 1, 0){ 60}}
\put(220,680){\line( 1, 1){ 20}}
\put(240,720){\line( 0,-1){ 80}}
\put(240,640){\line(-1,-1){ 20}}
\put(160,700){\line( 0,-1){ 80}}
\put(160,620){\line( 1, 0){ 60}}
\put(220,620){\line( 0, 1){ 80}}
\put(160,700){\line( 1, 0){ 60}}
\put(220,700){\line( 1, 1){ 20}}
\put(240,720){\line(-1, 0){ 60}}
\put(180,720){\line(-1,-1){ 20}}
\put(190,625){\makebox(0,0)[b]{\raisebox{0pt}[0pt][0pt]{\elvrm White}}}
\put(190,645){\makebox(0,0)[b]{\raisebox{0pt}[0pt][0pt]{\elvrm Red}}}
\put(190,665){\makebox(0,0)[b]{\raisebox{0pt}[0pt][0pt]{\elvrm White}}}
\put(190,685){\makebox(0,0)[b]{\raisebox{0pt}[0pt][0pt]{\elvrm Red}}}
\end{picture}

\caption{Marble texture}
\label{marble}
\end{centering}
\end{figure}
If you carve a shape put of this block of marble, you will get red and white
bands across it.

Now, consider wood\index{textures!wood}.  The rings in wood are
created when the tree grows a new outer shell every year.  Hence, we
have concentric cylinders of colours, as shown in Figure \ref{log}.
\begin{figure}[htbp]
\begin{centering}
\setlength{\unitlength}{0.0125in}%
\begin{picture}(85,85)(200,720)
\thinlines
\put(220,740){\circle{10}}
\put(220,740){\circle{20}}
\put(220,740){\circle{30}}
\put(220,740){\circle{40}}
\put(235,725){\line( 1, 1){ 50}}
\put(205,755){\line( 6, 5){ 60}}
\end{picture}

\caption{Wood texture}
\label{log}
\end{centering}
\end{figure}
Cutting a shape out of a piece of wood will tend to give you rings of colour.

Now, this is fine, but the textures are still a bit boring.  For the next
step, we blend the colours together to create a nice smooth transition.  This
makes the texture look a bit better.  The problem, though, is that it's too
regular -- real marble and wood aren't so perfect.

Before we make our wood and marble look any better, let's look at how we make
noise.  Noise\index{noise} (in raytracing) is sort of like a random
number generator, but
it has the following properties:
\begin{enumerate}
\item It's defined over 3D space i.e., it takes x, y, and z and returns the
noise value there.
\item If two points are far apart, the noise values at those points are
relatively random.
\item If two points are close together, the noise values at those points are
close to each other.
\end{enumerate}
You can visualize this as having a large room and a thermometer that ranges
from 0.0 to 1.0.  Each point in the room has a temperature.  Points that are
far apart have relatively random temperatures.  Points that are close together
have close temperatures. The temperature changes smoothly, but randomly as we
move through the room.

Now, let's place an object into this room along with an artist.  The artist
measures the temperature at each point on the object and paints that point a
different colour depending on the temperature.  What do we get?
{\tt BOZO}\keyindex{BOZO}\index{textures!noisy} texture!

Another function used in texturing is called DNoise.  This is sort of like
noise except that instead of giving a temperature, it gives a direction.  You
can think of it as the direction that the wind is blowing at that spot.

Finally, we have a function called turbulence\index{turbulence} which
uses DNoise to push a particle around a few times -- each time going
half as far as before. This procedure is roughly illustrated in
Figure \ref{turb}.
\begin{figure}[htbp]
\begin{centering}
\setlength{\unitlength}{0.0125in}%
\begin{picture}(154,117)(91,655)
\thinlines
\put(200,700){\vector( 1,-1){ 40}}
\put(220,700){\vector(-1, 0){ 20}}
\put(240,760){\vector(-1,-3){ 20}}
\put(160,760){\vector( 1, 0){ 80}}
\put(240,720){\makebox(0,0)[lb]{\raisebox{0pt}[0pt][0pt]{\elvrm Second move}}}
\put(170,765){\makebox(0,0)[lb]{\raisebox{0pt}[0pt][0pt]{\elvrm First move}}}
\put(245,655){\makebox(0,0)[lb]{\raisebox{0pt}[0pt][0pt]{\elvrm Final position}}}
\put(155,755){\makebox(0,0)[rb]{\raisebox{0pt}[0pt][0pt]{\elvrm Initial Position}}}
\end{picture}

\caption{Effect of turbulence}
\label{turb}
\end{centering}
\end{figure}
This is what we use to create the ``interesting'' marble and wood texture.  We
locate the point we want to colour (P), then push it around a bit using
Turbulence to get to a final point (Q) then look up the colour of point Q in
our ordinary boring wood and marble textures.  That's the colour that's used
for the point P.

\section{Layered Textures}

\index{textures!layered}
As of version 2.10, DKBTrace supports layered textures.  Here's how
that works.  Each object and each shape has a texture that may be
attached to it.  By default, shapes have no texture, but objects have
a default texture.  Internally, textures are marked as being constant
or variable.  A constant texture is one that was
{\tt DECLARE}'d\keyindex{DECLARE} as a
texture and is being shared by many shapes and objects.  Variable
textures are textures that have been declared totally within the
object or have used a {\tt DECLARE}'d texture and modified it in a
destructive way.  The idea here is that we want to save on memory by
sharing textures if possible.

If you have several texture blocks for an object or a shape, they are placed
into a linked list (First-in, Last-out)\index{textures!parsing of}.
For example, take the following
definition:
\begin{verbatim}
   OBJECT
      SPHERE <0 0 0> 1 END_SPHERE
      TEXTURE WOOD END_TEXTURE
      TEXTURE MARBLE END_TEXTURE
   END_OBJECT
\end{verbatim}
Here's what happens while parsing this object: Since this is an object
(as opposed to a shape -- {\tt SPHERE}, {\tt PLANE}, etc.), it starts
out with the default texture attached, as shown in Figure \ref{parse1}.
\begin{figure}[htbp]
\begin{centering}
\input{parse1}
\caption{Object parsing, step 1}
\label{parse1}
\end{centering}
\end{figure}
When the parser sees the first {\tt TEXTURE} block, it looks to see what it has
linked.  Since the thing that's linked is the default texture (not a copy), it
discards it and puts in the new texture, as shown in Figure \ref{parse2}.
\begin{figure}[htbp]
\begin{centering}
\input{parse2}
\caption{Object parsing, step 2}
\label{parse2}
\end{centering}
\end{figure}
On the next texture, it sees that the texture isn't the default one, so it
adds the second texture into the linked list, as shown in Figure \ref{parse3}.
\begin{figure}[htbp]
\begin{centering}
\setlength{\unitlength}{0.0125in}%
\begin{picture}(260,80)(80,700)
\thinlines
\put(240,750){\vector( 1, 0){ 40}}
\put(280,700){\framebox(60,80){}}
\put(140,750){\vector( 1, 0){ 40}}
\put(180,700){\framebox(60,80){}}
\put( 80,720){\framebox(60,60){}}
\put(310,736){\makebox(0,0)[b]{\raisebox{0pt}[0pt][0pt]{\elvrm Texture}}}
\put(310,750){\makebox(0,0)[b]{\raisebox{0pt}[0pt][0pt]{\elvrm WOOD}}}
\put(210,736){\makebox(0,0)[b]{\raisebox{0pt}[0pt][0pt]{\elvrm Texture}}}
\put(210,750){\makebox(0,0)[b]{\raisebox{0pt}[0pt][0pt]{\elvrm MARBLE}}}
\put(110,745){\makebox(0,0)[b]{\raisebox{0pt}[0pt][0pt]{\elvrm Object}}}
\end{picture}

\caption{Object parsing, step 3}
\label{parse3}
\end{centering}
\end{figure}

Now for a problem.  If you want to specify the
{\tt REFRACTION}\index{textures!and {\tt REFRACTION}} of the
texture, the raytracer must first calculate the surface colour.  It
does this by marching through the texture list and mixing all the
colours.  When it's finished, it checks the {\tt ALPHA}\keyindex{ALPHA}
value of the
surface colour and decides whether it should trace a refracting ray.
Where does it get the {\tt REFRACTION} value and the index of
refraction?  It simply takes the one in the topmost (the last one
defined) texture.  I don't see any reason to have refraction values
for any other textures in the layer as it applies to the whole object.

\section{Image Mapping}

\index{mapping, images|see{image mapping}}\index{image mapping}
One major problem people have when designing data files for DKBTrace is how to
position images onto the desired surfaces.  With version 2.10, this problem
becomes slightly easier since the image can be mapped onto the object in the
object's natural coordinate system.  Thereafter, when the object is
translated, rotated, or scaled, the image map will follow it.

The image mapping that DKBTrace currently supports is called a
``parallel projection''\index{projection|see{parallel projection}}%
\index{parallel projection} mapping.  This technique is simple (that's why
I implemented it), but it's not perfect.  It works like a slide
projector casting the desired image onto the scene.  The difference,
however, is that the image never gets larger as you move further away
from the slide projector.  In fact, there is no real slide projector.
Consider the cross section shown in Figure \ref{project}, which
shows an image with colours A, B, C, D, and E being mapped onto three
objects.
\begin{figure}[htbp]
\begin{centering}
\setlength{\unitlength}{0.0125in}%
\begin{picture}(313,120)(87,660)
\thinlines
\put(300,720){\framebox(80,60){}}
\put(385,730){\vector( 1, 0){ 15}}
\put(385,745){\vector( 1, 0){ 15}}
\put(385,760){\vector( 1, 0){ 15}}
\put(285,700){\vector( 1, 0){115}}
\put(275,715){\vector( 1, 0){125}}
\put(100,700){\vector( 1, 0){ 95}}
\put(100,715){\vector( 1, 0){100}}
\put(270,730){\vector( 1, 0){ 25}}
\put(100,730){\vector( 1, 0){110}}
\put(260,745){\vector( 1, 0){ 35}}
\put(255,760){\vector( 1, 0){ 40}}
\put(180,745){\vector( 1, 0){ 35}}
\put(190,760){\vector( 1, 0){ 35}}
\put(100,745){\vector( 1, 0){ 60}}
\put(100,760){\vector( 1, 0){ 50}}
\put(240,780){\line(-1,-2){ 60}}
\put(180,660){\line( 1, 0){120}}
\put(300,660){\line(-1, 2){ 60}}
\put(140,780){\line( 1, 0){ 60}}
\put(200,780){\line(-3,-4){ 30}}
\put(170,740){\line(-3, 4){ 30}}
\put(375,725){\makebox(0,0)[rb]{\raisebox{0pt}[0pt][0pt]{\elvrm C}}}
\put(375,740){\makebox(0,0)[rb]{\raisebox{0pt}[0pt][0pt]{\elvrm B}}}
\put(375,755){\makebox(0,0)[rb]{\raisebox{0pt}[0pt][0pt]{\elvrm A}}}
\put(340,725){\makebox(0,0)[b]{\raisebox{0pt}[0pt][0pt]{\elvrm C}}}
\put(305,725){\makebox(0,0)[lb]{\raisebox{0pt}[0pt][0pt]{\elvrm C}}}
\put(305,740){\makebox(0,0)[lb]{\raisebox{0pt}[0pt][0pt]{\elvrm B}}}
\put(305,755){\makebox(0,0)[lb]{\raisebox{0pt}[0pt][0pt]{\elvrm A}}}
\put(230,755){\makebox(0,0)[lb]{\raisebox{0pt}[0pt][0pt]{\elvrm A}}}
\put(250,755){\makebox(0,0)[rb]{\raisebox{0pt}[0pt][0pt]{\elvrm A}}}
\put(180,755){\makebox(0,0)[rb]{\raisebox{0pt}[0pt][0pt]{\elvrm A}}}
\put(170,745){\makebox(0,0)[b]{\raisebox{0pt}[0pt][0pt]{\elvrm B}}}
\put(255,740){\makebox(0,0)[rb]{\raisebox{0pt}[0pt][0pt]{\elvrm B}}}
\put(260,725){\makebox(0,0)[rb]{\raisebox{0pt}[0pt][0pt]{\elvrm C}}}
\put(270,710){\makebox(0,0)[rb]{\raisebox{0pt}[0pt][0pt]{\elvrm D}}}
\put(275,695){\makebox(0,0)[rb]{\raisebox{0pt}[0pt][0pt]{\elvrm E}}}
\put(205,695){\makebox(0,0)[lb]{\raisebox{0pt}[0pt][0pt]{\elvrm E}}}
\put(210,710){\makebox(0,0)[lb]{\raisebox{0pt}[0pt][0pt]{\elvrm D}}}
\put(220,725){\makebox(0,0)[lb]{\raisebox{0pt}[0pt][0pt]{\elvrm C}}}
\put(225,740){\makebox(0,0)[lb]{\raisebox{0pt}[0pt][0pt]{\elvrm B}}}
\put(160,755){\makebox(0,0)[lb]{\raisebox{0pt}[0pt][0pt]{\elvrm A}}}
\put( 95,695){\makebox(0,0)[rb]{\raisebox{0pt}[0pt][0pt]{\elvrm E}}}
\put( 95,710){\makebox(0,0)[rb]{\raisebox{0pt}[0pt][0pt]{\elvrm D}}}
\put( 95,725){\makebox(0,0)[rb]{\raisebox{0pt}[0pt][0pt]{\elvrm C}}}
\put( 95,740){\makebox(0,0)[rb]{\raisebox{0pt}[0pt][0pt]{\elvrm B}}}
\put( 95,755){\makebox(0,0)[rb]{\raisebox{0pt}[0pt][0pt]{\elvrm A}}}
\end{picture}

\caption{Parallel projection}
\label{project}
\end{centering}
\end{figure}
The raytracer performs a similar operation
to map a 2D picture onto a 3D object.  Note that objects cannot shadow each
other from the image being mapped.  This means that the image will also appear
on the back of the object as a mirror image.

The mapping takes the original image (regardless of the size) and maps
it onto the range 0,0 to 1,1 in two of the 3D coordinates.  Which two
coordinates is specified by the gradient\index{image mapping!gradient}
vector provided after the
image.  This vector must contain one positive number, one negative
number, and one zero.  The positive number identifies the $u$ axis
(the left-right direction in the image) and the negative number
represents the $v$ axis (the picture's up-down direction).  Note that
the magnitude of the number is irrelevant. For example:
\begin{verbatim}
    IMAGEMAP <1 -1 0> GIF "filename"
\end{verbatim}
will map the GIF\index{GIF} picture onto the square from {\tt <0 0 0>} to
{\tt <1 1 0>} as shown in Figure \ref{map1}.
\begin{figure}[htbp]
\begin{centering}
\setlength{\unitlength}{0.0125in}%
\begin{picture}(110,115)(170,685)
\thinlines
\put(180,700){\vector( 0, 1){100}}
\multiput(260,780)(0.00000,-7.61905){11}{\line( 0,-1){  3.810}}
\multiput(180,780)(7.61905,0.00000){11}{\line( 1, 0){  3.810}}
\put(180,700){\vector( 1, 0){100}}
\put(220,705){\makebox(0,0)[b]{\raisebox{0pt}[0pt][0pt]{\elvrm Bottom}}}
\put(220,770){\makebox(0,0)[b]{\raisebox{0pt}[0pt][0pt]{\elvrm Top}}}
\put(260,685){\makebox(0,0)[b]{\raisebox{0pt}[0pt][0pt]{\elvrm 1}}}
\put(175,775){\makebox(0,0)[rb]{\raisebox{0pt}[0pt][0pt]{\elvrm 1}}}
\put(185,750){\makebox(0,0)[lb]{\raisebox{0pt}[0pt][0pt]{\elvrm L}}}
\put(185,741){\makebox(0,0)[lb]{\raisebox{0pt}[0pt][0pt]{\elvrm e}}}
\put(185,732){\makebox(0,0)[lb]{\raisebox{0pt}[0pt][0pt]{\elvrm f}}}
\put(185,723){\makebox(0,0)[lb]{\raisebox{0pt}[0pt][0pt]{\elvrm t}}}
\put(255,760){\makebox(0,0)[rb]{\raisebox{0pt}[0pt][0pt]{\elvrm R}}}
\put(253,749){\makebox(0,0)[rb]{\raisebox{0pt}[0pt][0pt]{\elvrm i}}}
\put(255,742){\makebox(0,0)[rb]{\raisebox{0pt}[0pt][0pt]{\elvrm g}}}
\put(255,730){\makebox(0,0)[rb]{\raisebox{0pt}[0pt][0pt]{\elvrm h}}}
\put(254,720){\makebox(0,0)[rb]{\raisebox{0pt}[0pt][0pt]{\elvrm t}}}
\end{picture}

\caption{Image mapping with vector {\tt <1 -1 0>}}
\label{map1}
\end{centering}
\end{figure}
If we reversed the vector, the picture would be transposed:
\begin{verbatim}
    IMAGEMAP <-1 1 0> GIF "filename"
\end{verbatim}
produces the result shown in Figure \ref{map2}.
\begin{figure}[htbp]
\begin{centering}
\setlength{\unitlength}{0.0125in}%
\begin{picture}(109,115)(170,685)
\thinlines
\put(180,700){\vector( 0, 1){100}}
\multiput(260,780)(0.00000,-7.61905){11}{\line( 0,-1){  3.810}}
\multiput(180,780)(7.61905,0.00000){11}{\line( 1, 0){  3.810}}
\put(179,700){\vector( 1, 0){100}}
\put(184,711){\makebox(0,0)[lb]{\raisebox{0pt}[0pt][0pt]{\elvrm m}}}
\put(186,719){\makebox(0,0)[lb]{\raisebox{0pt}[0pt][0pt]{\elvrm o}}}
\put(186,727){\makebox(0,0)[lb]{\raisebox{0pt}[0pt][0pt]{\elvrm t}}}
\put(186,735){\makebox(0,0)[lb]{\raisebox{0pt}[0pt][0pt]{\elvrm t}}}
\put(186,745){\makebox(0,0)[lb]{\raisebox{0pt}[0pt][0pt]{\elvrm o}}}
\put(185,755){\makebox(0,0)[lb]{\raisebox{0pt}[0pt][0pt]{\elvrm B}}}
\put(220,705){\makebox(0,0)[b]{\raisebox{0pt}[0pt][0pt]{\elvrm Left}}}
\put(220,770){\makebox(0,0)[b]{\raisebox{0pt}[0pt][0pt]{\elvrm Right}}}
\put(260,685){\makebox(0,0)[b]{\raisebox{0pt}[0pt][0pt]{\elvrm 1}}}
\put(175,775){\makebox(0,0)[rb]{\raisebox{0pt}[0pt][0pt]{\elvrm 1}}}
\put(250,745){\makebox(0,0)[lb]{\raisebox{0pt}[0pt][0pt]{\elvrm T}}}
\put(251,736){\makebox(0,0)[lb]{\raisebox{0pt}[0pt][0pt]{\elvrm o}}}
\put(251,728){\makebox(0,0)[lb]{\raisebox{0pt}[0pt][0pt]{\elvrm p}}}
\end{picture}

\caption{Image mapping with vector {\tt <-1 1 0>}}
\label{square-map2}
\end{centering}
\end{figure}
Once the image orientation has been determined, it can be translated, rotated,
and scaled as desired to map properly onto the object.

\section{Output File Formats}

\index{output!formats}
People always ask me to describe the output file formats of DKBTrace.  I
received so many requests for this that I decided to put it into the document.
The normal ``default'' output format is ``Dump''\index{dump format}
format.  This is based on QRT format and goes like this, where each
character is a hex digit:

{\center
\fbox{\begin{tabular}{lll}
Header: \\
& {\tt wwww hhhh} & Width, height (16 bits each, LSB first) \\
\multicolumn{2}{l}{For each data line:} \\
 & {\tt llll}         &
\parbox[t]{2.5in}{line number (16 bits, LSB first, 0 to LINES-1)} \\
 & {\tt rr rr rr rr rr} \ldots & the red components for that line \\
 &                    & (8 bits each - 0=dark, 255=bright) \\
 & {\tt gg gg gg gg gg} \ldots & the green components for that line \\
 &                    & (8 bits each - 0=dark, 255=bright) \\
 & {\tt bb bb bb bb bb} \ldots & the blue components for that line \\
 &                    & (8 bits each - 0=dark, 255=bright) \\
\end{tabular}}}

\vskip10pt
Note that this format is slightly different from QRT's.

The\index{raw format} {\tt +fr}\optindex{f} option of the raytracer
produces ``raw'' files.  These are simply three files with no header
information and no line number information -- just the raw data.

The\index{Targe format} {\tt +ft} option writes out Targa format.
Specifically, the fields are:

{\center
\fbox{\begin{tabular}{lll}
Header: \\
 & {\tt 00 00 02 00 00} & Fixed header information for\ldots \\
 & {\tt 00 00 00} & \hspace{0.2in}\ldots uncompressed type 2 image \\
 & {\tt 0000} & Horizontal offset always is at 0000 \\
 & {\tt llll} &
\parbox[t]{2.5in}{Vertical offset (1st line number, 16 bits, LSB first)} \\
 & {\tt wwww hhhh} &
\parbox[t]{2.5in}{width, height of image (16 bits each, LSB first)} \\
 & {\tt 18 20} & 24 bits per pixel, Top-down raster \\
\multicolumn{2}{l}{For each data line:} \\
 & {\tt bb gg rr bb gg rr} \ldots  &
\parbox[t]{2.5in}{blue, green, and red data, 8 bits for each pixel in
that line.} \\
\end{tabular}}}

\chapter{Common Questions and Answers}

\index{questions}
I often get asked the same questions again and again.  I usually take this to
mean that the documentation is not complete or not sufficiently clear.  In
order to correct this problem, I've added some sections to the document
describing the features in more detail.  I've also collected some of the more
popular questions and answered them here.

\begin{itemize}
\item[Q:] Will you be implementing radiosity\index{radiosity}?
\item[A:] I don't expect so.  The techniques for radiosity are quite involved
and time consuming (although they are getting faster).  The amount of
effort required to implement radiosity is beyond my current plans.

\item[Q:] Do you intend make DKBTrace RenderMan\index{RenderMan} compatible?
\item[A:] Probably not.  RenderMan is a specification that requires much more
functionality than DKBTrace currently provides.  The camera models,
modelling primitives, and shading language of RenderMan are all very
involved and difficult to implement.  As well, RenderMan is not well
suited to a raytracing approach.  Don't expect to see DKBTrace
RenderMan compatible in the near (or distant) future.  Note, though
that several of the DKBTrace textures are similar to those obtainable
by RenderMan.

\item[Q:] I\index{light sources!position restriction} defined a light
source but the shadows and lighting are all wrong.
\item[A:] Light sources must be defined at the origin (0,0,0) and translated to
the proper place.  The reason for this is to allow the diffuse lighting
calculations to quickly determine where the center of the light source
is.  It's a very difficult task to calculate the center of an object
(in general), so I simply take the place that the object was translated
to as the center of the light source.

\item[Q:] I keep running out of memory\index{memory usage}.  What can I do?
\item[A:] Buy more memory.  But seriously, you can decrease the memory
requirements for any given picture in several ways:
\begin{enumerate}
\item {\tt DECLARE}\keyindex{DECLARE} texture constants and use them
(textures are shared).
\item Don't modify the texture that you are sharing.  On the first
modify, the texture is copied and (therefore) takes up more space.
\item Put the object transformations before the texture structure.  This
prevents the texture from being transformed (and hence, copied.
This may not always be desirable, though).
\item Use {\tt UNION}s\keyindex{UNION} instead of
{\tt COMPOSITE}\keyindex{COMPOSITE} objects to put
pieces together. Previous versions of the raytracer didn't allow
this because the texture applied to the entire object.  Version 2.10
and up allow you to change the texture on a per-shape basis.
\item Use fewer or smaller image maps.
\item Use GIF\index{GIF} or IFF\index{IFF} (non-HAM) images for image
maps.  These are stored internally as 8 bits per pixel with a colour
table instead of 24 bits per pixel.
\end{enumerate}

\item[Q:] I get a floating point exception error\index{errors} on
certain pictures. What's wrong?
\item[A:] Oh no! Not another one!  The raytracer (obviously) performs
{\em many} floating point operations when tracing a scene.  If I had to check
each one for overflow or underflow, the program would be much longer
and I would be much closer to going insane trying to locate all
possible cases.  If you get this problem, I'd suggest that you
first look through your data file to make sure you're not doing
something stupid like:
\begin{itemize}
\item Scaling something by 0.0 in {\em any} dimension;
\item Making the {\tt LOOK_AT}\keyindex{LOOK_AT} point the same as
the {\tt LOCATION}\keyindex{LOCATION};
\item Defining triangles\index{triangles!math errors in} with two
points the same (or nearly the same);
\item Using the zero vector for normals.
\end{itemize}
If it doesn't seem to be one of these problems, let me know, but there
may not be a lot I can do because overflows can occur almost anywhere.
Sorry.  If you do have such troubles, you can try to isolate the
problem in the input data file by commenting out objects or groups
of objects until you narrow it down to a particular section that
fails.  Then try commenting out the individual characteristics of the
offending object.

\item[Q:] No matter how much I scale a Cylinder, I can't make it fit on the
screen.  How big is it and how much do I have to
scale\index{scaling qaudrics}\index{quadrics!scaling} it?
\item[A:] Cylinders (like most quadrics) are infinitely long.  No matter how
much you scale them, they still won't fit on the screen.  To make a
capped cylinder, you must use CSG:
\begin{verbatim}
     INTERSECTION
         QUADRIC Cylinder_Y END_QUADRIC
         PLANE <0.0 1.0 0.0> 1.0 END_PLANE
         PLANE <0.0 -1.0 0.0> 1.0 END_PLANE
     END_INTERSECTION
\end{verbatim}
Cylinders {\em can} be scaled in cross-section, the two vectors
{\em not} in the name of the cylinder (X and Z, in our example above) can
be scaled to control the width and thickness of the cylinder.  Scaling
the Y value (which is normally infinite) is meaningless, unless you
have ``capped'' it as above, then scaling the entire
{\tt INTERSECTION}\keyindex{INTERSECTION} object in the Y dimension
will control the height of the cylinder.

\item[Q:] Why don't you define a primitive\index{cube}\index{box} for a
6-sided box?
\item[A:] Because you can do it so easily with
CSG:\index{constructive solid geometry!definition of box}
\begin{verbatim}
     INTERSECTION
        PLANE < 1.0  0.0  0.0> 1.0 END_PLANE
        PLANE <-1.0  0.0  0.0> 1.0 END_PLANE
        PLANE < 0.0  1.0  0.0> 1.0 END_PLANE
        PLANE < 0.0 -1.0  0.0> 1.0 END_PLANE
        PLANE < 0.0  0.0  1.0> 1.0 END_PLANE
        PLANE < 0.0  0.0 -1.0> 1.0 END_PLANE
     END_INTERSECTION
\end{verbatim}

\item[Q:] Are planes 2D objects or are they 3D but infinitely thin?
\item[A:] Neither.  Planes are 3D objects that divide the world into two
half-spaces.\index{plane!inside and outside}
The space in the direction of the surface normal is considered outside
and the other space is inside (see Figure \ref{plane}).
\begin{figure}[htbp]
\begin{centering}
\setlength{\unitlength}{0.0125in}%
\begin{picture}(125,92)(140,720)
\thinlines
\put(160,740){\vector( 0, 1){ 60}}
\put(140,740){\line( 1, 0){120}}
\put(160,805){\makebox(0,0)[b]{\raisebox{0pt}[0pt][0pt]{\elvrm Normal}}}
\put(265,735){\makebox(0,0)[lb]{\raisebox{0pt}[0pt][0pt]{\elvrm Plane}}}
\put(180,720){\makebox(0,0)[lb]{\raisebox{0pt}[0pt][0pt]{\elvrm Inside}}}
\put(190,750){\makebox(0,0)[lb]{\raisebox{0pt}[0pt][0pt]{\elvrm Outside}}}
\end{picture}

\caption{The ``sides'' of a plane}
\label{plane}
\end{centering}
\end{figure}
In other words, planes are 3D objects that are infinitely thick.

\item[Q:] Can DKBTrace render soft shadows\index{shadows}?
\item[A:] No.  This would require a lot more programming work and a lot more
calculation time.  You can place an unturbulated {\tt WOOD}\keyindex{WOOD}
texture ``filter'' over a lamp in a can, and use a colour map going from dark
({\tt ALPHA}\keyindex{ALPHA} 0.0) around the edges to clear
({\tt ALPHA} 1.0) at the center, and get a soft-shadow-like effect for
a ``spotlight''.

\item[Q:] I'd like to go through the program and hand-optimize the
assembly code in places to make it faster\index{processing speed}.
What should I optimize?
\item[A:] Don't bother.  With hand optimization, you'd spend a lot of
time to get perhaps a 5-10% speed improvement at the cost of total
loss of portability.  If you use a better ray-surface intersection
algorithm, you should be able to get an order of magnitude or more
improvement.  Check out some books and papers on raytracing for useful
techniques.  Specifically, check out ``Spatial Subdivision'' and ``Ray
Coherence'' techniques.

\item[Q:] Objects on the edges of the screen seem to be
distorted.\index{distortion}  Why?
\item[A:] If the {\tt DIRECTION}\keyindex{DIRECTION} vector of the
viewpoint is not very
long, you may get distortion at the edges of the screen.  The reason
for this is that the viewpoint's screen is flat, as illustrated in
Figure \ref{distort}.
\begin{figure}[htbp]
\begin{centering}
\setlength{\unitlength}{0.0125in}%
\begin{picture}(110,135)(90,625)
\thinlines
\put(180,730){\framebox(20,20){}}
\put(140,670){\vector( 2, 3){ 40}}
\put(140,670){\vector( 1, 2){ 40}}
\put(140,670){\vector( 4, 1){ 40}}
\put(140,670){\vector( 4,-1){ 40}}
\put(180,660){\framebox(20,20){}}
\put(160,760){\line( 0,-1){120}}
\put(190,665){\makebox(0,0)[b]{\raisebox{0pt}[0pt][0pt]{\elvrm B}}}
\put(190,735){\makebox(0,0)[b]{\raisebox{0pt}[0pt][0pt]{\elvrm A}}}
\put(135,665){\makebox(0,0)[rb]{\raisebox{0pt}[0pt][0pt]{\elvrm Viewpoint}}}
\put(160,625){\makebox(0,0)[b]{\raisebox{0pt}[0pt][0pt]{\elvrm Viewing screen}}}
\end{picture}

\caption{Distortion with short {\tt DIRECTION} vector}
\label{distort}
\end{centering}
\end{figure}
The object labelled A appears smaller than the object labelled B,
despite their actually being the size size.

\item[Q:] How do you position\index{image mapping!positioning}
image maps without a lot of trial and error?
\item[A:] By default, images will be mapped onto the range 0,0 to 1,1 in the
appropriate plane.  You should be able to translate, rotate, and
scale the image from there.

\item[Q:] What's the difference between {\tt ALPHA}\keyindex{ALPHA}
and {\tt REFRACTION}\keyindex{REFRACTION}?
\item[A:] The difference is a bit subtle.  Alpha is a component of
a colour that determines how much light can pass through that colour.
Refraction is a property of a surface that determines how much light
can come from inside the surface.  See the section above on
Transparency and Refraction for more details.

\item[Q:] How do you calculate the surface
normals\index{triangle!surface normals} for smooth triangles?
\item[A:] When I implemented smooth triangles, I never really expected
anyone to manually calculate the surface normals.  There are two ways
of getting another program to calculate them for you:
\begin{enumerate}
\item Depending on the type of input to the program, you may be able to
calculate the surface normals directly.  For example, if you have
a program that converts B-Spline or Bezier Spline surfaces into
DKB-format files, you can calculate the surface normals from the
surface equations.
\item If your original data was a polygon or triangle mesh, then it's
not quite so simple.  You have to first calculate the surface
normals of all the triangles.  This is easy to do -- you just use
the vector cross-product of two sides (make sure you get the
vectors in the right order).  Then, for every vertex, you average
the surface normals of the triangles that meet at that vertex.
These are the normals you use for smooth triangles.
\end{enumerate}

\item[Q:] When I render parts of a picture on different systems, the
textures\index{textures!randomness} don't match when I put them
together.  Why?
\item[A:] The appearance of a texture depends on the particular random number
generator used on your system.  DKBTrace seeds the random number
generator with a fixed value when it starts, so the textures will be
consistent from one run to another or from one frame to another so
long as you use the same executables.  Once you change executables,
you will likely change the random number generator and, hence, the
appearance of the texture.  There is an example of a standard ANSI
random number generator provided in {\tt IBM.C}, include it in your
machine-specific code if you are having consistency problems.

\item[Q:] What's the difference\index{colour!and {\tt TEXTURE}}
between a {\tt COLOUR}\keyindex{COLOUR}
declared inside a {\tt TEXTURE}\keyindex{TEXTURE} and one that's in
a shape or an object and not in a texture?
\item[A:] The colour in the texture specifies the colour to use for qualities
5 and up.  The colour on the shape and object are used for faster
rendering in qualities 4 and lower and for the colour of light sources.
See the {\tt -q}\optindex{q} option for details on the quality parameter.

\item[Q:] I created an object that passes through its bounding
volume\index{bounding shapes}.
At times, I can see the parts of the object that are outside the
bounding volume.  Why does this happen?
\item[A:] Bounding volumes are {\em not} designed to change the
shape of the object.  They are strictly a realtime improvement
feature.  The raytracer trusts you when you say that the object is
enclosed by a bounding volume.  The way it uses bounding volumes is
very simple: If the ray hits the bounding volume (or the ray's origin
is inside the bounding volume), then the object is tested against that
ray.  Otherwise, we ignore the object.  If the object extends beyond
the bounding volume, anything goes.  The results are undefined.  It's
quite possible that you could see the object outside the bounding
volume and it's also possible that it could be invisible.  It all
depends on the geometry of the scene.

\item[Q:] Will you be writing a Graphical User Interface for DKB?
\item[A:] No, but several other people have expressed an interest in writing
one.  I'd like to form a mailing list to get all these people in touch
with each other, so if you have an interest in this, please let me
know.

\item[Q:] When will the next version be available?
\item[A:] If I told you, I'd be lying because I don't know.  I'm finding that
releasing an official version of a program like this is a major
effort.  It requires not only the changes to the code, but also
changes to the documentation.  Sometimes (as with this release), I
have to change the older data files to conform to the new syntax.
Usually, I have to spend a lot of time re-rendering scenes that used
to work to make sure they still do.  That combined with sending the
files to beta testers, getting feedback, making fixes, and re-issuing
the changes adds up to a lot of work.  I don't expect I'll be doing
it terribly often.

Bottom line -- If I say ``next week'', don't believe me.  I'm probably
wrong.
\end{itemize}

\chapter{Converting Data Files from Versions Prior to 2.10}

Unfortunately, version 2.10 changed the format of the data files.
These changes were mostly to avoid confusion -- especially in the case
of layered textures.  There were also some changes that were made to
allow more flexible library objects.  This section details the changes
in the data file format and why it was necessary to make those
changes.  I'm sorry that you have to go back through all your old data
files and edit them all, but I think you'll agree (I hope you'll
agree) that the changes are for the better.

\begin{enumerate}
\item The following keywords are no longer accepted outside of a
{\tt TEXTURE}\keyindex{TEXTURE} block:
\keyindex{AMBIENT}\keyindex{DIFFUSE}\keyindex{BRILLIANCE}\keyindex{REFLECTION}
\keyindex{REFRACTION}\keyindex{IOR}\keyindex{PHONG}\keyindex{PHONGSIZE}
\keyindex{SPECULAR}\keyindex{ROUGHNESS}
\begin{quote}
{\tt AMBIENT}, {\tt DIFFUSE}, {\tt BRILLIANCE}, {\tt REFLECTION},
{\tt REFRACTION}, {\tt IOR}, {\tt PHONG}, {\tt PHONGSIZE},
{\tt SPECULAR}, {\tt ROUGHNESS}
\end{quote}
\begin{description}
\item[Reason:] For layered textures,\index{textures!layered} it's not
clear which texture
these keywords apply to if they're not in a {\tt TEXTURE} block.  I was
really getting messed up in the implementation trying to locate the
proper references for these keywords.  It's much cleaner to only allow
them inside a {\tt TEXTURE} block.
\end{description}

\item A new texture called {\tt COLOUR}\keyindex{COLOUR} (or
{\tt COLOR}) has been added.  This texture simply specifies a simple
colour to use.
\begin{description}
\item[Reason:] It's useful to be able to declare a {\tt TEXTURE}
with the colour embedded inside it.  Besides, I had to provide some
way of specifying a simple colour after I made the following change.
\end{description}

\item The keyword {\tt COLOUR} or {\tt COLOR} when used outside of a
{\tt TEXTURE} block is only used to provide a colour for low-quality
renderings (ones where the {\tt -q}\optindex{q} value is 5 or below.
\begin{description}
\item[Reason:] This is the same reason as for change 1.  For layered
textures, it wasn't clear what the object or shape colour meant when
the texture itself contained colour information.  Rather than have a
convoluted searching scheme (which I tried at first, then abandoned
due to difficulties in explaining it), I decided to keep it simple.
If you want a simple colour, put it in a {\tt TEXTURE} block.
\end{description}

\item When an object or shape is transformed, the textures attached to it are
transformed as well.
\begin{description}
\item[Reason:] Previously, you couldn't {\tt DECLARE}\keyindex{DECLARE}
objects with
textures, then create identical copies of them in different places.
This was especially annoying when you create an object that has an
image mapped onto it.  As soon as you moved the object, the image was
left behind.  Unfortunately, this means that the texture
transformations inside {\tt TEXTURE} blocks in old data files would double
transform the texture, and must be removed.
\end{description}

\item The interaction between {\tt ALPHA}\keyindex{ALPHA} colours
and {\tt REFRACTION}\keyindex{REFRACTION}
has changed.  Previously, there was no interaction at all.  If you had
a surface that contained an {\tt ALPHA} colour, you could see through
the surface.  If the surface had refraction, you could also see
through.  With the new release, {\tt ALPHA} and {\tt REFRACTION} in
combination tell you how much light is passed through from the inside
of the object.

To make things a bit easier, if an object has an {\tt ALPHA} component
but the {\tt REFRACTION} is 0.0 (or unspecified), the renderer will
simply transmit the ray through the object without any refractive
bending.
\begin{description}
\item[Reason:] It seemed to make sense.  The code for {\tt ALPHA} was
doing the same work as the code for {\tt REFRACTION} except for
actually bending the ray.  Now, you can create objects that are
partially opaque and partially transparent. Where they are
transparent, the light passing through the object is bent by the index
of refraction.  Makes sense, no?
\end{description}

\item When you use an {\tt IMAGEMAP}\keyindex{IMAGEMAP}
with the {\tt ONCE}\keyindex{ONCE}\index{image mapping!and {\tt ONCE}}
option, the colour outside the mapped image is transparent
{\tt (RED 1.0 GREEN 1.0 BLUE 1.0 ALPHA 1.0)}.
\begin{description}
\item[Reason:] This allows you to see the underlying textures.
\end{description}

\item I've removed {\tt BasicShapes.data} and replaced it with
{\tt shapes.dat}, {\tt colors.dat}, and {\tt textures.dat}.
\begin{description}
\item[Reason:] The old name didn't apply any more (since it contained
more than just shapes).  Also, I was tired of the IBM people using the
name {\tt BASICSHA.DAT}.
\end{description}

\item The textures in {\tt textures.dat} (formerly
{\tt BasicShapes.data}) were previously scaled to
{\tt <10.0 10.0 10.0>}.  This scaling has been removed.
\begin{description}
\item[Reason:] The factor of 10.0 seemed to be a totally arbitrary
scale factor that someone used.  I'd like to keep the scaling factors
up to the users.
\end{description}
\end{enumerate}

\chapter{Handy Hints}

This chapter provides a list of helpful hints.

\begin{itemize}
\item To see a quick version of your picture, use
{\tt -w64 -h80}\optindex{w}\optindex{h} as command
line parameters on the Amiga.  For the IBM, try {\tt -w80 -h50}.  This
displays the picture in a small rectangle so that you can see how
it will look.

\item Try using the sample default files for different usages --
{\tt QUICK.DEF}
shows a fast postage-stamp rendering (80x50, as above) to the
screen only, {\tt LOCKED.DEF} will display the picture with anti-aliasing
on (takes forever) with no abort (do this before you go to bed...).
The normal default options file
{\tt trace.def}\index{trace.def@{\tt trace.def} startup file} is read
and you can supersede this with another {\tt .def} file by specifying
it on the command line, for example:
\begin{verbatim}
     trace -iworld.dat -oworld.out quick.def
\end{verbatim}

\item When translating light sources\index{light sources!translating},
translate the {\tt OBJECT}\keyindex{OBJECT}, not the
{\tt QUADRIC}\keyindex{QUADRIC} surface.  The light source uses the
center of the object as the origin of the light.

\item When animating\index{textures!animation of} objects with solid
textures, the textures must
move with the object, i.e. apply the same {\tt ROTATE} or
{\tt TRANSLATE} functions to the texture as to the object itself.  This is
now done automatically in version 2.10 if the transformations are
placed {\em after} the {\tt TEXTURE} block.

\item You can declare constants for most of the data types in the program
including floats and vectors.  By combining this with
{\tt INCLUDE}\keyindex{INCLUDE} files,you can easily separate the
parameters for an animation into a separate file.

\item The amount of ambient light plus diffuse light should be less than
or equal to 1.0\index{lighting!total amount of}. The program accepts
any values, but may produce strange results.

\item When using {\tt RIPPLES}\keyindex{RIPPLES}, don't make the
ripples too deep or you may get
strange results (the dreaded ``digital zits''!).

\item {\tt WOOD}\keyindex{WOOD} textures\index{textures!transforming}
usually look better when they are scaled to different
values in $x$, $y$, and $z$, and then rotated to a different angle.

\item You can sort of dither\index{dithering, simulated} a colour
by placing a floating point number into the texture record:
\begin{verbatim}
     TEXTURE
          0.05
     END_TEXTURE
\end{verbatim}
This adds a small random value to the intensity of the diffuse
light on the object.  Don't make the number too big or you may get
strange results.  You generally won't want to use this if you are
rendering frames for animations\index{textures!animation of}, as
this ``dithering'' is random\index{textures!randomness}.
Better results can be obtained, however, by doing proper dithering
in a post-processor.

\item You can compensate for non-square aspect ratios\index{aspect ratio}
on the monitors by making the {\tt RIGHT}\keyindex{RIGHT} vector in
the {\tt VIEWPOINT}\keyindex{VIEWPOINT}\index{viewpoint!aspect ratio}
longer or shorter.  A good value for the Amiga is about 1.333.
This seems okay for IBM's too at 320x200 resolution.  If your spheres
and circles aren't round, try varying it.

\item If you are importing images from other systems, you may find that
the shapes are backwards (left-to-right inverted) and no rotation
can make them right.  All you have to do is negate the terms in the
{\tt RIGHT} vector of the viewpoint to flip the camera left-to-right (use
the ``right-hand'' coordinate system).

\item By making the {\tt DIRECTION}\keyindex{DIRECTION} vector in
the {\tt VIEWPOINT}\keyindex{VIEWPOINT} longer, you can achieve the
effect of a tele-photo\index{tele-photo lens effect} lens.

\item When rendering on the Amiga, use a resolution of 320 by 400 to
create a full sized HAM picture.
\end{itemize}

\chapter{Compiling the Code}

\index{compiling}
If you want to compile the source code on a supported platform, do the
following:

\begin{enumerate}
\item Copy or rename the file called {\tt {\em system}conf.h} to
{\tt config.h}, where {\em system} corresponds to the platform you
wish to compile for. For example:
\begin{verbatim}
       rename amigaconf.h config.h
\end{verbatim}

\item Copy or rename the appropriate make file to {\tt makefile}. For
example:
\begin{verbatim}
       rename amigamake makefile
\end{verbatim}

\item Edit the makefile to make sure all compiler options are set up
properly for your system.

\item Type {\tt make}.
\end{enumerate}

\chapter{Porting to Different Platforms}

\index{porting}
I've taken great pains to make DKBTrace as portable as possible.  So
far, it's working out fairly well.  For the most part, the core of the
raytracer will compile with any decent C compiler.

If you want to port the raytracer to another system, please try to
modify the core of the raytracer as little as possible.  Each system
will have its own makefile, config file, and C file.  The config file
is included by all of the raytrace modules and can be used to perform
special {\tt \#defines} for that system.  The C file should contain all
system-specific code.  It must implement the following functions:

{\footnotesize
\begin{verbatim}
   void display_init()      /* Open the graphics device and initialize it */
   void display_close()     /* Close the graphics device                  */
   void display_finished()  /* Perform any operations required after      */
                            /* finishing the rendering but before quitting*/
   void display_plot (x, y, Red, Green, Blue)
                            /* Display the specified colour at point x,y  */
\end{verbatim}
}

\noindent
In your config file, you may customize the following things:

{\footnotesize
\begin{verbatim}
   #define FILE_NAME_LENGTH        /* default 150                        */
   #define DBL_FORMAT_STRING       /* the string to use to read a double */
   #define DEFAULT_OUTPUT_FORMAT   /* 'd', 'r' or 't' - default output   */
                                   /*  file format                       */
   #define TEST_ABORT              /* the operation to perform while     */
                                   /* tracing to see if we should abort  */
                                   /* the trace.                         */
   #define RED_RAW_FILE_EXTENSION  /* the default extensions for +fr     */
   #define GREEN_RAW_FILE_EXTENSION
   #define BLUE_RAW_FILE_EXTENSION
   #define STARTUP_DKB_TRACE       /* the code to call immediately after */
                                   /* starting the main program. Useful  */
                                   /* if you don't have a command-line   */
                                   /* interface.                         */
   #define PARAMS(x)               /* "(x)" if you have prototypes, "()" */
                                   /* otherwise.                         */
\end{verbatim}
}

\chapter{Program History and Information}

\section*{Version 1.2}

Initial Amiga release.

\section*{Version 2.0}

\index{Collins, Aaron}\index{Buck, David}
\begin{itemize}
\item First release Version 2.0 Conversion to the IBM done by Aaron A. Collins.
\item New textures, Specular and Phong highlighting added by Aaron A. Collins.
\item Triangle, Smooth Triangle, Sphere, Plane support added by David Buck.
\item RAW, IFF and GIF image mapping added by David Buck and Aaron Collins.
\item Transparency and Fog added by David Buck.
\item GIF format file reader by Steve Bennett\index{Bennett, Steve}
(used with permission).
\item New Noise and DNoise functions by Robert Skinner\index{Skinner, Robert}
(used with permission).
\item TARGA format output file capability added by Aaron A. Collins.
\item ANSI-C function prototyping for ALL functions.
\item Reversal of the order of writing screen data from the original DKB/QRT
``RAW'' file format.
\item For IBM's, it has a crude VGA 320x200 by 256 color display rendering
ability.
\end{itemize}

Version 2.0 compiles under Turbo-C 2.0 on the IBM P.C. and Lattice C
5.05 on the Amiga.  The only file which contains the ANSI extensions
is {\tt dkbproto.h}, so for non-ANSI compilers, you only need to remove the
declaration of the parameters in the {\tt config.h} file and the whole thing
should compile.  There are several example {\tt config.h} files for Amiga,
IBM, and Unix.  The appropriate one should be copied over {\tt CONFIG.H},
and the {\tt MAKEFILE} should be edited for your particular system and
compiler configuration before compilation. 
    
Version 2.0 has a significant difference from prior releases: Speed!
The new primitives of {\tt SPHERE}\keyindex{SPHERE},
{\tt PLANE}\keyindex{PLANE}, {\tt TRIANGLE}\keyindex{TRIANGLE}, etc.
greatly speed up tracing\index{processing speed}. Another
significant speed-up is that world
X-Y-Z values beyond 10 Million or so are ignored, and ray tracing
beyond that distance will cease.  This produces 2 minor peculiarities:

\begin{enumerate}
\item A black stripe at the horizon point of Pre-2.0 scene description
{\tt .data} files that have ``ground'' and ``sky'' planes defined.  The
planes were traced out to a much greater ``infinity'' so this effect was
unnoticeable, prior to version 2.0.
\item Tiny black pixels in the texture, or ``Surface Acne''.
\end{enumerate}

This is usually caused by rays being refracted or reflected such that
the ray does not happen to hit any object, and eventually becomes
black in color as it gets too far away and gets clipped.  This effect
can be minimized by enclosing the scene with distant ``walls'',
``floors'', or placing ``ocean floors'' beneath water, etc.  So far, no
scenes have required placing such planes behind the camera, unless an
``environment map'' of sorts is desired.  See {\tt SKYTEST.DAT} for several
examples of spurious distant planes.  If your ``acne'' still doesn't go
away, it may be due to a large pixel sample area and it's accidentally
picking a point which is just inside the primitive being hit.  This is
a more tricky problem to solve, and anti-aliasing the image will
definitely help if this sort of thing occurs. 

\section*{Version 2.10}

A few unofficial releases were made between 2.01 and 2.10.  The following
points capture the major changes:

\begin{itemize}
\item Less memory is required for image mapped GIF and IFF files.
\item The output format command-line option was changed to
{\tt +f{\em x}}\optindex{f} where {\em x} denotes the output format.
\item The display option {\tt +d}\optindex{d} now takes an optional extra
character {\tt +d{\em x}} where {\em x} is system-dependent.  This
allows you to specify the graphics mode by a command-line switch.
\item The tokenizing pass has been removed.  It's now called directly by the
parser.
\item The environment variable
{\tt DKBOPT}\index{DKBOPT@{\tt DKBOPT} environment variable} is used
in addition to the
{\tt trace.def}\index{trace.def@{\tt trace.def} startup file} file
and the command-line options.
\item The numbers in the data file can now use full scientific
notation, eg. 10.23e-4.
\item The {\tt +c}\optindex{c} option was added to continue an aborted trace.
\item You can now colour or texture each component of the CSG's separately.
\item Layered textures\index{textures!layered} implemented (see the
section on textures).
\item When using GIF and IFF images for image mapping, you can now
indicate that specified registers are partially or completely transparent.
\item Textures are now transformed whenever the object or shape they
are attached to are transformed.
\item The texture {\tt CHECKER_TEXTURE}\keyindex{CHECKER_TEXTURE} has
been added.
\item All keywords relating to the appearance of the surface have been made
illegal in an object definition unless they are inside a {\tt TEXTURE} block.
\item The ``basicshapes.data'' file has been split up into {\tt shapes.dat},
{\tt colors.dat}, and {\tt textures.dat}.  These files have also been expanded
with more useful declarations.
\item The {\tt -l}\optindex{l} command-line option has been
added to support library directories.
\end{itemize}

\section*{Version 2.11}

\begin{itemize}
\item Quartic surfaces\index{quartic surfaces} (4th order) added by
Alexander Enzmann\index{Enzmann, Alexander}.
\item Parser now accepts {\tt Ctrl-Z} as a whitespace.
\item Keyword {\tt END_SMOOTH_TRIANGLE}\keyindex{END_SMOOTH_TRIANGLE}
added (previously, {\tt END_TRIANGLE} was used.)
\end{itemize}

\section*{Version 2.12}

\begin{itemize}
\item Bug in smooth triangles fixed to allow them to be scaled and translated.
\item {\tt METALLIC}\keyindex{METALLIC} texture added.
\end{itemize}

\chapter{References}

I'm always asked about good books on raytracing and graphics in general.
This section address that issue by listing several good books or periodicals
that you should be able to locate in your local computer book store or your
local university library.

\begin{itemize}
\item J. D. Foley and A. Van Dam.
{\em Fundamentals of Interactive Computer Graphics}.
Addison-Wesley, 1983.

\item J. D. Foley, A. van Dam, J. F. Hughes.
{\em Computer Graphics:  Principles and Practice}, (2nd Ed.).
Addison-Wesley, 1990.

\item Andrew S. Glassner (ed.).
{\em An Introduction to Raytracing}.
Academic Press, 1989.

\item Andrew S. Glassner (ed.)
{\em Graphics Gems}.
Academic Press, 1990.

\item Nadia Magnenat-Thalman and Daniel Thalmann.
{\em Image Synthesis:  Theory and Practice}.
Springer-Verlag, 1987.

\item Clifford Pickover.
{\em Computers, Pattern, Chaos, and Beauty}.
St. Martin's Press.

\item David von Seggern
{\em The CRC Handbook of Mathematical Curves and Surfaces}
CRC Press, 1990.

\item Steve Upstill.
{\em The RenderMan Companion}.
Addison Wesley, 1989.

\item {\em SIGGRAPH Conference Proceedings}.
Association for Computing Machinery,
Special Interest Group on Computer Graphics.

\item {\em IEEE Computer Graphics and Applications}
The Computer Society
10662, Los Vaqueros Circle
Los Alimitos, CA 90720

\item {\em The CRC Handbook of Standard Mathematical Tables}
CRC Press.
\end{itemize}

\chapter{Concluding Remarks}

It seems that in any project like this, as soon as you fix some bugs,
some more appear.  I expect that the new features I provided in this
release will cause some problems somewhere.  There's only so much you
can do to get all the bugs out.  For the next little while, I intend
to keep the program stable feature-wise and just tackle major bugs.
Hopefully things will stablize this way (and I can take a much needed
breather).

Thanks go out to Alexander Enzmann\index{Enzmann, Alexander} for
generously providing the code
for the {\tt QUARTIC} surface algorithms, and for his assistance in
implementing it and for helping squash a couple of other bugs.

I would like to thank all the people who used versions 2.01 and 2.04
and provided useful comments and helpful suggestions.  The program has
been much improved with your help.

I would also like to thank my beta testers for all the help, bug
reports, suggestions, comments, and time spent.  I'd like to say a
special thank-you to Aaron Collins\index{Collins, Aaron} who has
fielded many of the
questions on the raytracer and has been invaluable in debugging,
testing, and generally improving the program.  Thanks guys.

\vskip 0.5in
\hfill David Buck\index{Buck, David}

\chapter{About the Authors}

{\setlength{\parskip}{5pt}
David Buck\index{Buck, David} can be reached on the following BBS'es:

\begin{tabular}{l l}
OMX & (613) 731-3419 \\
Mystic & (613) 731-0088 or (613) 731-6698 \\
FidoNet & {\tt 1:163/109.9} \\
Internet & {\tt dbuck@ccs.carleton.ca} \\
CIS & {\tt 70521,1371} \\
"You can call me Ray" & (708) 358-5611
\end{tabular}

\noindent
Aaron Collins\index{Collins, Aaron} (IBM port) can be reached on the
following BBS'es:

\begin{tabular}{l l}
"You can call me Ray" & (708) 358-5611
\end{tabular}

As of July of 1990, there is a Ray-Trace specific BBS in the (708)
Area Code (Chicago suburbia) for all you Traceaholics out there.  The
phone number of this new BBS is (708) 358-5611.  Aaron Collins is
Co-Sysop of that board, and David Buck is a frequent caller.  There is now
also a Ray-Trace and Computer-Generated Art specific SIG on
Compuserve, {\tt GO COMART}.

The \LaTeX\ version of the manual was produced by George
Ferguson\index{Ferguson, George}, ({\tt ferguson@cs.rochester.edu}).
}


% Note: If the index file has underscores that aren't preceded by
% a backslash, then they come out as raised dots. Use sed on the
% output of makeindex, if you haven't already.
\begin{theindex}

  \item {\ptt a} option, 5, 8
  \item {\ptt AGATE} keyword, 48
  \item {\ptt ALL} keyword, 51
  \item {\ptt ALPHA} keyword, 20, 43, 47, 50, 51, 66, 71, 78, 79, 83
  \item {\ptt AMBIENT} keyword, 42, 82
  \item animation, \see{textures, animation of}{19}
  \item anti-aliasing, \see{{\ptt a} option}{8}
  \item aspect ratio, 86
  \item {\pem {}AutoCad}, 60
  \item {\pem {}Autodesk Animator}, 57

  \indexspace

  \item {\ptt b} option, 9
  \item Bennett, Steve, 1, 90
  \item bounding shapes, 53, 80
  \item box, 77
  \item {\ptt BOZO} keyword, 41, 46, 69
  \item {\ptt BRILLIANCE} keyword, 42, 82
  \item Buck, David, 90, 96, 97
  \item buffer size, \see{{\ptt b} option}{9}
  \item {\ptt BUMPS} keyword, 50

  \indexspace

  \item {\ptt c} option, 9, 92
  \item camera model, 30
  \item {\ptt catdump}, 61
  \item {\ptt CHECKER} keyword, 41, 45
  \item {\ptt CHECKER\_TEXTURE} keyword, 45, 92
  \item {\ptt Chem2DKB}, 60
  \item Collins, Aaron, 60, 62, 90, 96, 97
  \item {\ptt COLOUR} keyword, 39, 49, 80, 82
  \item colour
    \subitem and color, 25
    \subitem and {\tt TEXTURE}, 80
    \subitem data type, 25
  \item colour map data type, 25
  \item command-line options, 6
    \subitem {\ptt a} option, 5, 8
    \subitem {\ptt b} option, 9
    \subitem {\ptt c} option, 9, 92
    \subitem {\ptt d} option, 4, 7, 92
    \subitem {\ptt e} option, 9
    \subitem {\ptt f} option, 4, 6, 55, 56, 62, 74, 92
    \subitem {\ptt h} option, 4, 6, 85
    \subitem {\ptt i} option, 5, 8
    \subitem {\ptt l} option, 10, 12, 52, 92
    \subitem {\ptt o} option, 5, 55
    \subitem {\ptt p} option, 5, 7
    \subitem {\ptt q} option, 10, 39, 49, 80, 83
    \subitem {\ptt s} option, 9
    \subitem {\ptt v} option, 4, 6
    \subitem {\ptt w} option, 4, 6, 85
    \subitem {\ptt x} option, 5, 8
    \subitem {\ptt z} option, 10
  \item comments, 23
  \item {\ptt compdump}, 61
  \item compiling, 87
  \item {\ptt COMPOSITE} keyword, 52, 76
  \item constructive solid geometry, 37
    \subitem and bounding shapes, 53
    \subitem definition of box, 77
    \subitem difference, 37
    \subitem intersection, 37
    \subitem transforming, 38
    \subitem union, 37
  \item continue rendering, \see{{\ptt c} option}{9}
  \item coordinate system, 11
    \subitem inverted Y, 64
    \subitem left-handed, 24, 30
    \subitem right-handed, 30
  \item CSG, \see{constructive solid geometry}{37}
  \item cube, 77

  \indexspace

  \item {\ptt d} option, 4, 7, 92
  \item {\ptt d2iff}, 56
  \item data types
    \subitem colour, 25
    \subitem colour map, 25
    \subitem complex, 26
    \subitem float, 23
    \subitem vector, 24
  \item debugging, \see{{\ptt z} option}{10}
  \item declarations, 26
    \subitem composite, 27
    \subitem first principles, 26
    \subitem layered textures, 27
    \subitem textures, 27
  \item {\ptt DECLARE} keyword, 26, 70, 76, 83
  \item {\ptt DENTS} keyword, 50
  \item {\ptt DIFFERENCE} keyword, 18, 37
  \item {\ptt DIFFUSE} keyword, 42, 82
  \item {\ptt DIRECTION} keyword, 28, 64, 78, 86
  \item displaying
    \subitem after tracing, 55
    \subitem while tracing, \see{{\ptt d} option}{7}
  \item distortion, 78
  \item dithering, simulated, 86
  \item {\tt DKBOPT} environment variable, 4, 6, 10, 92
  \item dump format, 74
  \item {\ptt dump2i24}, 61
  \item {\ptt dump2mtv}, 61
  \item {\ptt dump2raw}, 62
  \item {\ptt DumpToIFF}, 55
  \item {\ptt DXF2DKB}, 60

  \indexspace

  \item {\ptt e} option, 9
  \item elevation, and {\tt UP} vector, 30
  \item {\ptt END\_SMOOTH_TRIANGLE} keyword, 93
  \item ending scan line, \see{{\ptt e} option}{9}
  \item Enzmann, Alexander, 93, 96
  \item errors, 36, 76
  \item exiting early, \see{{\ptt x} option}{8}

  \indexspace

  \item {\ptt f} option, 4, 6, 55, 56, 62, 74, 92
  \item Farmer, Dan, 60
  \item FBM utilities, 58
  \item Ferguson, George, 97
  \item filename
    \subitem input, \see{{\ptt i} option}{8}
    \subitem output, \see{{\ptt o} option}{8}
  \item float data type, 23
  \item fog, 30
  \item {\ptt FREQUENCY} keyword, 50

  \indexspace

  \item GIF, 1, 5, 51, 57, 62, 72, 76
  \item {\ptt gluetga}, 62
  \item {\ptt GRADIENT} keyword, 48
  \item gradient, \see{image mapping}{51}
  \item {\ptt GRANITE} keyword, 49, 50

  \indexspace

  \item {\ptt h} option, 4, 6, 85
  \item {\ptt halftga}, 62
  \item Hassi, Paul, 62
  \item height of picture, \see{{\ptt h} option}{6}

  \indexspace

  \item {\ptt i} option, 5, 8
  \item IFF, 5, 51, 55, 61, 76
  \item IFF format, 55
  \item {\pem {}Image Alchemy}, 57
  \item image mapping, 51, 71
    \subitem and {\tt ALL}, 51
    \subitem and {\tt ALPHA}, 51
    \subitem and {\tt ONCE}, 51, 83
    \subitem and {\tt TURBULENCE}, 52
    \subitem filenames, 52
    \subitem gradient, 51, 72
    \subitem positioning, 79
    \subitem transformation, 51
  \item {\ptt IMAGE\_MAP} keyword, 62
  \item {\ptt IMAGEMAP} keyword, 51, 83
  \item {\ptt INCLUDE} keyword, 12, 23, 85
  \item input filename, \see{{\ptt i} option}{8}
  \item inside, of objects, 17
  \item {\ptt INTERSECTION} keyword, 17, 37, 77
  \item {\ptt INVERSE} keyword, 19, 38
  \item {\ptt IOR} keyword, 43, 82

  \indexspace

  \item K\^{u}hk\"{u}nen, Jari, 62

  \indexspace

  \item {\ptt l} option, 10, 12, 52, 92
  \item left-handed coordinate system, 24, 30
  \item lemniscate curve, 36
  \item library path, \see{{\ptt l} option}{10}
  \item light
    \subitem reflected vs. filtered, 66
  \item light sources
    \subitem and {\tt TEXTURE}, 13
    \subitem and {\tt TRANSLATE}, 13
    \subitem colour, 40
    \subitem composite objects, 53
    \subitem position restriction, 40, 75
    \subitem translating, 85
  \item {\ptt LIGHT\_SOURCE} keyword, 40
  \item lighting
    \subitem total amount of, 86
  \item {\ptt Lissajou}, 60
  \item {\ptt LOCATION} keyword, 28, 64, 76
  \item {\ptt LOOK\_AT} keyword, 29, 65, 76

  \indexspace

  \item mapping, images, \see{image mapping}{71}
  \item {\ptt MARBLE} keyword, 41
  \item memory usage, 76
  \item {\ptt METALLIC} keyword, 45, 50, 93

  \indexspace

  \item noise, 19, 68

  \indexspace

  \item {\ptt o} option, 5, 8, 55
  \item {\ptt OBJECT} keyword, 38, 85
  \item objects, 38
    \subitem as light sources, \see{light sources}{40}
    \subitem colour of, 39
    \subitem composite, 52
      \subsubitem as light sources, 53
    \subitem inside and outside, 17
    \subitem surface lighting
      \subsubitem ambient, 42
      \subsubitem brilliance, 42
      \subsubitem diffuse, 42
      \subsubitem index of refraction, 43
      \subsubitem metallic, 45
      \subsubitem phong highlighting, 44
      \subsubitem reflection, 43
      \subsubitem refraction, 43
      \subsubitem refraction, and {\tt ALPHA}, 43
      \subsubitem refraction, and {\tt IOR}, 43
      \subsubitem specular reflection, 44
    \subitem transforming, 39
      \subsubitem and textures, 39
  \item {\ptt ONCE} keyword, 51, 83
  \item options,command-line, \see{command-line options}{6}
  \item output
    \subitem buffer size, \see{{\ptt b} option}{9}
    \subitem file generation, \see{{\ptt f} option}{6}
    \subitem filename, \see{{\ptt o} option}{8}
    \subitem formats, 72
  \item outside, of objects, 17

  \indexspace

  \item {\ptt p} option, 5, 7
  \item parallel projection, 71
  \item perturbation, 19
  \item {\ptt PHASE} keyword, 50
  \item {\ptt PHONG} keyword, 43, 44, 82
  \item {\ptt PHONGSIZE} keyword, 44, 82
  \item Pickover, Clifford, 60
  \item {\ptt PICLAB}, 56
  \item pinhole camera model, 30
  \item piriform curve, 37
  \item {\ptt PLANE} keyword, 33, 91
  \item plane
    \subitem inside and outside, 38, 78
    \subitem primitive, 33
  \item porting, 88
  \item post-processing, 55
  \item processing speed, 35, 54, 66, 78, 91
  \item projection, \see{parallel projection}{71}
  \item Puhl, Larry, 60

  \indexspace

  \item {\ptt q} option, 10, 39, 49, 80, 83
  \item {\ptt QUADRIC} keyword, 85
  \item quadric surfaces
    \subitem definition of, 31
    \subitem transformation of, 32
  \item quadrics
    \subitem scaling, 77
  \item quality, \see{{\ptt q} option}{10}
  \item {\ptt QUARTIC} keyword, 35
  \item quartic surfaces, 34, 93
    \subitem first principles declaration of, 35
    \subitem math errors in, 36
  \item questions, 75

  \indexspace

  \item radiosity, 75
  \item Rasmussen, Helge E., 61
  \item raw format, 74
  \item RayScene, 62
  \item {\ptt REFLECTION} keyword, 43, 82
  \item {\ptt REFRACTION} keyword, 43, 67, 79, 82, 83
  \item refraction, 66
  \item rendering quality, \see{{\ptt q} option}{10}
  \item RenderMan, 75
  \item {\ptt RIGHT} keyword, 28, 64, 86
  \item right-handed coordinate system, 30
  \item {\ptt RIPPLES} keyword, 49, 86
  \item {\ptt ROTATE} keyword, 24, 39
  \item {\ptt ROUGHNESS} keyword, 44, 82

  \indexspace

  \item {\ptt s} option, 9
  \item {\ptt SA2DKB}, 59
  \item Saari, Ville, 61
  \item {\ptt SCALE} keyword, 15, 24, 39
  \item scaling qaudrics, 77
  \item scan line
    \subitem ending, \see{{\ptt e} option}{9}
    \subitem starting, \see{{\ptt s} option}{9}
  \item von Seggern, David, 37
  \item shadows, 78
  \item {\tt shapes.dat} definitions file, 31
  \item {\ptt ShellGen}, 60
  \item Skinner, Robert, 2, 90
  \item {\ptt SKY} keyword, 29, 65
  \item {\ptt SMOOTH\_TRIANGLE} keyword, 34
  \item {\ptt SPECULAR} keyword, 43, 44, 82
  \item specular highlight, 14
  \item speed, \see{processing speed}{35}
  \item {\ptt SPHERE} keyword, 32, 54, 91
  \item sphere
    \subitem primitive, 32
    \subitem scaling of, 33
  \item {\ptt SPOTTED} keyword, 47, 50
  \item starting scan line, \see{{\ptt s} option}{9}
  \item surfaces
    \subitem quartic, 34

  \indexspace

  \item Targa, 4, 56, 57, 61, 62
  \item Targe format, 74
  \item tele-photo lens effect, 86
  \item {\ptt TEXTURE} keyword, 41, 80, 82
  \item textures, 41, 67
    \subitem and {\tt REFRACTION}, 70
    \subitem and {\tt TURBULENCE}, 19
    \subitem animation of, 19, 85, 86
    \subitem coloring
      \subsubitem candy cane, 48
      \subsubitem checkered, 45
      \subsubitem gradient, 48
      \subsubitem granite, 49
      \subsubitem marbled, 41, 48
      \subsubitem noisy, 46
      \subsubitem spotted, 47
      \subsubitem wood, 41
    \subitem colour maps, 20, 41, 46
    \subitem combining, 42
    \subitem famous cloud, 21
    \subitem image mapped, \see{image mapping}{51}
    \subitem layered, 21, 42, 66, 70, 82, 92
    \subitem marbled, 68
    \subitem noisy, 69
    \subitem parsing of, 70
    \subitem perturbation
      \subsubitem {\tt COLOUR} ignored in, 49
      \subsubitem bumpy, 50
      \subsubitem dented, 50
      \subsubitem frequency, 50
      \subsubitem phase, 50
      \subsubitem rippled, 49
      \subsubitem wavy, 49
      \subsubitem wrinkled, 50
    \subitem randomness, 41, 80, 86
    \subitem transforming, 41, 86
    \subitem turbulence, 41
    \subitem types, 41
    \subitem wood, 68
  \item {\ptt tga2dump}, 62
  \item {\pem {}The Art Department}, 55
  \item torus, 36
    \subitem scaling, 36
  \item {\tt trace.def} startup file, 4, 6, 10, 85, 92
  \item transformations, 24
    \subitem order of, 32
  \item {\ptt TRANSLATE} keyword, 24, 39
  \item {\ptt TRIANGLE} keyword, 33, 91
  \item triangle
    \subitem and CSG, 34
    \subitem inside and outside, 17
    \subitem primitive, 33
    \subitem smooth shaded primitive, 34
    \subitem surface normals, 34, 79
  \item triangles
    \subitem math errors in, 77
  \item {\ptt TURBULENCE} keyword, 41, 48, 52
  \item turbulence, 19, 69
  \item {\ptt Twister}, 60

  \indexspace

  \item {\ptt UNION} keyword, 17, 37, 76
  \item {\ptt UP} keyword, 28, 64
  \item utilities, 59

  \indexspace

  \item {\ptt v} option, 4, 6
  \item vector data type, 24
  \item verbosity, \see{{\ptt v} option}{6}
  \item {\ptt VIEWPOINT} keyword, 86
  \item viewpoint, 28, 64
    \subitem aspect ratio, 86
    \subitem direction, 28
    \subitem first principles declaration of, 28
    \subitem location, 28
    \subitem right, 28
    \subitem transformation of, 29
    \subitem up, 28

  \indexspace

  \item {\ptt w} option, 4, 6, 85
  \item wait for prompt, \see{{\ptt p} option}{7}
  \item {\ptt WAVES} keyword, 49
  \item Wells, Drew, 60
  \item width of picture, \see{{\ptt w} option}{6}
  \item {\ptt WOOD} keyword, 41, 78, 86
  \item {\ptt WRINKLES} keyword, 50

  \indexspace

  \item {\ptt x} option, 5, 8

  \indexspace

  \item {\ptt z} option, 10

\end{theindex}


\end{document}
