\chapter{Common Questions and Answers}

\index{questions}
I often get asked the same questions again and again.  I usually take this to
mean that the documentation is not complete or not sufficiently clear.  In
order to correct this problem, I've added some sections to the document
describing the features in more detail.  I've also collected some of the more
popular questions and answered them here.

\begin{itemize}
\item[Q:] Will you be implementing radiosity\index{radiosity}?
\item[A:] I don't expect so.  The techniques for radiosity are quite involved
and time consuming (although they are getting faster).  The amount of
effort required to implement radiosity is beyond my current plans.

\item[Q:] Do you intend make DKBTrace RenderMan\index{RenderMan} compatible?
\item[A:] Probably not.  RenderMan is a specification that requires much more
functionality than DKBTrace currently provides.  The camera models,
modelling primitives, and shading language of RenderMan are all very
involved and difficult to implement.  As well, RenderMan is not well
suited to a raytracing approach.  Don't expect to see DKBTrace
RenderMan compatible in the near (or distant) future.  Note, though
that several of the DKBTrace textures are similar to those obtainable
by RenderMan.

\item[Q:] I\index{light sources!position restriction} defined a light
source but the shadows and lighting are all wrong.
\item[A:] Light sources must be defined at the origin (0,0,0) and translated to
the proper place.  The reason for this is to allow the diffuse lighting
calculations to quickly determine where the center of the light source
is.  It's a very difficult task to calculate the center of an object
(in general), so I simply take the place that the object was translated
to as the center of the light source.

\item[Q:] I keep running out of memory\index{memory usage}.  What can I do?
\item[A:] Buy more memory.  But seriously, you can decrease the memory
requirements for any given picture in several ways:
\begin{enumerate}
\item {\tt DECLARE}\keyindex{DECLARE} texture constants and use them
(textures are shared).
\item Don't modify the texture that you are sharing.  On the first
modify, the texture is copied and (therefore) takes up more space.
\item Put the object transformations before the texture structure.  This
prevents the texture from being transformed (and hence, copied.
This may not always be desirable, though).
\item Use {\tt UNION}s\keyindex{UNION} instead of
{\tt COMPOSITE}\keyindex{COMPOSITE} objects to put
pieces together. Previous versions of the raytracer didn't allow
this because the texture applied to the entire object.  Version 2.10
and up allow you to change the texture on a per-shape basis.
\item Use fewer or smaller image maps.
\item Use GIF\index{GIF} or IFF\index{IFF} (non-HAM) images for image
maps.  These are stored internally as 8 bits per pixel with a colour
table instead of 24 bits per pixel.
\end{enumerate}

\item[Q:] I get a floating point exception error\index{errors} on
certain pictures. What's wrong?
\item[A:] Oh no! Not another one!  The raytracer (obviously) performs
{\em many} floating point operations when tracing a scene.  If I had to check
each one for overflow or underflow, the program would be much longer
and I would be much closer to going insane trying to locate all
possible cases.  If you get this problem, I'd suggest that you
first look through your data file to make sure you're not doing
something stupid like:
\begin{itemize}
\item Scaling something by 0.0 in {\em any} dimension;
\item Making the {\tt LOOK_AT}\keyindex{LOOK_AT} point the same as
the {\tt LOCATION}\keyindex{LOCATION};
\item Defining triangles\index{triangles!math errors in} with two
points the same (or nearly the same);
\item Using the zero vector for normals.
\end{itemize}
If it doesn't seem to be one of these problems, let me know, but there
may not be a lot I can do because overflows can occur almost anywhere.
Sorry.  If you do have such troubles, you can try to isolate the
problem in the input data file by commenting out objects or groups
of objects until you narrow it down to a particular section that
fails.  Then try commenting out the individual characteristics of the
offending object.

\item[Q:] No matter how much I scale a Cylinder, I can't make it fit on the
screen.  How big is it and how much do I have to
scale\index{scaling qaudrics}\index{quadrics!scaling} it?
\item[A:] Cylinders (like most quadrics) are infinitely long.  No matter how
much you scale them, they still won't fit on the screen.  To make a
capped cylinder, you must use CSG:
\begin{verbatim}
     INTERSECTION
         QUADRIC Cylinder_Y END_QUADRIC
         PLANE <0.0 1.0 0.0> 1.0 END_PLANE
         PLANE <0.0 -1.0 0.0> 1.0 END_PLANE
     END_INTERSECTION
\end{verbatim}
Cylinders {\em can} be scaled in cross-section, the two vectors
{\em not} in the name of the cylinder (X and Z, in our example above) can
be scaled to control the width and thickness of the cylinder.  Scaling
the Y value (which is normally infinite) is meaningless, unless you
have ``capped'' it as above, then scaling the entire
{\tt INTERSECTION}\keyindex{INTERSECTION} object in the Y dimension
will control the height of the cylinder.

\item[Q:] Why don't you define a primitive\index{cube}\index{box} for a
6-sided box?
\item[A:] Because you can do it so easily with
CSG:\index{constructive solid geometry!definition of box}
\begin{verbatim}
     INTERSECTION
        PLANE < 1.0  0.0  0.0> 1.0 END_PLANE
        PLANE <-1.0  0.0  0.0> 1.0 END_PLANE
        PLANE < 0.0  1.0  0.0> 1.0 END_PLANE
        PLANE < 0.0 -1.0  0.0> 1.0 END_PLANE
        PLANE < 0.0  0.0  1.0> 1.0 END_PLANE
        PLANE < 0.0  0.0 -1.0> 1.0 END_PLANE
     END_INTERSECTION
\end{verbatim}

\item[Q:] Are planes 2D objects or are they 3D but infinitely thin?
\item[A:] Neither.  Planes are 3D objects that divide the world into two
half-spaces.\index{plane!inside and outside}
The space in the direction of the surface normal is considered outside
and the other space is inside (see Figure \ref{plane}).
\begin{figure}[htbp]
\begin{centering}
\setlength{\unitlength}{0.0125in}%
\begin{picture}(125,92)(140,720)
\thinlines
\put(160,740){\vector( 0, 1){ 60}}
\put(140,740){\line( 1, 0){120}}
\put(160,805){\makebox(0,0)[b]{\raisebox{0pt}[0pt][0pt]{\elvrm Normal}}}
\put(265,735){\makebox(0,0)[lb]{\raisebox{0pt}[0pt][0pt]{\elvrm Plane}}}
\put(180,720){\makebox(0,0)[lb]{\raisebox{0pt}[0pt][0pt]{\elvrm Inside}}}
\put(190,750){\makebox(0,0)[lb]{\raisebox{0pt}[0pt][0pt]{\elvrm Outside}}}
\end{picture}

\caption{The ``sides'' of a plane}
\label{plane}
\end{centering}
\end{figure}
In other words, planes are 3D objects that are infinitely thick.

\item[Q:] Can DKBTrace render soft shadows\index{shadows}?
\item[A:] No.  This would require a lot more programming work and a lot more
calculation time.  You can place an unturbulated {\tt WOOD}\keyindex{WOOD}
texture ``filter'' over a lamp in a can, and use a colour map going from dark
({\tt ALPHA}\keyindex{ALPHA} 0.0) around the edges to clear
({\tt ALPHA} 1.0) at the center, and get a soft-shadow-like effect for
a ``spotlight''.

\item[Q:] I'd like to go through the program and hand-optimize the
assembly code in places to make it faster\index{processing speed}.
What should I optimize?
\item[A:] Don't bother.  With hand optimization, you'd spend a lot of
time to get perhaps a 5-10% speed improvement at the cost of total
loss of portability.  If you use a better ray-surface intersection
algorithm, you should be able to get an order of magnitude or more
improvement.  Check out some books and papers on raytracing for useful
techniques.  Specifically, check out ``Spatial Subdivision'' and ``Ray
Coherence'' techniques.

\item[Q:] Objects on the edges of the screen seem to be
distorted.\index{distortion}  Why?
\item[A:] If the {\tt DIRECTION}\keyindex{DIRECTION} vector of the
viewpoint is not very
long, you may get distortion at the edges of the screen.  The reason
for this is that the viewpoint's screen is flat, as illustrated in
Figure \ref{distort}.
\begin{figure}[htbp]
\begin{centering}
\setlength{\unitlength}{0.0125in}%
\begin{picture}(110,135)(90,625)
\thinlines
\put(180,730){\framebox(20,20){}}
\put(140,670){\vector( 2, 3){ 40}}
\put(140,670){\vector( 1, 2){ 40}}
\put(140,670){\vector( 4, 1){ 40}}
\put(140,670){\vector( 4,-1){ 40}}
\put(180,660){\framebox(20,20){}}
\put(160,760){\line( 0,-1){120}}
\put(190,665){\makebox(0,0)[b]{\raisebox{0pt}[0pt][0pt]{\elvrm B}}}
\put(190,735){\makebox(0,0)[b]{\raisebox{0pt}[0pt][0pt]{\elvrm A}}}
\put(135,665){\makebox(0,0)[rb]{\raisebox{0pt}[0pt][0pt]{\elvrm Viewpoint}}}
\put(160,625){\makebox(0,0)[b]{\raisebox{0pt}[0pt][0pt]{\elvrm Viewing screen}}}
\end{picture}

\caption{Distortion with short {\tt DIRECTION} vector}
\label{distort}
\end{centering}
\end{figure}
The object labelled A appears smaller than the object labelled B,
despite their actually being the size size.

\item[Q:] How do you position\index{image mapping!positioning}
image maps without a lot of trial and error?
\item[A:] By default, images will be mapped onto the range 0,0 to 1,1 in the
appropriate plane.  You should be able to translate, rotate, and
scale the image from there.

\item[Q:] What's the difference between {\tt ALPHA}\keyindex{ALPHA}
and {\tt REFRACTION}\keyindex{REFRACTION}?
\item[A:] The difference is a bit subtle.  Alpha is a component of
a colour that determines how much light can pass through that colour.
Refraction is a property of a surface that determines how much light
can come from inside the surface.  See the section above on
Transparency and Refraction for more details.

\item[Q:] How do you calculate the surface
normals\index{triangle!surface normals} for smooth triangles?
\item[A:] When I implemented smooth triangles, I never really expected
anyone to manually calculate the surface normals.  There are two ways
of getting another program to calculate them for you:
\begin{enumerate}
\item Depending on the type of input to the program, you may be able to
calculate the surface normals directly.  For example, if you have
a program that converts B-Spline or Bezier Spline surfaces into
DKB-format files, you can calculate the surface normals from the
surface equations.
\item If your original data was a polygon or triangle mesh, then it's
not quite so simple.  You have to first calculate the surface
normals of all the triangles.  This is easy to do -- you just use
the vector cross-product of two sides (make sure you get the
vectors in the right order).  Then, for every vertex, you average
the surface normals of the triangles that meet at that vertex.
These are the normals you use for smooth triangles.
\end{enumerate}

\item[Q:] When I render parts of a picture on different systems, the
textures\index{textures!randomness} don't match when I put them
together.  Why?
\item[A:] The appearance of a texture depends on the particular random number
generator used on your system.  DKBTrace seeds the random number
generator with a fixed value when it starts, so the textures will be
consistent from one run to another or from one frame to another so
long as you use the same executables.  Once you change executables,
you will likely change the random number generator and, hence, the
appearance of the texture.  There is an example of a standard ANSI
random number generator provided in {\tt IBM.C}, include it in your
machine-specific code if you are having consistency problems.

\item[Q:] What's the difference\index{colour!and {\tt TEXTURE}}
between a {\tt COLOUR}\keyindex{COLOUR}
declared inside a {\tt TEXTURE}\keyindex{TEXTURE} and one that's in
a shape or an object and not in a texture?
\item[A:] The colour in the texture specifies the colour to use for qualities
5 and up.  The colour on the shape and object are used for faster
rendering in qualities 4 and lower and for the colour of light sources.
See the {\tt -q}\optindex{q} option for details on the quality parameter.

\item[Q:] I created an object that passes through its bounding
volume\index{bounding shapes}.
At times, I can see the parts of the object that are outside the
bounding volume.  Why does this happen?
\item[A:] Bounding volumes are {\em not} designed to change the
shape of the object.  They are strictly a realtime improvement
feature.  The raytracer trusts you when you say that the object is
enclosed by a bounding volume.  The way it uses bounding volumes is
very simple: If the ray hits the bounding volume (or the ray's origin
is inside the bounding volume), then the object is tested against that
ray.  Otherwise, we ignore the object.  If the object extends beyond
the bounding volume, anything goes.  The results are undefined.  It's
quite possible that you could see the object outside the bounding
volume and it's also possible that it could be invisible.  It all
depends on the geometry of the scene.

\item[Q:] Will you be writing a Graphical User Interface for DKB?
\item[A:] No, but several other people have expressed an interest in writing
one.  I'd like to form a mailing list to get all these people in touch
with each other, so if you have an interest in this, please let me
know.

\item[Q:] When will the next version be available?
\item[A:] If I told you, I'd be lying because I don't know.  I'm finding that
releasing an official version of a program like this is a major
effort.  It requires not only the changes to the code, but also
changes to the documentation.  Sometimes (as with this release), I
have to change the older data files to conform to the new syntax.
Usually, I have to spend a lot of time re-rendering scenes that used
to work to make sure they still do.  That combined with sending the
files to beta testers, getting feedback, making fixes, and re-issuing
the changes adds up to a lot of work.  I don't expect I'll be doing
it terribly often.

Bottom line -- If I say ``next week'', don't believe me.  I'm probably
wrong.
\end{itemize}
