\chapter{Displaying the Images}

\index{displaying!after tracing}\index{post-processing}
When the ray tracer draws the picture on the screen, it doesn't make good
choices for the colour registers.  Without knowing all the needed colours
ahead of time, only approximate guesses can be made.  Usually, a post-
processor is really needed to view the final image correctly.

\section{Amiga Systems}

A post-processor has been provided for the Amiga which scans the picture and
chooses the best possible colour registers.  It then redisplays the picture.
For the Amiga, {\tt DumpToIFF}\ttindex{DumpToIFF}\index{IFF format}
can optionally save this picture in IFF\index{IFF} format. The syntax for the
{\tt DumpToIFF} post-processor is:
\begin{verbatim}
     DumpToIFF filename
\end{verbatim}
where the filename is the one given in the {\tt -o}\optindex{o}
parameter of the ray tracer.  If you didn't specify the {\tt -o}
option, then use:
\begin{verbatim}
     DumpToIFF data.dis
\end{verbatim}
If you want to save to an IFF file, then put the name of the IFF file after
the name of the data file:
\begin{verbatim}
     DumpToIFF data.dis picture
\end{verbatim}
This will create a file called {\tt picture} which contains the IFF image.

An alternative approach is to buy the commercial package called
{\em The Art Department}\emindex{The Art Department} from ASDG.
You can then use the {\tt +fr}\optindex{f} option
of the raytracer to produce raw files which can be read in to {\em TAD}
using Sculpt mode.  You can also render using {\tt +fd} to produce a
dump format file, and use {\tt d2iff}\ttindex{d2iff} to convert this
to a 24-bit IFF image to load into {\em TAD}.

\section{IBM Systems}

For the IBM, you will probably want to use the {\tt +ft}\optindex{f}
option (default if {\tt +f} is given) and write the image out in
Targa-24\index{Targa}
format.  If you have a Targa or compatible display adapter, you may
view the picture in the full 16 million colors (that's why they still
cost the big bucks, but Hercules and Everex, notably, are introducing
their lower-priced SVGA-compatible 24-bit color display systems for
the IBM PC and compatibles).  If you don't have one of these, there
are several different post-processing programs available to convert
the TARGA true-color image into a more suitable color-mapped image
(such as {\tt .GIF}).  If you have a VGA/MCGA or SVGA
adapter capable
of 320x200 by 256 colors or better, then you may use the +d option
which will display the image as it generates using only approximate
screen colors.  The {\tt +d} option will Autodetect the type of
display adapter card you have and briefly say what kind it found
before displaying the picture.  If you say {\tt +d{\em x}} where
{\em x} is one of the predefined IBM (S)VGA display adapter types, no
hardware test is performed, so if you don't have that type of (S)VGA
card, {\em don't} use that particular {\tt +d{\em x}} option!

When displaying the image to screen, a HSV conversion method is used (hue,
saturation, value).  This is a convenient way of translating colors from a
``true color'' format (16 million) down a ``colour mapped'' format of something
reasonable (like 256), while still approximating the color as closely as the
available display hardware permits.  As mentioned previously, the tracer has
no way of knowing which colors will be finally used in the image, nor can it
deal properly with all of the colors which will be generated (up to 16M), so
only 4 shades each of 64 possible hues are mapped into the palette DAC, as
well as black, white, and two grey levels. The advantage a post-processor has
in choosing mapped colors is that it can throw away all the unused colors in
the palette map, and thereby free up some space for making better gradient
shades of the colors that are actually used.

There are several available image processing programs that can do
this, a public domain one that is very good is
{\tt PICLAB}\ttindex{PICLAB}, by the
Stone Soup Group (the folks who brought you {\tt FRACTINT}).  The
procedure is to load the TARGA file, then use the {\tt MAKEPAL}
command to generate a 256 color map which is the histogram-weighted
average of the most-used colors in the image (You could also
{\tt PLOAD} a palette file from {\tt FRACTINT} or one you previously had
saved using {\tt PSAVE}).  You then {\tt MAP} the palette onto the
image one of two ways:
\begin{enumerate}
\item If the {\tt DITHER} variable is {\tt OFF}, a nearest-match
color-fit is used, which can sometimes produce unwanted ``banding'' of
colored regions (called false contours).
\item If the {\tt DITHER} variable is {\tt ON}, then a standard dither
is used to determine final color values.  This is much better at
blending the color bands, but can produce noise in reflections and
make mirrors appear dirty or imperfect.
\end{enumerate}

Then you would typically {\tt SHOW} the image or {\tt GSAVE} it into
GIF format.  While the picture is still in the unmapped form (TARGA,
etc.) you can perform a variety of advanced image processing
transformations and conversions, so save the {\tt .TGA} or {\tt .RAW}
files you make (in case you ever get a TARGA card, or give them to a
friend who has one!).

A commercial product that also does a good job of nearest-match
color-fit is the {\tt CONVERT} utility of
{\em AutoDesk Animator}\emindex{Autodesk Animator}.
However, the dither effect is not as good as that of {\tt PICLAB}.  To
convert the file in {\em AA}'s {\tt CONVERT}, you {\tt LOAD} TARGA, then in
the {\tt CONVERT} menu, go to the {\tt SCALE} function and just hit
{\tt RENDER}.  Click on the {\tt DITHER} (lights up with an asterisk
when on) if you want it to use it's dithering.  {\tt CONVERT} will
scale (if asked to) and then do a histogram of total used colors like
{\tt PICLAB}, but then makes 7 passes on the color map and image to
determine shading thresholds of each hue.  This nearly eliminates the
color banding (false contours) in a lot of cases without resorting to
good 'ol dithering.  By now you must get the feeling {\tt DITHER} is a
4-letter word.  If you have a low-resolution display, it is.  If you
have too few colors, however, it can be a saving grace.  At
resolutions of 640x400 or higher the ``spray paint'' effect of
dithering and anti-aliasing is much less noticeable, and effects a
much smoother blending appearance.

A new package to show up in the public domain/shareware circles for the IBM
is something called {\em Image Alchemy}\emindex{Image Alchemy}, by
Handmade Software.  It will convert
Targa\index{Targa} format to GIF\index{GIF} files and do a decent
job of palette selection and dithering.  To use it simply say
\begin{verbatim}
     ALCHEMY file.tga file.gif -g -8 -c256
\end{verbatim}
It also features a quick-and-dirty display mode where it uses a
standardized palette in much the same way DKB's {\tt +d} option does,
but offers dithering of the image while using the pre-defined palette,
for a somewhat better quick display.

\section{Unix Systems}

I don't have many details on Unix systems, but I hear that the
FBM\index{FBM utilities} utilities
work well to convert the Dump format files into various formats of images.
For people with access to anonymous FTP over USENET, the FBM utilities are
available from {\tt nl.cs.emu.edu} (128.2.222.56) in directory
{\tt /usr/mlm/ftp}.
