\chapter{The Scene Description Language}

The Scene Description Language allows the user to describe the world in a
readable and convenient way. This chapter is effectively a reference
manual for DKBTrace.

The language delimits comments\index{comments} by the left and right
braces (``{\tt \{}'' and ``{\tt \}}''). Nested comments are allowed,
but no sane person uses them anyway, right?
The language allows include files to be specified by placing the line:
\keyindex{INCLUDE}
\begin{verbatim}
     INCLUDE "filename"
\end{verbatim}
at any point in the input file (include files may be nested).  The filename
must be enclosed in double quotes and may be up to 40 characters long,
including the two double-quote ({\tt "}) characters.  You may have at most 10
{\tt INCLUDE}'d files per scene trace, whether nested or not.

\section{The Basic Data Types}

There are several basic types of data.

\subsection{Floats}

\index{float data type}\index{data types!float}
Floats are represented by an optional sign ({\tt -}), some digits, an optional
decimal point, and more digits.  Version 2.10 now supports the
``{\tt e}'' notation for exponents.  The following are valid floats:
\begin{verbatim}
     1.0  -2.0  -4  34  3.4e6 2e-5
\end{verbatim}

\subsection{Vectors}

\index{vector data type}\index{data types!vector}
Vectors are arrays of three floats.  They are bracketed by angle brackets
(``{\tt <}'' and ``{\tt >}''), and the three terms usually represent x, y,
and z. For example:
\begin{verbatim}
     < 1.0  3.2  -5.4578 >
\end{verbatim}
Vectors typically refer to relative points.  For example, the vector:
\begin{verbatim}
     <1 2 3>
\end{verbatim}
means the point that's 1 unit to the right, 2 units up, and 3 units in front.
``Of what?'' you ask?  Well, usually it's relative to the center of the
``universe'' at {\tt <0 0 0>}.

\subsection{Transformations -- {\tt TRANSLATE}, {\tt ROTATE}, and {\tt SCALE}}

\index{transformations}\keyindex{TRANSLATE}\keyindex{ROTATE}\keyindex{SCALE}
In a few places here and there, vectors are used as a convenient notation but
don't represent a point in space. This is the case for the transformations
{\tt TRANSLATE}, {\tt ROTATE}, and {\tt SCALE}:

\begin{description}
\item[{\tt TRANSLATE <{\em x y z\/}>}] Move the object {\em x} units
to the right, {\em y} units up, and {\em z} units away from us.
\item[{\tt SCALE <{\em xs ys zs\/}>}] Scale the object by {\em xs}
units in the left/right direction, {\em ys} units in the up/down
direction and {\em zs} units in the front/back direction.
\item[{\tt ROTATE <{\em xr yr zr\/}>}] Rotate the object {\em xr}
degrees about the X axis, then {\em yr} degrees about the Y axis,
then {\em zr} degres about the Z axis (note that the order matters).
\end{description}

To work out the rotation directions, you must perform the famous ``Computer
Graphics Aerobics'' exercise.  Hold up your left hand.
\index{left-handed coordinate system}\index{coordinate system!left-handed}
point your thumb in the
positive direction of the axis you want to rotate about.  Your fingers will
curl in the positive direction of rotation.  This is the famous ``left-hand
coordinate system'', illustrated in Figure \ref{lefthand}.
\begin{figure}[htbp]
\begin{centering}
\setlength{\unitlength}{0.0125in}%
\begin{picture}(60,110)(180,610)
\thinlines
\put(230,700){\line( 0, 1){ 10}}
\put(230,710){\line( 1, 0){ 10}}
\put(240,710){\vector( 0,-1){ 10}}
\put(210,620){\line( 0,-1){ 10}}
\put(185,620){\line( 0,-1){ 10}}
\multiput(220,630)(-0.40000,-0.40000){26}{\makebox(0.1111,0.7778){\fivrm .}}
\multiput(180,625)(0.41667,-0.41667){13}{\makebox(0.1111,0.7778){\fivrm .}}
\put(180,625){\line( 0, 1){ 75}}
\put(180,700){\line( 1, 0){ 10}}
\put(190,700){\line( 0, 1){ 15}}
\put(190,715){\line( 1, 0){ 10}}
\put(200,715){\line( 0, 1){  5}}
\put(200,720){\line( 1, 0){ 10}}
\put(210,720){\line( 0,-1){ 15}}
\put(210,705){\line( 1, 0){ 10}}
\put(220,705){\line( 0,-1){ 60}}
\multiput(220,645)(0.41667,-0.41667){13}{\makebox(0.1111,0.7778){\fivrm .}}
\put(225,640){\line( 1, 0){ 15}}
\put(240,640){\line( 0,-1){ 10}}
\put(240,630){\line(-1, 0){ 20}}
\put(210,705){\line( 0,-1){ 55}}
\put(200,715){\line( 0,-1){ 65}}
\put(190,700){\line( 0,-1){ 50}}
\end{picture}

\caption{Left-hand Coordinate System}
\label{lefthand}
\end{centering}
\end{figure}
If you want to use a right-hand system, as some CAD
systems do, the {\tt RIGHT} vector in the {\tt VIEW_POINT} needs to
be changed.  See the
detailed description of the {\tt VIEW_POINT}, and use your right hand for the
``Aerobics''. 

\subsection{Colours}

\index{colour!data type}\index{data types!colour}
A colour consists of a red component, a green component, a blue
component, and possibly an alpha component.  All four components are
floats in the range 0.0 to 1.0.  The syntax for Colours is the word
{\tt COLOUR} followed by any or all of the {\tt RED}, {\tt GREEN},
{\tt BLUE} or {\tt ALPHA} components in any order. For example:
\begin{verbatim}
     COLOUR  RED 1.0  GREEN 1.0  BLUE 1.0
     COLOUR  BLUE 0.56
     COLOUR  GREEN 0.45 RED 0.3 BLUE 0.1 ALPHA 0.3
\end{verbatim}
{\tt ALPHA} is a transparency indicator.  If an object's colour
contains some transparency, then you can see through it.  If
{\tt ALPHA} is 0.0, the object is totally opaque.  If it is 1.0, it is
totally transparent.

For those people who spell ``Colour'' the American way as ``Color'',
the program will also accept {\tt COLOR}
\index{colour!and color}
as equivalent to {\tt COLOUR} in all instances.

\subsection{Colour Maps}

\index{colour map data type}\index{data types!colour map}
For wood, marble, spotted, agate, granite, and gradient texturing, the user
may specify arbitrary colours to use for the texture.  This is done by a
colour map or ``colour spline''.  When the object is being textured, a number
between 0.0 and 1.0 is generated which is then used to form the colour of the
point.  A Colour map specifies the mapping used to change these numbers into
colours. The syntax is as follows:
\begin{verse}
{\tt COLOUR_MAP} \\
{\tt \ \ \ \ \ [{\em start-value end-value colour1 colour2\/}]} \\
{\tt \ \ \ \ \ [{\em start-value end-value colour1 colour2\/}]} \\
\ \ \ \ \ ... \\
{\tt END_COLOUR_MAP}
\end{verse}
The numeric value between 0.0 and 1.0 is located in the colour map and the
final colour is calculated by a linear interpolation (a smooth blending)
between the two colours in the located range.

\section{More Complex Data Types -- {\tt DECLARE}}

\keyindex{DECLARE}\index{declarations}\index{data types!complex}
The data types used to describe the objects in the world are a bit more
difficult to describe.  To make this task easier, the program allows users to
describe these types in two ways.  The first way is to define it from first
principles specifying all of the required parameters.  The second way allows
the user to define an object as a modification of another object (the other
object is usually defined from first principles but is much simpler).  Here's
how it works.

You can use the term {\tt DECLARE} to declare a type of object with a
certain description.\index{declarations!first principles}
The object is not included in the world but is
made known to the program that it can be used as a ``prototype'' for
defining other objects by the name given, as this basic example shows:
\begin{verbatim}
     DECLARE Sphere = QUADRIC  { First principles definition }
          <1.0 1.0 1.0>        {   of a sphere               }
          <0.0 0.0 0.0>
          <0.0 0.0 0.0>
          -1.0
     END_QUADRIC
\end{verbatim}
To then reference the declaration elsewhere in your source file or in another
included one, and to actually include the object in the world, you would
define the object using the {\tt DECLARE}'d object's name, like this:
\begin{verbatim}
     OBJECT
          QUADRIC Sphere
               SCALE <20.0 20.0 20.0>
          END_QUADRIC
          TEXTURE
             COLOUR White
             AMBIENT 0.2
             DIFFUSE 0.8
          END_TEXTURE
     END_OBJECT
\end{verbatim}

The real power of declaration becomes apparent when you declare
several primitive types of objects and then define an object with
several component shapes, using either {\tt COMPOSITE} methods or the
CSG methods {\tt INTERSECTION}, {\tt UNION}, or {\tt DIFFERENCE}.
\index{declarations!composite}
Also, using the {\tt DECLARE} keyword to pre-define textures can make
several objects share a texture without the need for each object to
store a duplicate copy of the same texture, for more efficient memory
usage.\index{declarations!textures}
For example:
\begin{verbatim}
     DECLARE Dull TEXTURE  { A Basic, Boring Texture }
          AMBIENT 0.3
          DIFFUSE 0.7
     END_TEXTURE
     OBJECT         { A Hot dog in a Hamburger Bun (?) }
          UNION
               QUADRIC Sphere
                    SCALE <20.0, 10.0, 20.0>
            TEXTURE Dull END_TEXTURE
               END_QUADRIC
               QUADRIC Cylinder_X
                    SCALE <40.0, 20.0, 20.0>
            TEXTURE Dull END_TEXTURE
               END_QUADRIC
          END_UNION
     END_OBJECT
\end{verbatim}

Layered textures, new to version 2.10, may also be {\tt DECLARE}'d.
The {\tt DECLARE} mechanism keeps looking for successive {\tt TEXTURE}
definitions and will layer them onto the same texture until another
{\tt DECLARE} (or any other statement except another {\tt TEXTURE}) is
encountered in the input data file.\index{declarations!layered textures}
For example:
\begin{verbatim}
     DECLARE Cloud_Wood TEXTURE  { This is the cloudy wood }
       Dark_Wood                 {   texture used earlier. }
       TURBULENCE 0.0
       SCALE <0.2 0.2 0.2>
       PHONG 1.0
     END_TEXTURE
     TEXTURE            { This is layered onto the wood }
       TURBULENCE 0.5   {   texture just defined.       }
       BOZO
       COLOUR_MAP
         { transparent to transparent }
         [0.0 0.6 COLOUR RED 1.0 GREEN 1.0 BLUE 1.0 ALPHA 1.0
                  COLOUR RED 1.0 GREEN 1.0 BLUE 1.0 ALPHA 1.0]
         { transparent to white }
         [0.6 0.8 COLOUR RED 1.0 GREEN 1.0 BLUE 1.0 ALPHA 1.0
                  COLOUR RED 1.0 GREEN 1.0 BLUE 1.0]
         { white to grey }
         [0.8 1.001 COLOUR RED 1.0 GREEN 1.0 BLUE 1.0
                    COLOUR RED 0.8 GREEN 0.8 BLUE 0.8]
       END_COLOUR_MAP
       SCALE <0.4  0.08  0.4>
     END_TEXTURE

    DECLARE (etc.) { Ends the layered definition. }
\end{verbatim}

\section{Viewpoint}

\index{viewpoint}
The viewpoint tells the ray tracer the location and orientation of the
camera.  The viewpoint is described by four vectors --
{\tt LOCATION},\index{viewpoint!location}\keyindex{LOCATION}
{\tt DIRECTION},\index{viewpoint!direction}\keyindex{DIRECTION}
{\tt UP},\index{viewpoint!up}\keyindex{UP} and
{\tt RIGHT}.\index{viewpoint!right}\keyindex{RIGHT}
Location determines where the camera is
located.  Direction determines the direction that the camera is
pointed.  Up determines the ``up'' direction of the camera.  Right
determines the direction to the right of the camera.

A first principle's declaration of a viewpoint would look like this:
\index{viewpoint!first principles declaration of}
\begin{verbatim}
     VIEWPOINT
          LOCATION < 0.0  0.0  0.0>
          DIRECTION < 0.0  0.0  1.0>
          UP < 0.0  1.0  0.0 >
          RIGHT < 1.0  0.0  0.0>
     END_VIEWPOINT
\end{verbatim}
This format becomes cumbersome, however, because the vectors must be hand
calculated.  This is especially difficult when the vectors are not lined up
with the X, Y, and Z axes as they are in the above example. To make it easier
to define the viewpoint, you can define one viewpoint, then modify the
description.\index{viewpoint!transformation of}  For example,
\begin{verbatim}
     VIEWPOINT
          LOCATION < 0.0  0.0  0.0>
          DIRECTION < 0.0  0.0  1.0>
          UP < 0.0  1.0  0.0 >
          RIGHT < 1.0  0.0  0.0 >
          TRANSLATE < 5.0  3.0  4.0 >
          ROTATE < 30.0  60.0  30.0 >
     END_VIEWPOINT
\end{verbatim}
In this example, the viewpoint is created, then translated to another point in
space and rotated by 30 degrees about the X axis, 60 degrees about the Y axis,
and 30 degrees about the Z axis.

Unfortunately, even this is somewhat cumbersome.  So, in version 2.0 and
above, you can now specify two more parameters:

\begin{verse}
{\tt SKY} {\em vector} \\
{\tt LOOK_AT} {\em vector}
\end{verse}

The {\tt SKY}\keyindex{SKY} keyword tells the viewpoint where the sky
is. It tries
to keep the camera's {\tt UP} direction aligned as closely as possible
to the sky. The {\tt LOOK_AT}\keyindex{LOOK_AT} keyword tells the
camera to look at a
specific point.  The camera is rotated as required to point directly
at that point.  By changing the {\tt SKY} vector, you can twist the
camera while still looking at the same point.

One subtle point: the {\tt SKY} direction is not necessarily the same
as the {\tt UP} direction. For example, consider the situation shown
in Figure \ref{skyup}.
\begin{figure}[htbp]
\begin{centering}
\setlength{\unitlength}{0.0125in}%
\begin{picture}(35,112)(210,660)
\thinlines
\put(220,700){\vector( 1, 2){ 15}}
\put(220,700){\vector( 1,-2){ 20}}
\put(220,700){\vector( 0, 1){ 60}}
\put(210,695){\makebox(0,0)[lb]{\raisebox{0pt}[0pt][0pt]{\elvrm C}}}
\put(245,660){\makebox(0,0)[lb]{\raisebox{0pt}[0pt][0pt]{\elvrm O}}}
\put(240,735){\makebox(0,0)[b]{\raisebox{0pt}[0pt][0pt]{\elvrm U}}}
\put(220,765){\makebox(0,0)[b]{\raisebox{0pt}[0pt][0pt]{\elvrm S}}}
\end{picture}

\caption{The Difference between {\tt SKY} and {\tt UP}}
\label{skyup}
\end{centering}
\end{figure}
If you said that the camera C has a {\tt SKY} direction S and is
looking at O, the up direction will not point to the sky.  {\tt UP}'s
like putting an antenna on your camera.  If you point the camera
downwards, the antenna will no longer point straight up.

The {\tt RIGHT} vector, as was mentioned previously, controls the
aspect ratio of the screen display.  It also determines the
``handedness'' of the coordinate system in use.  A normal
(left-handed)\index{left-handed coordinate system}%
\index{coordinate system!left-handed}
 {\tt RIGHT} statement would be:
\begin{verbatim}
     RIGHT < 1.3333 0.0 0.0 >
\end{verbatim}
To use a right-handed coordinate system,\index{right-handed coordinate system}%
\index{coordinate system!right-handed}
as is popular in some CAD
programs and some other ray-tracers, such as Sculpt for the Amiga, use
a {\tt RIGHT} like:
\begin{verbatim}
     RIGHT < -1.3333 0.0 0.0 >
\end{verbatim}
Some CAD systems also have the peculiar conception that the Z axis is
the ``elevation'' and is the {\tt UP} direction instead of the Y axis.
\index{elevation, and {\tt UP} vector}
If this is the case you will want to change your {\tt UP} statement
to:
\begin{verbatim}
     UP < 0.0 0.0 1.0 >
\end{verbatim}
Note that a pinhole camera model\index{pinhole camera model}%
\index{camera model}
is used, so no focus or depth-of-field
effects are supported at this time.  For more detailed information on the
camera model, see the section ``How it All Works''.

\section{Fog}

\index{fog}
Version 2.0 of the raytracer includes the ability to render fog.  To add fog
to a scene, place the following declaration outside of any object definitions:
\begin{verse}
{\tt FOG} \\
{\tt\ \ \ \ \ COLOUR  {\em colour} \{ the fog colour \}}\\
{\tt\ \ \ \ \ 200.0   \{ the fog distance \}} \\
{\tt END_FOG}
\end{verse}

The fog to colour ratio is calculated as
$1 - e^{-depth / distance}$,
so at depth 0, the colour is pure (1.0)
with no fog (0.0).  At the fog distance, you'll get 63% of the colour
from the object's colour and 37% from the fog colour.

\section{Shapes}

Shapes describe the shape of an object without mentioning any surface
characteristics like colour, surface lighting and reflectivity.

\subsection{Quadric Shapes}

The most general shape used by this system is called a Quadric Surface.
Quadric Surfaces can produce shapes like spheres, cones, cylinders,
paraboloids (dish shapes), and hyperboloids (saddle or hourglass
shapes).

The easiest way to define these shapes is to include the standard file
{\tt shapes.dat}\index{shapes.dat@{\tt shapes.dat} definitions file} into
your program and to transform these shapes
(using {\tt TRANSLATE}, {\tt ROTATE}, and {\tt SCALE}) into the ones
you want.  To be complete, however, I will describe the mathematical
principles behind quadric surfaces.  Those who are not interested in
the mathematical details can skip to the next section.

The quadric\index{quadric surfaces!definition of}
\begin{verbatim}
     QUADRIC
         <A B C>
         <D E F>
         <G H I>
         J
     END_QUADRIC
\end{verbatim}
defines a surface in three dimensions which satisfies the following
equation
\begin{displaymath}
{\tt A}x^{2} + {\tt B}y^{2} + {\tt C}z^{2} +
{\tt D}xy + {\tt E}xz + {\tt F}yz + {\tt G}x + {\tt H}y +{\tt I}z +
{\tt J} = 0
\end{displaymath}
Did you really want to know that?  I didn't think so.

Different values of {\tt A}, {\tt B}, {\tt C}, \ldots {\tt J} will
give different shapes.  So, if you take any three dimensional point
and use its x, y, and z coordinates in the above equation, the answer
will be 0 if the point is on the surface of the object.  The answer
will be negative if the point is inside the object and positive if the
point is outside the object.  Here are some examples:
\begin{eqnarray*}
x^{2} + y^{2} + z^{2} - 1 & = & 0 \mbox{\ -- Sphere} \\
x^{2} + y^{2} - 1 & = & 0 \mbox{\ -- Cylinder along Z axis} \\
x^{2} + y^{2} - z^{2} & = & 0 \mbox{\ -- Cone along Z axis}
\end{eqnarray*}

\subsection{Quadric surfaces the easy way}

Now that doesn't sound so hard, does it?  Well, actually, it does.
Only the hard-core graphics fanatic would define his objects using the
{\tt QUADRIC} definition directly.  Even I don't do it that way and I
know how it works.\footnote{Well, at least I worked it out once
or twice.}

Fortunately, there is an easier way.
\index{quadric surfaces!transformation of}
The file {\tt shapes.dat} already
includes the definitions of many quadric surfaces.  They are centered
about the origin {\tt < 0 0 0 >} and have a radius of 1.  To use them,
you can define shapes simply as follows:
\begin{verbatim}
     INCLUDE "colors.dat"
     INCLUDE "shapes.dat"   { important for this example }
     INCLUDE "textures.dat"

     QUADRIC
          Cylinder_X
          SCALE < 50.0  50.0  50.0 >
          ROTATE < 30.0  10.0  45.0 >
          TRANSLATE < 2.0  5.0  6.0 >
     END_QUADRIC
\end{verbatim}
You may have as many transformation lines (scale, rotate, and translate) as
you like in any order.  Usually, however, it's easiest to do a scale first,
one or more rotations, then finally a translation.  Otherwise, the results may
not be what you expect. (The transformations always transform the object about
the origin.  If you have a sphere at the origin and you translate it then
rotate it, the rotation will spin the sphere around the origin like planets
about the sun).\index{transformations!order of}

\subsection{Spheres}

\index{sphere!primitive}\keyindex{SPHERE}
Since spheres are so common in ray traced graphics, a {\tt SPHERE}
primitive has been added to the system.  This primitive will render
much more quickly than the corresponding quadric shape.  The syntax
is:
\begin{verbatim}
     SPHERE  <center> radius END_SPHERE
\end{verbatim}

You can also add translations, rotations, and scaling to the sphere. For
example, the following two objects are identical:
\begin{verbatim}
     OBJECT
          SPHERE  < 0.0 25.0 0.0 > 10.0 END_SPHERE
          TEXTURE
             COLOR Blue
             AMBIENT 0.3
             DIFFUSE 0.7
          END_TEXTURE
     END_OBJECT
\end{verbatim}

\begin{verbatim}
     OBJECT
          SPHERE  < 0.0 0.0 0.0 > 1.0
               TRANSLATE <0.0  25.0  0.0> 
               SCALE <10.0  10.0  10.0>
          END_SPHERE
          TEXTURE
             COLOR Blue
             AMBIENT 0.3
             DIFFUSE 0.7
          END_TEXTURE
     END_OBJECT
\end{verbatim}
Note that Spheres may only be scaled uniformly.\index{sphere!scaling of}
You cannot use:
\begin{verbatim}
     SCALE <10.0 5.0 2.0>
\end{verbatim}
on a sphere.  If you need oblate (flattened) spheroids, use a scaled quadric
``Sphere'' shape instead, as it can be arbitrarily scaled in any dimension.

\subsection{Planes}

\index{plane!primitive}\keyindex{PLANE}
Another primitive provided to speed up the raytracing is the {\tt PLANE}.
This is a flat infinite plane.  To declare a {\tt PLANE}, you
specify the direction of the surface normal to the plane (the {\tt UP}
direction) and the distance from the origin of the plane to the
world's origin. As with spheres, you can translate, rotate, and scale
planes.  Examples:
\begin{verbatim}
     { plane in the X-Z axes 10 units below the origin. }
     PLANE <0.0  1.0  0.0> -10.0 END_PLANE
     { plane in the X-Z axes 10 units above the origin. }
     PLANE <0.0  1.0  0.0>  10.0 END_PLANE
     { plane in the X-Y axes 10 units behind the origin.}
     PLANE <0.0  0.0  1.0>  -10.0 END_PLANE
\end{verbatim}

\subsection{Triangles}

\index{triangle!primitive}\keyindex{TRIANGLE}
In order to make more complex objects than the class of quadrics will
permit, a new primitive shape for triangles has been added.  There are
two different types of triangles:  flat shaded triangles and smooth
shaded (Phong) triangles.

Flat shaded triangles are defined by listing the three vertices of the 
triangle.  For example:
\begin{verbatim}
     TRIANGLE  < 0.0   20.0  0.0>
               < 20.0  0.0   0.0>
               <-20.0  0.0   0.0>
     END_TRIANGLE
\end{verbatim}

The smooth shaded triangles
\index{triangle!smooth shaded primitive}\keyindex{SMOOTH_TRIANGLE}
use Phong Normal Interpolation to calculate the
surface normal for the triangle.  This makes the triangle appear to be a
smooth curved surface.  In order to define a smooth triangle, however, you
must supply not only the vertices, but also the surface normals at those
vertices.  For example:
\begin{verbatim}
     SMOOTH_TRIANGLE
          {      points             surface normals     }
          <  0.0  30.0  0.0 >    <0.0   0.7071   -0.7071>
          < 40.0 -20.0  0.0 >    <0.0   -0.8664  -0.5   >
          <-50.0 -30.0  0.0 >    <0.0   -0.5     -0.8664>
     END_SMOOTH_TRIANGLE
\end{verbatim}
As with the other shapes, triangles can be translated, rotated, and scaled.

\begin{description}
\item[Note 1:] Triangles cannot be used in CSG {\tt INTERSECTION}
or {\tt DIFFERENCE} types (described next) since triangles have no
clear ``inside''.\index{triangle!and CSG}
The CSG {\tt UNION} type works acceptably, but with no real
difference from a {\tt COMPOSITE} made up of {\tt TRIANGLE} primitives.
\item[Note 2:] To be honest, I don't expect mere mortals to work out
the surface normals of triangles.\index{triangle!surface normals}
Ideally, you'd have another program
generate them for you.  See the section on ``Common Questions and
Answers'' for details on how this might be done.
\end{description}

\subsection{Quartic Surfaces}

\index{quartic surfaces}\index{surfaces!quartic}
% use the underscores as usual
\catcode`\_=8
Quartic surfaces are 4th order surfaces, and can be used to describe a large
class of shapes including the torus, the lemniscate, etc.  The general
equation for a quartic equation in three variables is (hold onto your hat):
\begin{displaymath}
\begin{array}{cccccccccccccl}
a_{00}x^{4}      & + & a_{01}x^{3}y  & + &
a_{02}x^{3}z     & + & a_{03}x^{3}   & + &
a_{04}x^{2}y^{2} & + \\
a_{05}x^{2}yz    & + & a_{06}x^{2}y  & + &
a_{07}x^{2}z^{2} & + & a_{08}x^{2}z  & + &
a_{09}x^{2}      & + \\
a_{10}xy^{3}     & + & a_{11}xy^{2}z & + &
a_{12}xy^{2}     & + & a_{13}xyz^{2} & + &
a_{14}xyz        & + \\
a_{15}xy         & + & a_{16}xz^{3}  & + &
a_{17}xz^{2}     & + & a_{18}xz      & + &
a_{19}x          & + \\
a_{20}y^{4}      & + & a_{21}y^{3z}  & + &
a_{22}y^{3}      & + & a_{23}y^{2}z^{2} & + &
a_{24}y^{2}z     & + \\
a_{25}y^{2}      & + & a_{26}yz^{3}  & + &
a_{27}yz^{2}     & + & a_{28}yz      & + &
a_{29}y          & + \\
a_{30}z^{4}      & + & a_{31}z^{3}   & + &
a_{32}z^{2}      & + & a_{33}z       & + &
a_{34}           & = 0
\end{array}
\end{displaymath}
To declare a quartic surface requires that each of the coefficients
($a_{00} \ldots a_{34}$) be placed in order into a single long vector
of 35 terms.

As an example, the following object declaration is for a torus having major
radius 6.3 minor radius 3.5 (Making the maximum width just under 10).
\index{quartic surfaces!first principles declaration of}
\keyindex{QUARTIC}
\begin{verbatim}
     { Torus with major radius sqrt(40),
                  minor radius sqrt(12) }
     OBJECT
      QUARTIC
      < 1.0   0.0   0.0    0.0     2.0
        0.0   0.0   2.0    0.0  -104.0
        0.0   0.0   0.0    0.0     0.0
        0.0   0.0   0.0    0.0     0.0
        1.0   0.0   0.0    2.0     0.0
       56.0   0.0   0.0    0.0     0.0
        1.0   0.0 -104.0   0.0   784.0 >
      END_QUARTIC
      BOUNDED_BY
       SPHERE <0 0 0> 10 END_SPHERE
      END_BOUND
     END_OBJECT
\end{verbatim}
The code to calculate Ray-Surface intersections for quartics is somewhat
slower than that for intersections with quadric surfaces.  Benchmarks on
a stock 8Mhz AT (with FPU) give results of around 1400 solutions per second
for 2nd order equations (quadrics) vs 444 per second for 3rd order equations
and 123 per second for 4th order equations (quartics).  So clever use of
bounding shapes can make a big difference in processing
time.\index{processing speed}\index{speed|see{processing speed}}

While a great deal of time has gone into debugging the code for handling
quartic surfaces, I know for a fact that there are bad combinations of
surface equations and lighting orientations that cause math errors (crash).
If this happens to you, as the joke goes, ``then don't do that.''
\index{errors}\index{quartic surfaces!math errors in}

Here are some surfaces to get you started. A Torus can be represented
by the equation:\index{torus}
\begin{eqnarray*}
& & x^{4} + y^{4} + z^{4} + 2x^{2}y^{2} + 2x^{2}z^{2} + 2y^{2}z^{2} - \\
& & 2(r_{0}^{2} + r_{1}^{2})x^{2}
+ 2(r_{0}^{2} - r_{1}^{2})y^{2}
- 2(r_{0}^{2} - r_{1}^{2})z^{2} + \\
& & \hspace*{2.2in}(r_{0}^{2} - r_{1}^{2})^{2} = 0 
\end{eqnarray*}

Where $r_{0}$ is the ``major'' radius of the torus -- the distance
from the hole of the donut to the middle of the ring of the donut, and
$r_{1}$ is the ``minor'' radius of the torus -- the distance from the
middle of the ring of the donut to the outer surface (see Figure
\ref{torus}).
\begin{figure}[htbp]
\begin{centering}
\setlength{\unitlength}{0.0125in}%
\begin{picture}(45,69)(180,705)
\thinlines
\put(200,740){\circle{20}}
\put(200,740){\circle{40}}
\put(220,710){\vector(-4, 3){ 20}}
\put(215,765){\vector(-3,-4){ 15}}
\put(185,740){\vector(-1, 0){  0}}
\put(185,740){\vector( 1, 0){ 30}}
\put(225,705){\makebox(0,0)[lb]{\raisebox{0pt}[0pt][0pt]{\elvrm Minor radius}}}
\put(220,765){\makebox(0,0)[lb]{\raisebox{0pt}[0pt][0pt]{\elvrm Major radius}}}
\end{picture}

\caption{Radii of a torus}
\label{torus}
\end{centering}
\end{figure}
Described
another way, a torus is a circle that is revolved around an axis.  The
major radius is the distance from the axis to the center of the
circle, the minor radius is the radius of the circle.

Note that scaling surfaces like a torus scales everything. In order to change
the relationship between the size of the hole and the size of the ring, it is
necessary to enter new coefficients for the torus.\index{torus!scaling}

The only coefficients of the 35 that need to be filled in
are:
\begin{displaymath}
\begin{array}{ccccccl}
a_{00} & = & a_{20} & = & a_{30} & = & 1 \\
a_{04} & = & a_{07} & = & a_{23} & = & 2 \\
a_{09} & = & a_{32} &   &        & = & -2(r_{0}^{2} + r_{1}^{2}) \\
a_{25} &   &        &   &        & = & 2(r_{0}^{2} - r_{1}^{2}) \\
a_{34} &   &        &   &        & = & 2(r_{0}^{2} - r_{1}^{2})^{2}
\end{array}
\end{displaymath}
The torus can then be rotated and translated into position once its shape
has been defined. See the file {\tt TORUS.DAT} for examples.

A generalization of the lemniscate of Gerono can be represented by the
equation:\index{lemniscate curve}
\begin{displaymath}
c^{2}x^{4} - a^{2}c^{2}x^{2} + y^{2} + z^{2} = 0
\end{displaymath}
where good start values are $a = 1.0$ and $c = 1.0$, giving the
coefficients:
\begin{displaymath}
\begin{array}{ccccccl}
a_{00} & = & a_{25} & = & a_{32} & = & 1 \\
a_{09} &   &        &   &        & = & -1 \\
\end{array}
\end{displaymath}
See the example file {\tt LEMNISCA.DAT} for a more complete example.

A generalization of a piriform can be represented by the
equation:\index{piriform curve}
\begin{displaymath}
-bc^{2}x^{4} - ac^{2}x^{3} + y^{2} + z^{2} = 0
\end{displaymath}
where good start values are $a = 4$, $b = -4$, $c = 1$, giving the
coefficients
\begin{displaymath}
\begin{array}{ccccl}
a_{00} &   &        & = & 4 \\
a_{03} &   &        & = & -4 \\
a_{25} & = & a_{32} & = & 1 \\
\end{array}
\end{displaymath}
See the file {\tt PIRIFORM.DAT} for more on this\ldots
% back to leaving the underscores as they are
\catcode`\_=12

There are really so many different quartic shapes, we can't even begin to
list or describe them all.  If you are interested and mathematically inclied,
an excellent reference book for curves and surfaces where you'll find more
quartic shape formulas is:\index{Seggern@von Seggern, David}
\begin{verse}
{\em The CRC Handbook of Mathematical Curves and Surfaces} \\
David von Seggern, CRC Press, 1990.
\end{verse}

\subsection{Constructive Solid Geometry}

\index{constructive solid geometry}\index{CSG|see{constructive solid geometry}}
This ray tracer supports Constructive Solid Geometry (CSG) in order to
make the object definition abilities more powerful.  Constructive
Solid Geometry allows you to define shapes which are the union,
intersection, or difference of other shapes.  Unions superimpose the
two shapes.  This has the same effect as defining two separate
objects, but is simpler to create and/or manipulate.  Intersections
define the space where the two surfaces meet.  Differences allow you
to cut one object out of another.

CSG Intersections, Unions, and Differences can consist of two or more
shapes.
\index{constructive solid geometry!union}\keyindex{UNION}
\index{constructive solid geometry!difference}\keyindex{DIFFERENCE}
\index{constructive solid geometry!intersection}\keyindex{INTERSECTION}
They are defined as follows:
\begin{verbatim}
     OBJECT
          INTERSECTION
               QUADRIC
                    ...
               END_QUADRIC

               QUADRIC
                    ...
               END_QUADRIC

               QUADRIC
                    ...
               END_QUADRIC
          END_INTERSECTION
          ...
     END_OBJECT
\end{verbatim}
{\tt UNION} or {\tt DIFFERENCE} may be used instead of
{\tt INTERSECTION}.  The order of the shapes doesn't matter except for the
{\tt DIFFERENCE} shapes.  For CSG differences, the first shape is
visible and the remaining shapes are cut out of the first (in reverse
order from version 1.2 {\tt DIFFERENCE}'s).

Constructive solid geometry shapes may be translated, rotated, or scaled in
the same way as a Quadric surface.  The quadric surfaces making up the CSG
object may be individually translated, rotated, and scaled as
well.\index{constructive solid geometry!transforming}

When using CSG, it is often useful to invert an shape so that it's
inside-out.  The {\tt INVERSE}\keyindex{INVERSE} keyword can be
used to do this for any
{\tt SPHERE}, {\tt PLANE}, or {\tt QUADRIC}.  When {\tt INVERSE} is
used, the ``inside'' of the object is flipped to become the
``outside''.  For Planes\index{plane!inside and outside}, ``inside'' is
defined to be in the opposite direction from the ``normal'' or ``up''
direction.

Note that performing an {\tt INTERSECTION} between an shape and some
other {\tt INVERSE} shapes is the same as performing a
{\tt DIFFERENCE}.  In fact, the {\tt DIFFERENCE} is actually implemented in
this way in the code.

\section{Objects}

\index{objects}
There is more to defining an object than just its shape.  You must tell the
ray tracer about the properties of the object like surface colour, alpha,
reflectivity, refractivity, the index of refraction, and so on.  This may be
done by specifying it in the shape or the object.  Generally, an
{\tt OBJECT}\keyindex{OBJECT}
definition will contain two pieces of information about an object: First, what
shape it is, and second, what it's looks like (on the surface and throughout).
A typical object definition looks something like this:
\begin{verbatim}
     OBJECT
          QUADRIC Sphere
               TRANSLATE < 40.0 40.0 60.0 >
          END_QUADRIC

          TEXTURE
             AMBIENT  0.3
             DIFFUSE   0.7
             REFLECTION  0.3
             REFRACTION  0.3
             IOR 1.05
             COLOUR RED 1.0 GREEN 1.0 BLUE 1.0 ALPHA 0.5
          END_TEXTURE
     END_OBJECT
\end{verbatim}

\noindent
The following keywords may be used when defining objects:

\subsection{{\tt COLOUR}}

\index{objects!colour of}\keyindex{COLOUR}
When used on an {\tt OBJECT} (i.e., not inside a {\tt TEXTURE} block),
the {\tt COLOUR} keyword determines what colour to use for a low
quality rendering when {\tt TEXTURE}'s are ignored.  See the
{\tt +q}\optindex{q}
option for details on setting the low-quality option. Examples:
\begin{verbatim}
     COLOUR RED 1.0  BLUE 0.4
\end{verbatim}
or
\begin{verbatim}
     DECLARE Yellow = COLOUR RED 1.0 GREEN 1.0
          ...
     COLOUR Yellow
\end{verbatim}

\subsection{{\tt TRANSLATE}, {\tt ROTATE}, and {\tt SCALE}}

\index{objects!transforming}
Objects can be translated, rotated, and scaled just like surfaces.
If\index{objects!transforming!and textures}
an object is transformed, all textures attached to the object at that
time are transformed as well.  This means that if you have a
{\tt TRANSLATE}\keyindex{TRANSLATE},
{\tt ROTATE}\keyindex{ROTATE},
or {\tt SCALE}\keyindex{SCALE} before a {\tt TEXTURE}, the
texture will {\em not} be transformed.  If the {\tt SCALE},
{\tt TRANSLATE}, or {\tt ROTATE} is after the {\tt TEXTURE}, the texture
will be transformed.

\subsection{{\tt LIGHT_SOURCE}}

\index{objects!as light sources|see{light sources}}
If the {\tt LIGHT_SOURCE}\keyindex{LIGHT_SOURCE} keyword is used
in the definition of an
object, then the object is included in the list of light sources.  It
can light objects and produce shadows.  (You should also specify the
{\tt COLOUR} of the light source, it will usually be ``White'').
Light sources have a peculiar
restriction:\index{light sources!position restriction}
The light source {\em must}
be {\tt TRANSLATE}'d to its final position in the scene, so the normal
way to define a light source is a sphere or quadric centered about the
origin, then {\tt TRANSLATE}'d to where desired in the final scene.
For example:
\begin{verbatim}
     OBJECT
          { we use a sphere, could use a quartic, too }
          SPHERE <0.0  0.0  0.0> 2.0 END_SPHERE
          TRANSLATE <100.0  120.0  40.0>

          LIGHT_SOURCE
          COLOUR White     { colour of the light emitted }
          TEXTURE
             COLOUR White  { surface colour of the sphere }
             AMBIENT 1.0
             DIFFUSE 0.0
          END_TEXTURE
     END_OBJECT
\end{verbatim}

\begin{description}
\item[Note:] \index{light sources!colour}You {\em must}
specify the colour of the light outside the
{\tt TEXTURE} block. This allows the renderer to quickly determine the
colour of the light source without having to plow through the
textures.  Any colour information inside the {\tt TEXTURE} block is
used to render the light source object itself if it is visible in the
scene.  The subtle difference between the actual sphere object and the
light ray emanation point (the center of the sphere) is why
{\tt LIGHT_SOURCE}'s must be defined at {\tt <0,0,0>} then
{\tt TRANSLATE}'d to where you want them.  It ties together and
{\tt TRANSLATE}'s both the object itself and the light ray source point to
the specified point in the scene.  Usually, light sources have an
{\tt AMBIENT} value of 1.0 and a {\tt DIFFUSE} of 0.0, but this is not a
hard and fast rule.
\end{description}

\section{Texture}

\index{textures}
The texture feature is an experiment into functionally based modelling.  This
feature allows you to assign colouring schemes to objects.  Many procedural
surface textures are provided, and by using different colour maps with them,
nearly infinite permutations are possible.  For example, you can make some
object look like wood or marble, etc.  In DKBTrace, any parameter that changes
the appearance of the surface {\em must} be put into a
{\tt TEXTURE}\keyindex{TEXTURE} block.

The basic {\tt TEXTURE} syntax is as follows:
\begin{verbatim}
     TEXTURE
          0.05
          WOOD
          TURBULENCE 0.2
          TRANSLATE < 1.0 2.0 3.0 >
          ROTATE < 0.0 10.0 40.0 >
          SCALE < 10.0 10.0 10.0 >
     END_TEXTURE
\end{verbatim}
Transformations are optional.  They allow you to transform the texture
independent of the object itself.  If you are doing animation, then the
transformations should be the same as the object transformations so that the
texture follows the object through 3-D space.\index{textures!transforming}

The\index{textures!randomness}
floating-point value given immediately following the texture
keyword is an optional ``texture randomness'' value, which causes a
minor random scattering of calculated colour values and produces a
sort of ``dithered'' appearance.  Note this is {\bf bad, bad, bad} for
animations!!  This is the {\em only} ``truly random'' thing in all of
DKB, and will produce a most annoying flicker of flying pixels on any
textures animated with a ``randomness'' value used.

Instead\index{textures!types} of using
{\tt WOOD},\keyindex{WOOD}\index{textures!coloring!wood} you may use
{\tt MARBLE},\keyindex{MARBLE}\index{textures!coloring!marbled}
{\tt BOZO}\keyindex{BOZO},
{\tt CHECKER}\keyindex{CHECKER}, or a handful of other interesting
textures.  The
{\tt WOOD} and {\tt MARBLE} textures are perturbed by a turbulence
function.  This makes them look more random and irregular than they
would normally appear.  The amount of turbulence\index{textures!turbulence}
can be changed by the
{\tt TURBULENCE}\keyindex{TURBULENCE} keyword followed by a number.
Values from 0.1 to 0.3
seem to give the best results.  The default is 0.0, or no turbulence.

Note some of the textures given are coloration textures, such as
{\tt MARBLE}, {\tt WOOD}, {\tt CHECKER}, {\tt GRANITE}, and {\tt AGATE}.
These work with colour maps\index{textures!colour maps}, and have
default ``colour maps'' that
they resort to if none are given.  The rest of the textures available
are ``surface perturbation'' textures, and do not directly affect the
colour of the object, but rather the surface's apparent orientation in
space.  Examples of these are {\tt WAVES}, {\tt RIPPLES}, {\tt DENTS},
{\tt BUMPS}, and {\tt WRINKLES}.  Note\index{textures!combining} that
any given texture may
include up to two actual textures, one coloration and one surface
perturbation choice per texture.  This would allow rippled wood, or
dented granite combinations, etc., but, keep in mind that any texture
transformations applied to one texture (i.e. {\tt TRANSLATE} or
{\tt SCALE}) will also transform the other one in the same fashion.

As of version 2.10, it is possible to create layered\index{textures!layered}
textures. If you use more
that one texture block, the raytracer will compute the colour of the last
texture and if there's any transparency in the colour (i.e., any alpha), it
will mix in some of the colour from the underlying
textures.

\begin{description}
\item[Important Note:] As of version 2.10, the keywords in this
following section {\em cannot} be used outside of a
{\tt TEXTURE}-{\tt END_TEXTURE} structure. This is a change in the
input language from prior versions.
\end{description}

\subsection{Object Surface Lighting Characteristics}

The following object surface lighting characteristics are available.
Each keyword should be followed by a value in the specified range.

\subsubsection{{\tt AMBIENT}}

\keyindex{AMBIENT}\index{objects!surface lighting!ambient}
Ambient light is light that is scattered everywhere in the room.  An object
lit only by ambient light will appear to have the same brightness over the
entire surface.  The default value is very little ambient light (0.3).  The
value can range from 0.0 to 1.0. 

\subsubsection{{\tt DIFFUSE}}

\keyindex{DIFFUSE}\index{objects!surface lighting!diffuse}
Diffuse light is light coming from a light source that is scattered in all
directions.  An object lit only by diffuse light looks like a rubber ball with
a spot light shining on it.  The value can range from 0.0 to 1.0.  By default,
there is mostly diffuse lighting (0.7).

\subsubsection{{\tt BRILLIANCE}}

\keyindex{BRILLIANCE}\index{objects!surface lighting!brilliance}
Objects can be made to appear more metallic by increasing their brilliance.
This controls the tightness of the basic diffuse illumination on objects and
minorly adjusts the appearance of surface shininess.  The default value is
1.0.  Higher values from 3.0 to about 10.0 can give objects a somewhat more
shiny or metallic appearance.  This is best used in concert with either the
{\tt SPECULAR}\keyindex{SPECULAR} or {\tt PHONG}\keyindex{PHONG} highlighting.

\subsubsection{{\tt REFLECTION}}

\keyindex{REFLECTION}\index{objects!surface lighting!reflection}
By setting the reflection value to be non-zero, you can give the object a
mirrored finish.  It will reflect all other objects in the room.  The value
can range from 0.0 to 1.0.  By default there is no reflection. 

\subsubsection{{\tt REFRACTION}}

\keyindex{REFRACTION}\index{objects!surface lighting!refraction}
By setting the refraction value to be non-zero, the object is made transparent.
Light will be refracted through the object like a lens.  The value can be set
between 0.0 and 1.0.  There is no refraction by default. 

\begin{description}
\item[Note 1:] New for
2.10:\index{objects!surface lighting!refraction, and
{\tt ALPHA}}\keyindex{ALPHA}
In order to refraction to work properly, you must have
some {\tt ALPHA} component in the surface colour.  In the places where the
{\tt ALPHA} is high, the refracted light can get through.  In places where
the {\tt ALPHA} is low, the refracted light is suppressed.  This is a
change in the input language from prior versions.
\item[Note 2:] The refracted light is filtered by (takes on) the
surface colour.
\item[Note 3:] In\index{objects!surface lighting!refraction, and {\tt IOR}}
layered textures, the {\tt REFRACTION} and
{\tt IOR} keywords {\em must} be in the
last texture, otherwise they will not take effect.
\item[Note 4:] If a texture has an {\tt ALPHA} component and no value
for {\tt REFRACTION} was
supplied, the renderer will simply transmit the ray through the
surface with no bending.
\end{description}

\subsubsection{{\tt IOR}}

\keyindex{IOR}\index{objects!surface lighting!index of refraction}
If the object is refracting light, then the {\tt IOR} or ``index of
refraction'' should be set.  This determines how dense the object is.
A value of 1.0 will give no refraction.  The index of refraction for
air is 1.0, water is 1.33, glass is 1.5, and diamond is 2.4.

\subsubsection{{\tt PHONG}}

\keyindex{PHONG}\index{objects!surface lighting!phong highlighting}
Controls the amount of Phong Specular Reflection highlighting on the
object.  Causes bright shiny spots on the object, the colour of the
light source that is being reflected.  The size of the spot is defined
by the value given for {\tt PHONGSIZE} below.  {\tt PHONG}'s value is
typically from 0.0 to 1.0, where 1.0 causes complete saturation of the
object's colour to the light source's colour at the brightest area
(center) of the highlight.  There is no {\tt PHONG} highlighting given
by default.

\subsubsection{{\tt PHONGSIZE}}

\keyindex{PHONGSIZE}\index{objects!surface lighting!phong highlighting}
Controls the size of the {\tt PHONG} Highlight on the object, sort of
an arbitrary ``glossiness'' factor.  Values range from 1.0 (Very Dull)
to 100 (Highly Polished).  Default {\tt PHONGSIZE} is 40 (plastic?) if
not specified.  This is simulating the fact that slightly reflective
objects, especially metallic ones, have microscopic facets, some of
which are facing in the mirror direction.  The more that are facing
that way, the shinier the object appears, and the tighter the specular
highlights become.  Phong measures the average of facets facing in the
mirror direction from the light sources to the viewer.

\subsubsection{{\tt SPECULAR}}

\keyindex{SPECULAR}\index{objects!surface lighting!specular reflection}
Very similar to {\tt PHONG} Specular Highlighting, but a better model
is used for determining light ray/object intersection, so a more
credible spreading of the highlights occur near the object horizons,
supposedly.  {\tt PHONG} is thus included for mostly academic reasons,
but try them both and you decide which you like better.  This effect
is most obvious for light sources behind objects.  The size of the
spot is defined by the value given for {\tt ROUGHNESS} below.  Like
{\tt PHONG}, {\tt SPECULAR} values are typically from 0.0 to 1.0 for
full saturation.  Default is no {\tt SPECULAR} highlighting.

Note that Specular and Phong highlights are {\em not} mutually
exclusive.  It is possible to specify both and they will both take
effect.  Normally, however, you will only specify one or the other.

\subsubsection{{\tt ROUGHNESS}}

\keyindex{ROUGHNESS}\index{objects!surface lighting!specular reflection}
Controls the size of the {\tt SPECULAR} Highlight on the object,
relative to the object's ``roughness''.  Values range from 1.0 (Very
Rough) to 0.001 (Very Smooth).  The default value, if not specified,
is 0.05 (Plastic?).  The roughness or average directional distribution
of the microfacets facing in the same direction as the perpendicular
surface ``normal'' cause the most notable reflection of the highlight
to the observer.


\subsubsection{{\tt METALLIC}}

\keyindex{METALLIC}\index{objects!surface lighting!metallic}
This keyword indicates that the colour of the specular and phong
hightlights will be the surface colour instead of the lightsource
colour.  This creates a metallic appearance.  When using this feature,
you should set {\tt AMBIENT} to about 0.5 and set {\tt DIFFUSE} to
0.0. This keyword does not take a value.

\subsection{Surface Colouring Textures}

The first type of advanced procedural textures provided by DKBTrace
are surface coloring textures. These affect the colour of an object.

\subsubsection{{\tt CHECKER}}

\keyindex{CHECKER}\index{textures!coloring!checkered}
{\tt CHECKER} texturing gives a checker-board appearance.  This option
works best on planes.  When using the {\tt CHECKER} texturing, you
must specify two colours immediately following the word {\tt CHECKER}.
These colours are the colours of alternate squares in the checker
pattern.  The default orientation of the {\tt CHECKER} texture is on
an X-Z plane (good for ground work, etc.) but to use it on an object
which has mostly X-Y orientation (such as a sphere, for instance), you
must {\tt ROTATE} the texture. To rotate the {\tt CHECKER} texture
onto an X-Y plane:
\begin{verbatim}
     TEXTURE
          CHECKER COLOUR White COLOUR Red
          SCALE <10.0 10.0 10.0>
          ROTATE <-90.0 0.0 0.0>   { now in the X-Y plane }
     END_TEXTURE
\end{verbatim}

\subsubsection{{\tt CHECKER_TEXTURE}}

\keyindex{CHECKER_TEXTURE}\index{textures!coloring!checkered}
I've had many requests for a checker pattern to allow you to alternate
between wood and marble or any other two textures.  So, in version
2.10, I've added a texture called {\tt CHECKER_TEXTURE} that takes
two textures instead of two colours.

In order to support layered textures, I've made the syntax a bit more verbose
than it would be otherwise.  The syntax is:
\begin{verbatim}
     TEXTURE
        CHECKER_TEXTURE
           TEXTURE ... texture ... END_TEXTURE
           TEXTURE ... optional layers ... END_TEXTURE
        TILE2
           TEXTURE ... second texture ... END_TEXTURE
           TEXTURE ... optional layers ... END_TEXTURE
        END_CHECKER_TEXTURE
        AMBIENT ...
        DIFFUSE ...
    END_TEXTURE
\end{verbatim}
Note that the textures in {\tt CHECKER_TEXTURE} only use the surface
colouring texture information.  Information about {\tt AMBIENT},
{\tt DIFFUSE}, {\tt REFLECTION}, etc.  and surface normal information
({\tt WAVES}, {\tt RIPPLES}) are ignored inside the {\tt CHECKER_TEXTURE}.
(Hey, what do you want for a 10 minute change?)

\subsubsection{{\tt BOZO}}

\keyindex{BOZO}\index{textures!coloring!noisy}
{\tt BOZO} texture basically takes a noise function and maps it onto
the surface of an object.  This ``noise'' is well-defined for every
point in space.  If two points are close together, they will have
noise values that are close together.  If they are far apart, their
noise values will be fairly random relative to each other.

As mentioned above, for coloration textures such as {\tt WOOD},
{\tt MARBLE}, and {\tt BOZO}, etc., you may change the colouring scheme by
using a colour map.\index{textures!colour maps}
This map allows you to convert numbers from 0.0
to 1.0 (which are generated by the ray tracer) into ranges of colours.
For example, the default {\tt BOZO} colouring can be specified by:
\begin{verbatim}
     TEXTURE
          BOZO
          COLOUR_MAP
               [0.0 0.4 COLOUR White COLOUR White]
               [0.4 0.6 COLOUR Green COLOUR Green]
               [0.6 0.8 COLOUR Blue COLOUR Blue]
               [0.8 1.0 COLOUR Red COLOUR Red]
          END_COLOUR_MAP
     END_TEXTURE
\end{verbatim}

The easiest way to see how it works is to try it.  With a good choice of
colours it produces some of the most realistic looking cloudscapes you have
ever seen indoors!  Try a cloud color map such as:
\begin{verbatim}
     TEXTURE
          BOZO
          TURBULENCE 1.0    { A blustery day.  For a calmer }
          COLOUR_MAP        {   one, try 0.2.                }
               { blue to blue }
               [0.0 0.5  COLOUR RED 0.5 GREEN 0.5 BLUE 1.0
                         COLOUR RED 0.5 GREEN 0.5 BLUE 1.0]
               { blue to white }
               [0.5 0.6  COLOUR RED 0.5 GREEN 0.5 BLUE 1.0
                         COLOUR RED 1.0 GREEN 1.0 BLUE 1.0]
               { white to grey }
               [0.6 1.001 COLOUR RED 1.0 GREEN 1.0 BLUE 1.0
                          COLOUR RED 0.5 GREEN 0.5 BLUE 0.5]
          END_COLOUR_MAP
          SCALE <800.0 800.0 800.0>
          TRANSLATE <200.0 400.0 100.0>
     END_TEXTURE
\end{verbatim}
The color map above indicates that for small values of texture, use a sky blue
color solidly until about halfway turbulent, then fade through to white on a
fairly narrow range.  As the white clouds get more turbulent and solid towards
the center, pull the color map toward grey to give them the appearance of
holding water vapor (like typical clouds). 
Check out {\tt sunset.dat} for a really neat (but slow) sky pattern
using {\tt ALPHA}.\keyindex{ALPHA}

\subsubsection{{\tt SPOTTED}}

\keyindex{SPOTTED}\index{textures!coloring!spotted}
Spotted texture is a sort of swirled random spotting of the colour of the
object.  If you've ever seen a metal organ pipe up close you know about what
it looks like (a galvanized garbage can is close). Play with this one, it
might render a decent cloudscape during a very stormy day.  No extra
keywords are required.  Should work with colour maps.  With small scaling
values, looks like masonry or concrete.

\subsubsection{{\tt AGATE}}

\keyindex{AGATE}\index{textures!coloring!marbled}
This texture is similar to {\tt MARBLE}, but uses a different turbulence
function.  The {\tt TURBULENCE} keyword has no effect, and as such it
is always very turbulent.\keyindex{TURBULENCE}

\subsubsection{{\tt GRADIENT}}

\keyindex{GRADIENT}\index{textures!coloring!gradient}
This is a specialized texture that uses approximate local coordinates
of an object to control colour map gradients.  This texture {\em only}
works with colour maps (one {\em must} be given!) and has a special
{\tt <X,Y,Z>} triple given after the {\tt GRADIENT} keyword, which
specifies any (or all) axes to perform the gradient action on.
Example: a Y gradient {\tt <0.0 1.0 0.0>} will give an ``altitude
colour map'', along the Y axis.  Values given other than 0.0 are taken
as 1.0.

For smooth repeating gradients, you should use a ``circular'' colour map, that
is, one in which the first colour value (0.0) is the same as the last one
(1.001) so that it ``wraps around'' and will cause smooth repeating gradient
patterns.  Scaling the texture is normally required to achieve the number of
repeating shade cycles you want.  Transformation of the texture is useful to
prevent a ``mirroring'' effect from either side of the central 0 axes.  Here is
an example of a gradient texture that uses a sharp ``circular'' color mapped
gradient rather than a smooth one, and uses both X and Y gradients to get a
diagonally-oriented gradient.  It produces a dandy candy cane
texture!\index{textures!coloring!candy cane}
\begin{verbatim}
     TEXTURE
          GRADIENT < 1.0 1.0 0.0 >
          COLOUR_MAP
              [0.00 0.25  COLOUR RED 1.0 GREEN 0.0 BLUE 0.0
                          COLOUR RED 1.0 GREEN 0.0 BLUE 0.0]
              [0.25 0.75  COLOUR RED 1.0 GREEN 1.0 BLUE 1.0
                          COLOUR RED 1.0 GREEN 1.0 BLUE 1.0]
              [0.75 1.001 COLOUR RED 1.0 GREEN 0.0 BLUE 0.0
                          COLOUR RED 1.0 GREEN 0.0 BLUE 0.0]
          END_COLOUR_MAP
          SCALE <30.0 30.0 30.0>
          TRANSLATE <30.0 -30.0 0.0>
     END_TEXTURE
\end{verbatim}

You may also specify a {\tt TURBULENCE}\keyindex{TURBULENCE}
value with the gradient to give a more
irregular colour gradient.  This may help to do neat things like fire or
corona effects.

\subsubsection{{\tt GRANITE}}

\keyindex{GRANITE}\index{textures!coloring!granite}
This texture uses a simple $1/f$ fractal noise function to give a pretty darn
good grey granite texture.  Typically used with small scaling values (2.0 to
5.0).  Also looks good with a little dithering (texture randomness).  Should
work with colour maps, so try your hand at pink granite or alabaster! 

\subsection{Surface Perturbation Textures}

The other type of advanced procedural texture provided by DKBTrace are
surface perturbation textures. These affect the 3D appearance of an
object, rather than its colour directly. A surface perturbation
texture can be used in conjunction with the above coloration textures.

\subsubsection{{\tt RIPPLES}}

\keyindex{RIPPLES}\index{textures!perturbation!rippled}
{\tt RIPPLES} makes a surface look like ripples of water.  The
{\tt RIPPLES} option requires a value to determine how deep the ripples
are:
\begin{verbatim}
     TEXTURE
          WOOD
          RIPPLES 0.3
          TRANSLATE < 1.0 2.0 3.0 >
          ROTATE < 0.0 10.0 40.0 >
          SCALE < 10.0 10.0 10.0 >
     END_TEXTURE
\end{verbatim}
In this case, the {\tt WOOD}, {\tt MARBLE}, or {\tt BOZO}, etc.
keywords are optional.  If a different colouring is specified
({\tt WOOD}, {\tt MARBLE}, or {\tt BOZO}), then the
{\tt COLOUR}\keyindex{COLOUR}%
\index{textures!perturbation!{\tt COLOUR} ignored in}
parameter is ignored (except for light sources where it gives the
light colour or when rendering with a low {\tt -q}\optindex{q} option).

\subsubsection{{\tt WAVES}}

\keyindex{WAVES}\index{textures!perturbation!wavy}
Another option that you may want to experiment with is called {\tt
WAVES}.  This works in a similar way to {\tt RIPPLES} except that it
makes waves with different frequencies.  The effect is to make waves
that look more like deep ocean waves. (I haven't done much testing on
{\tt WAVES}, so I can't guarantee that it works very well).

\subsubsection{{\tt PHASE}}

\keyindex{PHASE}\index{textures!perturbation!phase}
Both {\tt WAVES} and {\tt RIPPLES} respond to a texturing option
called {\tt PHASE}.  The {\tt PHASE} option allows you to create
animations in which the water seems to move.  This is done by making
the {\tt PHASE} increment slowly between frames.  The range from 0.0
to 1.0 gives one complete cycle of a wave.

\subsubsection{{\tt FREQUENCY}}

\keyindex{FREQUENCY}\index{textures!perturbation!frequency}
The {\tt WAVES} and {\tt RIPPLES} textures also respond to a keyword
called {\tt FREQUENCY}.  If you increase the frequency of the waves,
they get closer together.  If you decrease it, they get farther apart.


\subsubsection{{\tt BUMPS}}

\keyindex{BUMPS}\index{textures!perturbation!bumpy}
Approximately the same turbulence function as
{\tt SPOTTED}\keyindex{SPOTTED}, but uses
the derived value to perturb the surface normal.  This gives the
impression of a ``bumpy'' surface, random and irregular, sort of like
an orange.  After the {\tt BUMPS} keyword, supply a single floating
point value for the amount of surface perturbation.  Values typically
range from 0.0 (No Bumps) to 1.0 (Extremely Bumpy).  Values beyond 1.0
may do weird things.

\subsubsection{{\tt DENTS}}

\keyindex{DENTS}\index{textures!perturbation!dented}
Interesting when used with
{\tt METALLIC}\keyindex{METALLIC} textures, it gives impressions into
the metal surface that
look like dents.  A single value is supplied after the {\tt DENTS}
keyword to indicate the amount of denting required.  Values range from
0.0 (Showroom New) to 1.0 (Insurance Wreck).  Use larger values at
your own risk.  Scale the texture to make the pitting more or less
frequent.

\subsubsection{{\tt WRINKLES}}

\keyindex{WRINKLES}\index{textures!perturbation!wrinkled}
This is sort of a 3-D (normal perturbing) {\tt GRANITE}\keyindex{GRANITE}.
It uses a
similar $1/f$ fractal noise function to perturb the surface normal in
3-D space.  With {\tt ALPHA}\keyindex{ALPHA} values of greater than
0.0, could look
like wrinkled cellophane.  Requires a single value after the
{\tt WRINKLES} keyword to indicate the amount of wrinkling desired.  Values
from 0.0 (No Wrinkles) to 1.0 (Very Wrinkled) are typical.

\subsection{{\tt IMAGEMAP}}

\keyindex{IMAGEMAP}\index{image mapping}
\index{textures!image mapped|see{image mapping}}
This is a very special coloration texture that allows you to import a
bitmapped file in ``dump'' format (the format output by the ray-tracer),
IFF\index{IFF} format or Compuserve GIF\index{GIF} format and map that
bitmap onto an object.  In the texture of an object, you must give the
{\tt IMAGEMAP} keyword, the format, and a file name.  The syntax is:
\begin{verse}
{\tt IMAGEMAP {\em optional-gradient} DUMP "{\em filename\/}" {\em optional-}ONCE} \\
{\tt IMAGEMAP {\em optional-gradient} IFF "{\em filename\/}" {\em optional-}ONCE} \\
{\tt IMAGEMAP {\em optional-gradient} GIF "{\em filename\/}" {\em optional-}ONCE}
\end{verse}
The texture will then be mapped onto the object as a repeating
pattern.  The optional
{\tt ONCE}\keyindex{ONCE}\index{image mapping!and {\tt ONCE}} keyword
places only one image onto the object
instead of an infinitely repeating tiled pattern.  When {\tt ONCE} is
used, the colour outside the mapped texture is set to transparent.
You can use the layered textures to place other textures or colours
below the image.

In version 2.10 and up, you can specify the
{\tt ALPHA}\keyindex{ALPHA}\index{image mapping!and {\tt ALPHA}}
values for the
colour registers of IFF\index{IFF} or GIF\index{GIF} pictures (at
least for the modes that
use colourmaps).  You can do this by putting the keyword {\tt ALPHA}
immediately following the filename followed by the register value and
transparency.  If the
{\tt ALL}\keyindex{ALL}\index{image mapping!and {\tt ALL}} keyword
is used instead of a register number, then all colours in that
colourmap get that alpha value. For example:
\begin{verbatim}
     IMAGEMAP <1.0 -1.0 0.0> IFF "mypic.iff"
         ALPHA ALL 0.0
         ONCE
\end{verbatim}
or
\begin{verbatim}
     IMAGEMAP <1.0 -1.0 0.0> IFF "mypic.iff"
         ALPHA 0   0.5
         ALPHA 1   1.0
         ALPHA 2   0.3
         ONCE
\end{verbatim}
By default, the image is mapped onto the X-Y plane in the range (0.0,
0.0) to (1.0, 1.0).  If you would like to change this default, you may
use an optional gradient\index{image mapping!gradient}%
\index{gradient|see{image mapping}} {\tt <x y z>} vector after the word
{\tt IMAGEMAP}.  This vector indicates which axes are to be used as the $u$
and $v$ (local surface X-Y) axes.  The vector should contain one
positive number and one negative number to indicate the $u$ and $v$
axes, respectively.  You\index{image mapping!transformation} may
translate, rotate, and scale the texture
to map it onto the object's surface as desired.  Here is an example:
\begin{verbatim}
     INCLUDE "shapes.data"
     OBJECT
          QUADRIC Plane_XY END_QUADRIC
          TRANSLATE <0.0  -20.0  0.0>
          TEXTURE
               { make this texture use the x and z axes
                 for the mapping.                       }
               IMAGEMAP <1.0  0.0  -1.0> GIF "image.gif"
               SCALE <40.0 40.0 40.0>
          END_TEXTURE
     END_OBJECT
\end{verbatim}

Filenames\index{image mapping!filenames}
specified in the {\tt IMAGEMAP} statements will be searched
for in the home (current) directory first, and if not found, will then
be searched for in directories specified by any {\tt -l}\optindex{l}
(library path) options active.  This would facilitate keeping all your
imagemaps ({\tt .dis}, {\tt .gif} or {\tt .iff}) files in a
``textures'' subdirectory, and giving an {\tt -l} option on the
command line to where your library of imagemaps are.

When I was bored with nothing to do, I decided that it would be neat
to have turbulent texture maps.  So, I said -- ``What the hell!''  You
can specify
{\tt TURBULENCE}\keyindex{TURBULENCE}%
\index{image mapping!and {\tt TURBULENCE}}
with texture maps and it will perturb the
image.  It may give some bizarre results.  Is this useful?  I dunno.
It was easy to do so I did it.  Try it out and see what you get.

\section{Composite Objects}

\index{objects!composite}\keyindex{COMPOSITE}
Often it's useful to combine several objects together to act as a whole.  A
car, for example, consists of wheels, doors, a roof, etc.  A composite object
allows you to combine all of these pieces into one object.  This has two
advantages.  It makes it easier to move the object as a whole and it allows
you to speed up the ray tracing by defining bounding shapes that contain the
objects.  (Rays are first tested to see if they intersect the bounding shape.
If not, the entire composite object is ignored).  Composite objects are
defined as follows:
\begin{verbatim}
     COMPOSITE
          OBJECT
               ...
          END_OBJECT
          OBJECT
               ...
          END_OBJECT
          ...
          SCALE < 2.0 2.0 2.0 >
          ROTATE < 30.0 45.0 160.0 >
          TRANSLATE < 100.0 20.0 40.0 >
     END_COMPOSITE
\end{verbatim}
Composite objects can contain other composite objects as well as regular
objects. Composite\index{objects!composite!as light sources}%
\index{light sources!composite objects}
objects cannot be light sources (although any number of
their components can).  This is because it is nearly impossible to determine
the true ``center'' of the composite object, and our light sources are pinpoint
ray sources from the center of the light source object, wherever that may be.

\section{Bounding Shapes}

\index{bounding shapes}
Let's face it.  This program is no speed demon.  You can save yourself a lot
of time, however, if you use bounding shapes around any complex objects.
Bounding shapes tell the ray tracer that the object is totally enclosed by a
simple shape.  When tracing rays, the ray is first tested against the simple
bounding shape.  If it strikes the bounding shape, then the ray is further
tested against the more complicated object inside.

\begin{description}
\item[Note:] Don't use bounding shapes instead of
CSG\index{constructive solid geometry!and bounding shapes} to clip objects.
You will not get the result you want.  For the raytracer to work
properly, you must have the entire object inside the bounding shape.
\end{description}

To use bounding shapes, you simply include the following lines into the
declaration of your {\tt OBJECT} or {\tt COMPOSITE}:
\begin{verse}
{\tt BOUNDED_BY} \\
{\tt \ \ \ \ \ {\em shape}} \\
{\tt END_BOUND}
\end{verse}
An example of a Bounding Shape is:
\begin{verbatim}
     OBJECT
          INTERSECTION
               SPHERE <0.0 0.0 0.0> 2.0 END_SPHERE
               PLANE <0.0 1.0 0.0> 0.0 END_PLANE
               PLANE <1.0 0.0 0.0> 0.0 END_PLANE
          END_INTERSECTION
          BOUNDED_BY
               SPHERE <0.0 0.0 0.0> 2.0 END_SPHERE
          END_BOUND
     END_OBJECT
\end{verbatim}
The best bounding shape is a {\tt SPHERE}\keyindex{SPHERE} since
this shape is highly optimized.\index{processing speed}
Any shape may be used, however, if more convenient.
