\chapter{Getting Started}

This chapter describes what you need to do to jump right in and start
generating images. The next chapter describes the command-line options
in more detail. Here's what you need to do:

\begin{enumerate}
\item Put the file {\tt sunset.dat} into your current directory.
\item Make sure the files {\tt colors.dat}, {\tt shapes.dat}, and
{\tt textures.dat} are present.
\item Set up the default parameters.  This can be done by creating a file
called {\tt trace.def}\index{trace.def@{\tt trace.def} startup file}
or by setting the
{\tt DKBOPT}\index{DKBOPT@{\tt DKBOPT} environment variable}
environment variable.
In either case, use the following line (you may change these defaults
to suit your system):
\begin{quote}
{\tt -w320 -h400 -v +f +d +p +x -a}
\end{quote}
Meaning:
\begin{description}
\item[{\tt -v}] \optindex{v}Don't be verbose, i.e. don't show
line numbers during trace.
\item[{\tt +f}] \optindex{f}Write an output file in ``dump'' or ``Targa'' format (machine
specific; IBM default is ``Targa''\index{Targa}, Amiga and Unix
default is ``dump'').
\item[{\tt +d}] \optindex{d}Display the image while rendering
(on some systems, an
additional character may follow this option to specify the
graphics mode to use for the display).
\item[{\tt -w320}] \optindex{w}Make the image 320 pixels wide.
\item[{\tt -h400}]  \optindex{h}Make the image 400 pixels high.
\item[{\tt +p}]  \optindex{p}Prompt before exitting to let you look at the picture.
\item[{\tt +x}]  \optindex{x}Allow exitting with a key hit before the trace is finished.
\item[{\tt -a}]  \optindex{a}Don't Antialias (Antialiasing smooths out jagged edges).
\end{description}
\item To render a scene, type:
\begin{quote}
{\tt dkb{\em xxx} -isunset.dat -osunset.dis}
\end{quote}
Meaning:
\begin{description}
\item[{\tt dkb{\em xxx}}] On different systems, the name of the
executable may vary.  Check to see what it is on your system.  For the
Amiga, for example, two versions are supplied: {\tt dkb881} and
{\tt dkbieee} for systems with and without a floating-point accelerator.
For the IBM, {\tt DKB.EXE} is the 20286 optimized/uses 80287 math
coprocessor version, and {\tt DKBNO87.EXE} is the plain 8086/no 8087
math coprocessor version.
\item[{\tt -isunset.dat}] \optindex{i}Read the input file {\tt sunset.dat}.
\item[{\tt -osunset.dis}] \optindex{o}Call the output file {\tt sunset.dis}.
This is the usual file name extension for ``dump'' format.
``Targa'' format files (default for IBM's) generally use the extension
{\tt .tga}.
\end{description}
\item Once the image has been rendered, you must use a post-processor to
create the final viewable image file (i.e. an IFF\index{IFF} or
GIF\index{GIF} file), unless
you possess 24-bit display hardware and can view the generated output
files directly.  The post-processor used depends on your system.  See
the section ``Displaying the Images'' for more details on post-processing
the image.
\end{enumerate}
