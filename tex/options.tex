\chapter{Command Line Options}

\newlength{\origitemsep}
\setlength{\origitemsep}{\itemsep}
\newcommand{\noitemsep}{\setlength{\itemsep}{-\parskip}}
\newcommand{\doitemsep}{\setlength{\itemsep}{\origitemsep}}

\index{command-line options}
\index{options,command-line|see{command-line options}}
This program is designed to be run from a command line.  The
command-line options
may be specified in any order.  Repeated options overwrite
the previous values.  Options may also be specified in a file called
{\tt trace.def}\index{trace.def@{\tt trace.def} startup file} or by
the environment variable
{\tt DKBOPT}.\index{DKBOPT@{\tt DKBOPT} environment variable}
\begin{description}
\item[{\tt -w{\em width}}] \optdefindex{w}{width of picture}Width
of the picture in pixels. (On the Amiga, use 320 for full-sized pictures.)

\item[{\tt -h{\em height}}] \optdefindex{h}{height of picture}Height
of the picture in pixels. (On the Amiga, use 400 for full-sized pictures.)

\item[{\tt +v}] \optdefindex{v}{verbosity}Verbose option.
\noitemsep
\item[{\tt -v}] Disable verbose option.
\doitemsep

In verbose mode, the scan line number is printed as each line is traced.

\item[{\tt +f{\em x}}] \optdefindex{f}{output!file generation}Produce
an output file.
\noitemsep
\item[{\tt -f}] Don't produce an output file.
\doitemsep

If the {\tt +f} option is used, the ray tracer will produce an output file
of the picture.  This output file describes each pixel with 24 bits.
Currently, three formats of output files are supported:
\begin{itemize}
\item[{\tt +fd}] Default -- Dump format (QRT-style).
\noitemsep
\item[{\tt +fr}] Raw format -- three files for R, G and B.
\item[{\tt +ft}] Uncompressed Targa-24 format.
\doitemsep
\end{itemize}
Normally, a post-processor is required to create the final finished
image from the data file.  See the section on ``Displaying the Images''
for details.

\item[{\tt +d{\em x}}] \optdefindex{d}{displaying!while tracing}Display
the picture while tracing.
\noitemsep
\item[{\tt -d}] Don't display the picture while tracing.
\doitemsep

If the {\tt +d} option is used, then the picture will be displayed while the
program performs the ray tracing.  On most systems, this picture is
not as good as the one created by the post-processor because it does
not try to make optimum choices for the colour registers.

Depending on the system, a letter may follow the {\tt +d} option to specify
the graphics mode to use.
\begin{description}
\item[All systems:] \mbox{}
\begin{itemize}
\item[{\tt +d}] Default Format (same as {\tt +d0})
\end{itemize}
\item[Amiga:] \mbox{}
\begin{itemize}
\item[{\tt +d0}] Ham format
\item[{\tt +dE}] Ham-E format
\end{itemize}
\item[IBM:] \mbox{}
\begin{itemize}
\item[{\tt +d0}] Autodetect (S)VGA type
\item[{\tt +d1}] Standard VGA 320x200
\item[{\tt +d2}] Simulated SVGA 360x480
\item[{\tt +d3}] Tseng Labs 3000 SVGA 640x480
\item[{\tt +d4}] Tseng Labs 4000 SVGA 640x480
\item[{\tt +d5}] AT\&T VDC600 SVGA 640x400
\item[{\tt +d6}] Oak Technologies SVGA 640x480
\item[{\tt +d7}] Video 7 SVGA 640x480
\item[{\tt +d8}] Video 7 Vega (Cirrus) VGA 360x480
\item[{\tt +d9}] Paradise SVGA 640x480
\item[{\tt +dA}] Ahead Systems Ver. A SVGA 640x480
\item[{\tt +dB}] Ahead Systems Ver. B SVGA 640x480
\item[{\tt +dC}] Chips \& Technologies SVGA 640x480
\item[{\tt +dD}] ATI SGVA 640x480
\item[{\tt +dE}] Everex SVGA 640x480
\item[{\tt +dF}] Trident SVGA 640x480
\item[{\tt +dG}] VESA Standard SVGA Adapter 640x480
\end{itemize}
\end{description}

\item[{\tt +p}] \optdefindex{p}{wait for prompt}Wait for prompt
(IBM: beep and pause) before quitting.
\noitemsep
\item[{\tt -p}] Finish without waiting.
\doitemsep

The {\tt +p} option makes the program wait for a carriage return before
exitting (and closing the graphics screen).  This gives you time to
admire the final picture before destroying it.

\item[{\tt -i{\em filename}}] \optdefindex{i}{input filename}%
\optdefindex{i}{filename!input}Set the input filename.
\noitemsep
\item[{\tt -o{\em filename}}] \optdefindex{o}{output!filename}%
\optdefindex{o}{filename!output}Set the output filename.
\doitemsep

If your input file is not {\tt Object.dat}, then you can use {\tt -i}
to set the filename.  The default output filename will be
{\tt data.dis} for dump mode, {\tt data.red}, {\tt data.grn} or
{\tt data.blu} for raw mode, and {\tt data.tga} for Targa mode.
If you want a different output file name, use the {\tt -o} option.
(On IBM's, the default extensions for raw mode are {\tt .r8},
{\tt .g8}, and {\tt .b8} to conform to PICLAB's ``raw'' format.)

\item[{\tt +a{\em level}}] \optdefindex{a}{anti-aliasing}Anti-alias --
{\em level\/} is an optional tolerance level (default 0.3).
\noitemsep
\item[{\tt -a}] Don't anti-alias.
\doitemsep

The {\tt +a} option enables adaptive anti-aliasing.  The number
after the {\tt +a} option determines the threshold for the
anti-aliasing.  If the colour of a pixel differs from its neighbor
(to the left or above) by more than the threshold, then the
pixel is subdivided and super-sampled.

If the anti-aliasing threshold is 0.0, then every pixel is
supersampled.  If the threshold is 1.0, then no anti-aliasing
is done.  Good values seem to be around 0.2 to 0.4.

The super-samples are jittered to introduce noise and make the
pictures look better.  Note that the jittering "noise" is non-
random and repeatable in nature, based on an object's 3-D
orientation in space.  Thus, it's okay to use anti-aliasing for
animation sequences, as the anti-aliased pixels won't vary and
flicker annoyingly from frame to frame.

\item[{\tt +x}] \optdefindex{x}{exiting early}Allow early exit by
hitting any key.\hfill[IBM only]
\noitemsep
\item[{\tt -x}] Lock in trace until finished.\hfill[IBM only]
\doitemsep

On the IBM, the {\tt -x} option disables the ability to abort the
trace by hitting a key.  If you are unusually clumsy or have
cats that stomp on your keyboard, you
may want to use it.  If you are writing a file, the system
will recognize {\tt Ctrl-C} at the end of line if the system {\tt BREAK}
is {\tt ON}.  If you aren't writing a file, you won't be able to
abort the trace until it's done.

This option was meant for big, long late-nite traces that take
all night (or longer!), and you don't want them interrupted by
anything less important than a natural disaster such as hurricane,
fire, flood, famine, etc.

\item[{\tt -b{\em size}}] \optdefindex{b}{output!buffer size}%
\optdefindex{b}{buffer size}Use an output file buffer of
{\em size} kilobytes.

The {\tt -b} option allows you to assign large buffers to the output
file.  This reduces the amount of time spent writing to the
disk and prevents unnecessary wear (especially for floppies). 
If this parameter is zero (the default), then as each scanline is finished,
the line is written to the file and the file is flushed.  On
most systems, this operation insures that the file is written
to the disk so that in the event of a system crash or other
catastrophic event, at least part of the picture has been
stored properly on disk.

\item[{\tt +c}] \optdefindex{c}{continue rendering}Continue Rendering.

If, for some reason, you abort a raytrace while it's in progress
or if you used the {\tt -e} option (below) to end the raytrace
prematurely, you can use the {\tt +c} option to continue the raytrace
when you get back to it.  This option reads in the previously
generated output file, displays the image to date on the screen,
then proceeds with the raytracing.  In many cases, this feature
can save you a lot of rendering time when things go wrong.\footnote{If
you want to impress your friends with the speed of your
computer, take an image you've already rendered and use {\tt +c}
in the command-line.  It renders {\em real} fast that way!}

\item[{\tt -s{\em line}}] \optdefindex{s}{starting scan line}%
\optdefindex{s}{scan line!starting}Start tracing at line number {\em line}.
\noitemsep
\item[{\tt -e{\em line}}] \optdefindex{e}{ending scan line}%
\optdefindex{e}{scan line!ending}End tracing at line number {\em line}.
\doitemsep

The {\tt -s} option allows you to start rendering an image starting
from a specific scan line.  This is useful for rendering part
of a scene to see what it looks like without having to render
the entire scene from the top.  Alternatively, you can render
groups of scanlines on different systems and concatenate them
later.  WARNING: If you are merging output files from different
systems, make sure that the random number generators are the
same.  If not, the textures from one will not blend in with the
textures from the other.  There is an example of a standard
ANSI-C random number generator in the file {\tt IBM.C}.  Cut it
out and paste it into your machine-specific {\tt .c} file if you
plan to try ``distributed processing'' and are not sure if you
need this standardization.

The {\tt -s} option has no effect when continuing a raytrace using
the {\tt +c} option.  The renderer will figure out where to restart.

\item[{\tt -q{\em level}}] \optdefindex{q}{quality}%
\optdefindex{q}{rendering quality}Rendering quality.

The {\tt -q} option allows you to specify the image rendering quality,
for quickly rendering images for testing.  The {\em level} parameter can
range from 0 to 9.  The values correspond to the following
quality levels:

\begin{itemize}
\item[0,1:] Just show colours.  Ambient lighting only.
\noitemsep
\item[2,3:] Show Diffuse and Ambient light.
\item[4,5:] Render shadows.
\item[6,7:] Create surface textures.
\item[8,9:] Compute reflected, refracted, and transmitted rays.
\end{itemize}
\doitemsep

The default is {\tt -q9} (maximum quality) if not specified.

\item[{\tt -l{\em path}}] \optdefindex{l}{library path}The {\tt -l}
option may be used to specify a ``library'' pathname to
look into for data files to include or for images.  Up to 10
{\tt -l} options may be used to specify a search path.  The home
(current) directory will be searched first followed by the
indicated library directories in order.

\item[{\tt +z}] \optdefindex{z}{debugging}The {\tt +z} option is
an undocumented feature.
You will not see any references to it in this or any other
documentation file for DKBTrace.  In fact, no other section of the
document will even admit that it was mentioned here. If you really
want to know what it does, then you will have to look into the source
code ({\tt trace.c}) and read the comment just above the {\tt +z}
option that says
``Turn on debugging print statements.''  The full purpose of this option
will, therefore, be left as an exercise for the reader, but believe me
-- it's nothing terribly exciting.\footnote{For those people who run
the raytracer on super-fast systems and want to slow it down, you may
try this option.}
\end{description}

You may specify the default parameters by modifying the file
{\tt trace.def}\index{trace.def@{\tt trace.def} startup file}
which contains the parameters in the above format. 
This filename contains a complete command line as though you 
had typed it in, and is processed before any options supplied
on the command line are recognized. You may also set the environment
variable
{\tt DKBOPT}\index{DKBOPT@{\tt DKBOPT} environment variable}
to the desired set of command-line parameters.
