\chapter{Program History and Information}

\section*{Version 1.2}

Initial Amiga release.

\section*{Version 2.0}

\index{Collins, Aaron}\index{Buck, David}
\begin{itemize}
\item First release Version 2.0 Conversion to the IBM done by Aaron A. Collins.
\item New textures, Specular and Phong highlighting added by Aaron A. Collins.
\item Triangle, Smooth Triangle, Sphere, Plane support added by David Buck.
\item RAW, IFF and GIF image mapping added by David Buck and Aaron Collins.
\item Transparency and Fog added by David Buck.
\item GIF format file reader by Steve Bennett\index{Bennett, Steve}
(used with permission).
\item New Noise and DNoise functions by Robert Skinner\index{Skinner, Robert}
(used with permission).
\item TARGA format output file capability added by Aaron A. Collins.
\item ANSI-C function prototyping for ALL functions.
\item Reversal of the order of writing screen data from the original DKB/QRT
``RAW'' file format.
\item For IBM's, it has a crude VGA 320x200 by 256 color display rendering
ability.
\end{itemize}

Version 2.0 compiles under Turbo-C 2.0 on the IBM P.C. and Lattice C
5.05 on the Amiga.  The only file which contains the ANSI extensions
is {\tt dkbproto.h}, so for non-ANSI compilers, you only need to remove the
declaration of the parameters in the {\tt config.h} file and the whole thing
should compile.  There are several example {\tt config.h} files for Amiga,
IBM, and Unix.  The appropriate one should be copied over {\tt CONFIG.H},
and the {\tt MAKEFILE} should be edited for your particular system and
compiler configuration before compilation. 
    
Version 2.0 has a significant difference from prior releases: Speed!
The new primitives of {\tt SPHERE}\keyindex{SPHERE},
{\tt PLANE}\keyindex{PLANE}, {\tt TRIANGLE}\keyindex{TRIANGLE}, etc.
greatly speed up tracing\index{processing speed}. Another
significant speed-up is that world
X-Y-Z values beyond 10 Million or so are ignored, and ray tracing
beyond that distance will cease.  This produces 2 minor peculiarities:

\begin{enumerate}
\item A black stripe at the horizon point of Pre-2.0 scene description
{\tt .data} files that have ``ground'' and ``sky'' planes defined.  The
planes were traced out to a much greater ``infinity'' so this effect was
unnoticeable, prior to version 2.0.
\item Tiny black pixels in the texture, or ``Surface Acne''.
\end{enumerate}

This is usually caused by rays being refracted or reflected such that
the ray does not happen to hit any object, and eventually becomes
black in color as it gets too far away and gets clipped.  This effect
can be minimized by enclosing the scene with distant ``walls'',
``floors'', or placing ``ocean floors'' beneath water, etc.  So far, no
scenes have required placing such planes behind the camera, unless an
``environment map'' of sorts is desired.  See {\tt SKYTEST.DAT} for several
examples of spurious distant planes.  If your ``acne'' still doesn't go
away, it may be due to a large pixel sample area and it's accidentally
picking a point which is just inside the primitive being hit.  This is
a more tricky problem to solve, and anti-aliasing the image will
definitely help if this sort of thing occurs. 

\section*{Version 2.10}

A few unofficial releases were made between 2.01 and 2.10.  The following
points capture the major changes:

\begin{itemize}
\item Less memory is required for image mapped GIF and IFF files.
\item The output format command-line option was changed to
{\tt +f{\em x}}\optindex{f} where {\em x} denotes the output format.
\item The display option {\tt +d}\optindex{d} now takes an optional extra
character {\tt +d{\em x}} where {\em x} is system-dependent.  This
allows you to specify the graphics mode by a command-line switch.
\item The tokenizing pass has been removed.  It's now called directly by the
parser.
\item The environment variable
{\tt DKBOPT}\index{DKBOPT@{\tt DKBOPT} environment variable} is used
in addition to the
{\tt trace.def}\index{trace.def@{\tt trace.def} startup file} file
and the command-line options.
\item The numbers in the data file can now use full scientific
notation, eg. 10.23e-4.
\item The {\tt +c}\optindex{c} option was added to continue an aborted trace.
\item You can now colour or texture each component of the CSG's separately.
\item Layered textures\index{textures!layered} implemented (see the
section on textures).
\item When using GIF and IFF images for image mapping, you can now
indicate that specified registers are partially or completely transparent.
\item Textures are now transformed whenever the object or shape they
are attached to are transformed.
\item The texture {\tt CHECKER_TEXTURE}\keyindex{CHECKER_TEXTURE} has
been added.
\item All keywords relating to the appearance of the surface have been made
illegal in an object definition unless they are inside a {\tt TEXTURE} block.
\item The ``basicshapes.data'' file has been split up into {\tt shapes.dat},
{\tt colors.dat}, and {\tt textures.dat}.  These files have also been expanded
with more useful declarations.
\item The {\tt -l}\optindex{l} command-line option has been
added to support library directories.
\end{itemize}

\section*{Version 2.11}

\begin{itemize}
\item Quartic surfaces\index{quartic surfaces} (4th order) added by
Alexander Enzmann\index{Enzmann, Alexander}.
\item Parser now accepts {\tt Ctrl-Z} as a whitespace.
\item Keyword {\tt END_SMOOTH_TRIANGLE}\keyindex{END_SMOOTH_TRIANGLE}
added (previously, {\tt END_TRIANGLE} was used.)
\end{itemize}

\section*{Version 2.12}

\begin{itemize}
\item Bug in smooth triangles fixed to allow them to be scaled and translated.
\item {\tt METALLIC}\keyindex{METALLIC} texture added.
\end{itemize}
