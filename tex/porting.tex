\chapter{Porting to Different Platforms}

\index{porting}
I've taken great pains to make DKBTrace as portable as possible.  So
far, it's working out fairly well.  For the most part, the core of the
raytracer will compile with any decent C compiler.

If you want to port the raytracer to another system, please try to
modify the core of the raytracer as little as possible.  Each system
will have its own makefile, config file, and C file.  The config file
is included by all of the raytrace modules and can be used to perform
special {\tt \#defines} for that system.  The C file should contain all
system-specific code.  It must implement the following functions:

{\footnotesize
\begin{verbatim}
   void display_init()      /* Open the graphics device and initialize it */
   void display_close()     /* Close the graphics device                  */
   void display_finished()  /* Perform any operations required after      */
                            /* finishing the rendering but before quitting*/
   void display_plot (x, y, Red, Green, Blue)
                            /* Display the specified colour at point x,y  */
\end{verbatim}
}

\noindent
In your config file, you may customize the following things:

{\footnotesize
\begin{verbatim}
   #define FILE_NAME_LENGTH        /* default 150                        */
   #define DBL_FORMAT_STRING       /* the string to use to read a double */
   #define DEFAULT_OUTPUT_FORMAT   /* 'd', 'r' or 't' - default output   */
                                   /*  file format                       */
   #define TEST_ABORT              /* the operation to perform while     */
                                   /* tracing to see if we should abort  */
                                   /* the trace.                         */
   #define RED_RAW_FILE_EXTENSION  /* the default extensions for +fr     */
   #define GREEN_RAW_FILE_EXTENSION
   #define BLUE_RAW_FILE_EXTENSION
   #define STARTUP_DKB_TRACE       /* the code to call immediately after */
                                   /* starting the main program. Useful  */
                                   /* if you don't have a command-line   */
                                   /* interface.                         */
   #define PARAMS(x)               /* "(x)" if you have prototypes, "()" */
                                   /* otherwise.                         */
\end{verbatim}
}
