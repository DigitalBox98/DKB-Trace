\chapter{System Overview}

This program is a ray tracer written completely in C.  It supports
arbitrary quadric surfaces (spheres, ellipsoids, cones, cylinders,
planes, etc.), constructive solid geometry, and various shading models
(reflection, refraction, marble, wood, and many others).  It also has
special-case code to handle spheres, planes, triangles, and smooth
triangles.  By using these special primitives, the rendering can be
done much more quickly than by using the more general quadrics.  In
order to create pictures with this program, you must describe the
objects in the world.  This description is a text file called
{\em filename}.{\tt data}, and {\em filename\/} defaults to {\tt object}
if not specified.  Normally, such files are difficult to write and to
read.  In order to make this task easier, the program contains a
parser to read the data file.  It allows the user to easily create
complex worlds from simple components.  Since the parser allows
include files, the user may put the object descriptions into different
files and combine them all into one final image.

This document is organized as follows. The first chapters describe how
to run DKBTrace, the various options you can give to alter its
behaviour, and provide a ``walk-through'' tutorial to get you familiar
with its concepts and features. The largest chapter then specifies the
scene description language in detail, and serves as a reference manual
for the system. Subsequent chapters discuss issues related to
DKBTrace, such as common questions (with answers), compiling and
porting the system, utilities for use with DKBTrace, and system
history and compatibility notes. Finally, a short bibliography of
ray-tracing references and an index are provided.
