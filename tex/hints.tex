\chapter{Handy Hints}

This chapter provides a list of helpful hints.

\begin{itemize}
\item To see a quick version of your picture, use
{\tt -w64 -h80}\optindex{w}\optindex{h} as command
line parameters on the Amiga.  For the IBM, try {\tt -w80 -h50}.  This
displays the picture in a small rectangle so that you can see how
it will look.

\item Try using the sample default files for different usages --
{\tt QUICK.DEF}
shows a fast postage-stamp rendering (80x50, as above) to the
screen only, {\tt LOCKED.DEF} will display the picture with anti-aliasing
on (takes forever) with no abort (do this before you go to bed...).
The normal default options file
{\tt trace.def}\index{trace.def@{\tt trace.def} startup file} is read
and you can supersede this with another {\tt .def} file by specifying
it on the command line, for example:
\begin{verbatim}
     trace -iworld.dat -oworld.out quick.def
\end{verbatim}

\item When translating light sources\index{light sources!translating},
translate the {\tt OBJECT}\keyindex{OBJECT}, not the
{\tt QUADRIC}\keyindex{QUADRIC} surface.  The light source uses the
center of the object as the origin of the light.

\item When animating\index{textures!animation of} objects with solid
textures, the textures must
move with the object, i.e. apply the same {\tt ROTATE} or
{\tt TRANSLATE} functions to the texture as to the object itself.  This is
now done automatically in version 2.10 if the transformations are
placed {\em after} the {\tt TEXTURE} block.

\item You can declare constants for most of the data types in the program
including floats and vectors.  By combining this with
{\tt INCLUDE}\keyindex{INCLUDE} files,you can easily separate the
parameters for an animation into a separate file.

\item The amount of ambient light plus diffuse light should be less than
or equal to 1.0\index{lighting!total amount of}. The program accepts
any values, but may produce strange results.

\item When using {\tt RIPPLES}\keyindex{RIPPLES}, don't make the
ripples too deep or you may get
strange results (the dreaded ``digital zits''!).

\item {\tt WOOD}\keyindex{WOOD} textures\index{textures!transforming}
usually look better when they are scaled to different
values in $x$, $y$, and $z$, and then rotated to a different angle.

\item You can sort of dither\index{dithering, simulated} a colour
by placing a floating point number into the texture record:
\begin{verbatim}
     TEXTURE
          0.05
     END_TEXTURE
\end{verbatim}
This adds a small random value to the intensity of the diffuse
light on the object.  Don't make the number too big or you may get
strange results.  You generally won't want to use this if you are
rendering frames for animations\index{textures!animation of}, as
this ``dithering'' is random\index{textures!randomness}.
Better results can be obtained, however, by doing proper dithering
in a post-processor.

\item You can compensate for non-square aspect ratios\index{aspect ratio}
on the monitors by making the {\tt RIGHT}\keyindex{RIGHT} vector in
the {\tt VIEWPOINT}\keyindex{VIEWPOINT}\index{viewpoint!aspect ratio}
longer or shorter.  A good value for the Amiga is about 1.333.
This seems okay for IBM's too at 320x200 resolution.  If your spheres
and circles aren't round, try varying it.

\item If you are importing images from other systems, you may find that
the shapes are backwards (left-to-right inverted) and no rotation
can make them right.  All you have to do is negate the terms in the
{\tt RIGHT} vector of the viewpoint to flip the camera left-to-right (use
the ``right-hand'' coordinate system).

\item By making the {\tt DIRECTION}\keyindex{DIRECTION} vector in
the {\tt VIEWPOINT}\keyindex{VIEWPOINT} longer, you can achieve the
effect of a tele-photo\index{tele-photo lens effect} lens.

\item When rendering on the Amiga, use a resolution of 320 by 400 to
create a full sized HAM picture.
\end{itemize}
