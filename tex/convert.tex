\chapter{Converting Data Files from Versions Prior to 2.10}

Unfortunately, version 2.10 changed the format of the data files.
These changes were mostly to avoid confusion -- especially in the case
of layered textures.  There were also some changes that were made to
allow more flexible library objects.  This section details the changes
in the data file format and why it was necessary to make those
changes.  I'm sorry that you have to go back through all your old data
files and edit them all, but I think you'll agree (I hope you'll
agree) that the changes are for the better.

\begin{enumerate}
\item The following keywords are no longer accepted outside of a
{\tt TEXTURE}\keyindex{TEXTURE} block:
\keyindex{AMBIENT}\keyindex{DIFFUSE}\keyindex{BRILLIANCE}\keyindex{REFLECTION}
\keyindex{REFRACTION}\keyindex{IOR}\keyindex{PHONG}\keyindex{PHONGSIZE}
\keyindex{SPECULAR}\keyindex{ROUGHNESS}
\begin{quote}
{\tt AMBIENT}, {\tt DIFFUSE}, {\tt BRILLIANCE}, {\tt REFLECTION},
{\tt REFRACTION}, {\tt IOR}, {\tt PHONG}, {\tt PHONGSIZE},
{\tt SPECULAR}, {\tt ROUGHNESS}
\end{quote}
\begin{description}
\item[Reason:] For layered textures,\index{textures!layered} it's not
clear which texture
these keywords apply to if they're not in a {\tt TEXTURE} block.  I was
really getting messed up in the implementation trying to locate the
proper references for these keywords.  It's much cleaner to only allow
them inside a {\tt TEXTURE} block.
\end{description}

\item A new texture called {\tt COLOUR}\keyindex{COLOUR} (or
{\tt COLOR}) has been added.  This texture simply specifies a simple
colour to use.
\begin{description}
\item[Reason:] It's useful to be able to declare a {\tt TEXTURE}
with the colour embedded inside it.  Besides, I had to provide some
way of specifying a simple colour after I made the following change.
\end{description}

\item The keyword {\tt COLOUR} or {\tt COLOR} when used outside of a
{\tt TEXTURE} block is only used to provide a colour for low-quality
renderings (ones where the {\tt -q}\optindex{q} value is 5 or below.
\begin{description}
\item[Reason:] This is the same reason as for change 1.  For layered
textures, it wasn't clear what the object or shape colour meant when
the texture itself contained colour information.  Rather than have a
convoluted searching scheme (which I tried at first, then abandoned
due to difficulties in explaining it), I decided to keep it simple.
If you want a simple colour, put it in a {\tt TEXTURE} block.
\end{description}

\item When an object or shape is transformed, the textures attached to it are
transformed as well.
\begin{description}
\item[Reason:] Previously, you couldn't {\tt DECLARE}\keyindex{DECLARE}
objects with
textures, then create identical copies of them in different places.
This was especially annoying when you create an object that has an
image mapped onto it.  As soon as you moved the object, the image was
left behind.  Unfortunately, this means that the texture
transformations inside {\tt TEXTURE} blocks in old data files would double
transform the texture, and must be removed.
\end{description}

\item The interaction between {\tt ALPHA}\keyindex{ALPHA} colours
and {\tt REFRACTION}\keyindex{REFRACTION}
has changed.  Previously, there was no interaction at all.  If you had
a surface that contained an {\tt ALPHA} colour, you could see through
the surface.  If the surface had refraction, you could also see
through.  With the new release, {\tt ALPHA} and {\tt REFRACTION} in
combination tell you how much light is passed through from the inside
of the object.

To make things a bit easier, if an object has an {\tt ALPHA} component
but the {\tt REFRACTION} is 0.0 (or unspecified), the renderer will
simply transmit the ray through the object without any refractive
bending.
\begin{description}
\item[Reason:] It seemed to make sense.  The code for {\tt ALPHA} was
doing the same work as the code for {\tt REFRACTION} except for
actually bending the ray.  Now, you can create objects that are
partially opaque and partially transparent. Where they are
transparent, the light passing through the object is bent by the index
of refraction.  Makes sense, no?
\end{description}

\item When you use an {\tt IMAGEMAP}\keyindex{IMAGEMAP}
with the {\tt ONCE}\keyindex{ONCE}\index{image mapping!and {\tt ONCE}}
option, the colour outside the mapped image is transparent
{\tt (RED 1.0 GREEN 1.0 BLUE 1.0 ALPHA 1.0)}.
\begin{description}
\item[Reason:] This allows you to see the underlying textures.
\end{description}

\item I've removed {\tt BasicShapes.data} and replaced it with
{\tt shapes.dat}, {\tt colors.dat}, and {\tt textures.dat}.
\begin{description}
\item[Reason:] The old name didn't apply any more (since it contained
more than just shapes).  Also, I was tired of the IBM people using the
name {\tt BASICSHA.DAT}.
\end{description}

\item The textures in {\tt textures.dat} (formerly
{\tt BasicShapes.data}) were previously scaled to
{\tt <10.0 10.0 10.0>}.  This scaling has been removed.
\begin{description}
\item[Reason:] The factor of 10.0 seemed to be a totally arbitrary
scale factor that someone used.  I'd like to keep the scaling factors
up to the users.
\end{description}
\end{enumerate}
