\chapter{DKBTrace Utilities}

\index{utilities}
In many cases, creating data files for DKBTrace is difficult and tedious.  To
help remedy this problem, I and various other people have developed some
utilities to create data files.  These utilities are described below.

As well, there are some utilities that perform operations on the image files
created by DKBTrace.  These utilities convert between various formats and
allow you to modify or merge output files together.

I'd like to thank all the people who wrote these utilities and sent them to
me.  If anybody else comes up with other utilities, please let me know and
I'll include them in the distribution.

Some of these utilities are written in BASIC in IBM systems.  As such, they
are not easily portable from system to system.  If anyone wants to convert
them to C, let me know and I'll post the C versions.

\section{Data File Creation Utilities}

The data creation utilities fall into two categories:  Those that convert from
some other format into DKB format, and those that generate DKB files using
algorithmic techniques.  These utilities are described below.

\subsection{{\tt SA2DKB}}

\ttindex{SA2DKB}
This program converts Sculpt-Animate 3D and 4D data files into DKB
format.  It currently only supports the basic triangles and textures.
It doesn't support smooth triangles (it treats them like normal
triangles), light sources, cameras, or floors.  (This utility was
formerly called {\tt Sculpt2DKB} but the IBM systems out there kept
calling it {\tt SCULPT2D}, then couldn't figure out what a 2D program
had to do with raytracing or what the nonexistent Amiga program called
{\tt Sculpt-2D} was.

\subsection{{\tt DXF2DKB}}

\ttindex{DXF2DKB}\index{Collins, Aaron}
This utility converts AutoCAD DXF (Drawing eXchange Format) files into
DKBTrace format scene description files.  It was written by Aaron
Collins.  It does not support all of the DXF primitives, but will
suffice for simple objects and scenes after {\tt EXPLODE}'ing and
{\tt DXFOUT}'ing then in {\em AutoCAD}\emindex{AutoCad}.

\subsection{{\tt ShellGen}}

\ttindex{ShellGen}\index{Farmer, Dan}\index{Pickover, Clifford}
{\tt ShellGen} is a BASIC program written by Dan Farmer.  It's based
on a short code fragment from Clifford Pickover's book
{\em Computers, Pattern, Chaos, and Beauty} (St. Martin's Press).
This code fragment was reprinted in {\em Ray Tracing News}, issue 3.3.

As far as I know, the BASIC program only works on IBM's.  It does,
however, allow you to change the parameters and see a quick outline of
what the result will look like.  For those people without IBM's, I've
changed the original code fragment to at least output a DKB-format
file.  No user interface has been provided, however.

\subsection{{\tt Twister}}

\ttindex{Twister}\index{Wells, Drew}
{\tt Twister} is a C program written by Drew Wells (CIS {\tt 73767,1244}).
It creates data files for twisted shapes.  The program
uses a text interface and prompts the user with a question/answer
format.

\subsection{{\tt Chem2DKB}}

\ttindex{Chem2DKB}\index{Farmer, Dan}\index{Puhl, Larry}
{\tt Chem2DKB} is an IBM BASIC program written by Dan Farmer.  It takes models
generated by the {\tt CHEM.EXE} program written by Larry Puhl.

\subsection{{\tt Lissajou}}

\ttindex{Lissajou}\index{Farmer, Dan}\index{Pickover, Clifford}
This is an IBM BASIC program written by Dan Farmer.  It creates data
files for lissajous figures.  The basic algorithms were from Clifford
Pickover.  See {\em Scientific American}, January 1991 and {\em Omni},
February 1990 for examples.

\section{Output File Manipulation Utilities}

These utilities perform some useful manipulations on the dump format
and Targa\index{Targa}
format output files from DKBTrace.  I'd like to thank the people who wrote
these utilities and provided them for general distribution.

\subsection{{\tt dump2i24}}

\ttindex{dump2i24}\index{Rasmussen, Helge E.}
Also known as {\tt DumpToIFF24}, this program was written by Helge
E.\ Rasmussen ({\tt her@compel.dk}).  It converts the dump format files
produced by DKBTrace into 24-bit IFF\index{IFF} format files.  These files can
then be read by a variety of programs including
{\em The Art Department} by ASDG.

\subsection{{\tt catdump}}

\ttindex{catdump}\index{Saari, Ville}
This utility was written by Ville Saari ({\tt vsaari@niksula.hut.fi},
and copyright by the Ferry Island Pixelboys.)  It takes two or more
partially rendered files in DKBTrace's dump format and merges them
into one file.  This is useful for all sorts of things like rendering
different parts on different computers and combining the results.

\begin{description}
\item[NOTE:] Be careful if you combine pictures produced on
different systems.  If the random number generator works differently
between the two systems, the textures may look completely different
from one another.  So long as you use the same executable, you should
be fine.
\end{description}

\subsection{{\tt combdump}}

\ttindex{compdump}\index{Saari, Ville}
This utility was also written by Ville Saari.  It takes two images generated
with DKBTrace with slightly different viewpoints, and creates one dump-format
image file to be viewed with Red-Blue or Red-Green 3D glasses.  The program
allows you to compensate for the exact filtering characteristics of your
glasses to get the best possible result.

\subsection{{\tt dump2mtv}}

\ttindex{dump2mtv}\index{Saari, Ville}
This is yet another utility written by Ville Saari.  This one converts
DKBTrace dump format files onto MTV format used by the MTV and RayShade
raytracers.

\subsection{{\tt dump2raw}}

\ttindex{dump2raw}\index{Collins, Aaron}
The {\tt dump2raw} utility was written by Aaron Collins to convert the
dump format output of DKBTrace into three separate files for red,
green, and blue.  On the IBM, the extensions for these files are
{\tt r8}, {\tt g8}, and {\tt b8}.  On the other systems, they are
{\tt red}, {\tt grn} and {\tt blu}.

Version 2.10 of the raytrace allows you to use the {\tt +fr}\optindex{f}
option to output raw format files directly without the need for
a conversion program like this.

\subsection{{\tt halftga}}

\ttindex{halftga}\index{Collins, Aaron}
The {\tt halftga} utility (written by Aaron Collins) shrinks a
Targa\index{Targa} file to exactly half its original size.  This file can
then be converted into a GIF\index{GIF} image and used in an
{\tt IMAGE_MAP}\keyindex{IMAGE_MAP}
statement.  For systems with little memory available for imagemaps,
this command can be a life-saver.

\subsection{{\tt gluetga}}

\ttindex{gluetga}\index{Collins, Aaron}
This utility (by Aaron Collins) is similar to {\tt catdump} but works
for Targa\index{Targa} format files.  It takes several
partially-rendered Targa\index{Targa} files and glues them together
into one image. 

\subsection{{\tt tga2dump}}

\ttindex{tga2dump}\index{Collins, Aaron}
This utility was written by Aaron Collins.  It converts Targa\index{Targa}
format 16, 24, and 32 bit images into DKB's dump format for use in
image-mapping.

\section{Animation Utilities}

One of the most frequent questions I'm asked is whether or not DKBTrace
supports animation.  The answer is no, not directly.  However, I have made
some changes to the program to provide frame-to-frame consistency so you can
use it for animation if you want to.  The problem, then, is creating the data
files for each individual frame.  That's what this section is all about.

\subsection{RayScene}

\index{RayScene}\index{Kuhkunen@K\^{u}hk\"{u}nen, Jari}\index{Hassi, Paul}
Although RayScene is not being distributed with this raytracer, I thought I'd
at least mention it and tell you where you can get it.  RayScene is a program
that creates data files for DKBTrace based on a high-level (higher-level?)
description of the motion of the camera and the objects.  It was written by
Jari K\^{u}hk\"{u}nen ({\tt hole@tolsun.oulu.fi}) and
Panu Hassi ({\tt oldfox@tolsun.oulu.fi}) and
is available by anonymous FTP from {\tt tolsun.oulu.fi} (128.214.5.6) in the
directory {\tt /pub/rayscene}
or from {\tt iear.arts.rpi.edu} in the directory
{\tt /pub/graphics/ray/rayscene}.
This explanation of RayScene was sent to me by Panu Hassi:
\begin{quotation}
I've tried animation with DBW before DKBTrace2.0 was released.  
The procedure was this: First I wrote the first scene file, copied it 
for {\tt NUMBER_OF_FRAMES} times and then edited some parts of those files 
to create movement etc.  If something went wrong (I accidentally edited 
wrong value etc), I had to edit all those scene files again to make the
changes.  Not so nice if there are 100 files to edit\ldots

So a friend of mine, Jari K\^{u}hk\"{u}nen, and I decided to write RayScene to
make that process even a little easier.  With RayScene the process
goes like this:  you create a scene file and mark the places that 
should be changed with a variable, like:
\begin{verbatim}
    BOUNDED_BY                              
         SPHERE <0.0 0.0 0.0> #sphere_size# END_SPHERE
    END_BOUND
\end{verbatim}

Then you create another file where the values for these variables are
listed.  Rayscene simply creates N scene files inserting current value of
each variable to proper place.  That's all.
    
We have included couple of simple utilities that help with creating
those variable values, but the original scene files are still created
``manually''.  Still, the results have been really nice.  There are 
several animations for Amiga and PC in {\tt tolsun.oulu.fi}.
\end{quotation}
